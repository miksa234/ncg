\documentclass[a4paper]{article}

\usepackage[T1]{fontenc}
\usepackage[utf8]{inputenc}

\usepackage{mathptmx}

\usepackage{subcaption}
\usepackage[shortlabels]{enumitem}
\usepackage{amsmath,amssymb}
\usepackage{amsthm}
\usepackage{bbm}
\usepackage{graphicx}
\usepackage[colorlinks=true,naturalnames=true,plainpages=false,pdfpagelabels=true]{hyperref}
\usepackage[parfill]{parskip}

\theoremstyle{definition}
\newtheorem{definition}{Definition}

\theoremstyle{definition}
\newtheorem{question}{Question}

\theoremstyle{definition}
\newtheorem{example}{Example}

\theoremstyle{theorem}
\newtheorem{theorem}{Theorem}

\theoremstyle{theorem}
\newtheorem{exercise}{Exercise}

\theoremstyle{theorem}
\newtheorem{lemma}{Lemma}

\theoremstyle{definition}
\newtheorem{solution}{Solution}

\newtheorem*{idea}{Proof Idea}


\title{Notes on \\ Noncommutative Geometry and Particle Physics}
\author{Popovic Milutin}
\date{Week 2: 12.02 - 19.02}

\begin{document}

\maketitle
\tableofcontents

\section{Noncommutative Geometric Spaces}
\subsection{Noncommutative Matrix Algebras}
\subsubsection{Balanced Tensor Product and Hilbert Bimodules}

\begin{definition}
    Let $A$ be an algebra, $E$ be a \textit{right} $A$-module and $F$ be a \textit{left} $A$-module.
    The \textit{balanced tensor product} of $E$ and $F$ forms a $A$-bimodule.
    \begin{align*}
        E \otimes _A F := E \otimes F /\left\{\sum _i e_i a_i \otimes f_i - e_i \otimes a_i f_i : \;\;\;
                                         a_i \in A,\ e_i \in E,\ f_i \in F \right\}
    \end{align*}
\end{definition}
In other words the balanced tensor product forms only elements of
\begin{itemize}
    \item $E$ that preserver the \textit{left} representation of $A$ and
    \item $F$ that preserver the \textit{right} representation of $A$.
\end{itemize}
Which is the same saying:
\begin{align*}
    E \otimes _A F = \left\{e a\otimes _A f = e \otimes _A a f: \;\;\; a \in A,\ e \in E,\ f \in F \right\}
\end{align*}

\begin{definition}
    Let $A$, $B$ be \textit{matrix algebras}. The \textit{Hilbert bimodule} for $(A, B)$ is given by
    \begin{itemize}
        \item $E$, an $A$-$B$-bimodue $E$ and by
        \item an $B$-valued \textit{inner product} $\langle \cdot,\cdot\rangle_E: E\times E \rightarrow B$
    \end{itemize}
$\langle \cdot,\cdot\rangle_E$ needs to satisfy the following for $e, e_1, e_2 \in E,\ a \in A$ and $b \in B$.
\begin{align*}
    \langle e_1, a\cdot e_2\rangle_E &= \langle a^*\cdot e_1, e_2\rangle_E \;\;\;\; & \text{sesquilinear in $A$}\\
    \langle e_1, e_2 \cdot b\rangle_E &= \langle e_1, e_2\rangle_E b \;\;\;\; & \text{scalar in $B$} \\
    \langle e_1, e_2\rangle_E &= \langle e_2,e_1\rangle^*_E \;\;\;\; & \text{hermitian} \\
    \langle e, e\rangle_E &\ge 0 \;\;\;\; & \text{equality holds iff $e=0$}
\end{align*}

\end{definition}

We denote $KK_f(A,B)$ the set of all \textit{Hilbert bimodules} of $(A,B)$.

\begin{exercise}
    Check that a representation $\pi:\ A \ \rightarrow L(H)$ of a matrix algebra $A$ turns $H$ into
    a Hilbert bimodule for $(A, \mathbb{C})$.
\end{exercise}

\begin{solution}
\end{solution}

\begin{exercise}
    Show that the $A-A$ bimodule given by $A$ is in $KK_f(A,A)$ by taking the following inner product
    $\langle \cdot,\cdot\rangle_A:A \times A \rightarrow A$:
    \begin{align*}
        \langle a, a\rangle_A = a^*a' \;\;\;\; a,a'\in A
    \end{align*}
    \label{exercise: inner-product}
\end{exercise}
\begin{solution}
\end{solution}

\begin{example}
    Consider a $*$ homomorphism between two matrix algebras $\phi:A\rightarrow B$.
    From it we can construct a Hilbert bimodule $E_{\phi} \in KK_f(A, B)$ in the following way.
    We let $E_{\phi}$ be $B$ in the vector space sense and an inner product from the above
    Exercise \ref{exercise: inner-product}, with $A$ acting on the left with $\phi$.
    \begin{align*}
        a\cdot b = \phi(a)b \;\;\;\; a\in A, b\in E_{\phi}
    \end{align*}
\end{example}



\subsubsection{Kasparov Product and Morita Equivalence}
\begin{definition}
    Let $E \in KK_f(A, B)$ and $F \in KK_F(B, D)$ the \textit{Kasparov product} is defined as
    with the balanced tensor product
    \begin{align*}
        F \circ E := E \otimes _B F
    \end{align*}
    Such that $F\circ E \in KK_f(A,D)$ with a $D$-valued inner product.
    \begin{align*}
        <e_1 \otimes f_1, e_2 \otimes f_2>_{E\otimes _B F} = <f_1,<e_1, e_2>_E f_2>_F
    \end{align*}
\end{definition}

\begin{question}
What is the meaning of 'associative up to isomorphism? Isomorphism of $F \circ E$ or of $A, B$ or $D$?
\end{question}

\begin{exercise}
    Show that the association $\phi \leadsto E_\phi$ (from the previous Example) is natrual
    in the sense
    \begin{itemize}
        \item $E_{\text{id}_A} \simeq A \in KK_f(A,A)$
        \item for $*$-algebra homomorphism $\phi: A \rightarrow B$ and $\psi: B \rightarrow C$ we have
            an isomorphism
            \begin{align*}
                E_{\psi} \circ E_{\phi}\ \equiv\ E_{\phi} \otimes _B E_{\psi}\ \simeq\
                E_{\psi \circ \phi} \in KK_f(A,C)
            \end{align*}
    \end{itemize}
\end{exercise}

\begin{solution}
\end{solution}

\begin{exercise}
    In the definition of Morita equivalence:
    \begin{itemize}
        \item Check that $E \otimes _B F$ is a $A-D$ bimodule
        \item Check that $<\cdot,\cdot>_{E\oplus _B F}$ defines a $D$ valued inner product
        \item Check that $<a^*(e_1 \otimes f_1), e_2 \otimes f_2>_{E \otimes _B F} = <e_1 \otimes f_1, a(e_2 \otimes f_2)>_{E \otimes _B F}$.
    \end{itemize}
\end{exercise}

\begin{solution}
\end{solution}

\begin{definition}
    Let $A$, $B$ be \textit{matrix algebras}. They are called \textit{Morita eqivalent} if there
    exists an $E \in KK_f(A, B)$ and an $F \in KK_f(B, A)$ such that:
    \begin{align*}
        E \otimes _B F \simeq A \;\;\; \text{and} \;\;\; F \otimes _A E \simeq B
    \end{align*}
    Where $\simeq$ denotes the isomorphism between Hilbert bimodules, note that $A$ or $B$ is a bimodule by
    itself.
\end{definition}

\begin{question}
    Why are $E$ and $F$ each others inverse in the Kasparov Product?
\end{question}

\begin{example}
    \
    \begin{itemize}
        \item Hilber bimodule of $(A,A)$ is $A$
        \item Let $E \in KK_f(A,B)$, we take $E \circ A = A\oplus _A E \simeq E$
        \item we conclude, that $_A A_A$ is the identity in the Kasparov product (up to isomorphism)
    \end{itemize}
\end{example}

\begin{example}
    Let $E = \mathbb{C}^n$, which is a $(M_n(\mathbb{C}), \mathbb{C})$ Hilbert bimodule with the
    standard $\mathbb{C}$ inner product.\\
    On the other hand let $F = \mathbb{C}^n$, which is a $(\mathbb{C}, M_n(\mathbb{C}))$ Hilbert
    bimodule by right matrix multiplication with $M_n(\mathbb{C})$ valued inner product:
    \begin{align*}
        <v_1, v_2>=\bar{v_1}v_2^t \;\; \in M_n(\mathbb{C})
    \end{align*}
    Now we take the Kasparov product of $E$ and $F$:
    \begin{itemize}
        \item $F\circ E\ =\  E\otimes _{\mathbb{C}}F\ \;\;\;\;\;\; \simeq \  M_n(\mathbb{C})$
        \item $E\circ F\ =\ F\otimes _{M_n(\mathbb{C})}E\ \simeq\ \mathbb{C}$
    \end{itemize}
    $M_n(\mathbb{C})$ and $\mathbb{C}$ are Morita equivalent
\end{example}

\begin{theorem}
    Two matrix algebras are Morita Equivalent iff their their Structure spaces
    are isomorphic as discreet spaces (have the same cardinality / same number of elements)
\end{theorem}
\begin{proof}
    Let $A$, $B$ be \textit{Morita equivalent}. So there exists $_A E_B$ and $_B F_A$ with
    \begin{align*}
        E \otimes _B F \simeq A \;\;\; \text{and} \;\;\; F \otimes _A E \simeq B
    \end{align*}
    Consider $[(\pi _B, H)] \in \hat{B}$ than we construct a representation of $A$,
    \begin{align*}
        \pi _A \rightarrow L(E \otimes _B H)\;\;\; \text{with} \;\;\; \pi _A(a) (e \otimes v) = a e \otimes w
    \end{align*}
    \begin{question}
        Is $E \simeq H$ and $F \simeq W$?
    \end{question}
    \textit{vice versa}, consider $[(\pi _A, W)] \in \hat{A}$ we can construct $\pi _B$
    \begin{align*}
        \pi _B: B \rightarrow L(F \otimes _A W) \;\;\; \text{and}\;\;\; \pi _B(b) (f\otimes w) = bf\otimes w
    \end{align*}
    These maps are each others inverses, thus $\hat{A} \simeq \hat{B}$
\end{proof}

\begin{exercise}
    Fill in the gaps in the above proof:
    \begin{itemize}
        \item show that the representation of $\pi _A$ defined is irreducible iff $\pi _B$ is.
        \item Show that the association of the class $[\pi _A]$ to $[\pi _B]$ is independent
            of the choice of representatives $\pi _A$ and $\pi _B$
    \end{itemize}
\end{exercise}

\begin{solution}
\end{solution}

\begin{lemma}
    The matrix algebra $M_n(\mathbb{C})$ has a unique inrreducible representation (up to isomorphism)
    given by the defining representation on $\mathbb{C}^n$.
\end{lemma}
\begin{proof}
    We know $\mathbb{C}^n$ is a irreducible representation of $A= M_n(\mathbb{C})$. Let $H$ be irreducible
    and of dimension $k$, then we define a map
    \begin{align*}
        \phi : A\oplus...\oplus A &\rightarrow H^* \\
        (a_1,...,a_k)             &\mapsto e^1\circ a_1^t+...+e^k\circ a_k^t
    \end{align*}
    With $\{e^1,...,e^k\}$ being the basis of the dual space $H^*$ and $(\circ)$ being the pre-composition
    of elements in $H^*$ and $A$ acting on $H$. This forms a morphism of $M_n(\mathbb{C})$ modules,
    provided a matrix $a \in A$ acts on $H^*$ with $v\mapsto v\circ a^t$ ($v\in H^*$).
    Furthermore this morphism is surjective, thus making the pullback $\phi ^*:H\mapsto (A^k)^*$ injective.
    Now identify $(A^k)^*$ with $A^k$ as a $A$-module and note that
    $A=M_n(\mathbb{C}) \simeq \oplus ^n \mathbb{C}^n$ as a n A module.
    It follows that $H$ is a submodule of $A^k \simeq \oplus ^{nk}\mathbb{C}$. By irreducibility
    $H \simeq \mathbb{C}$.
\end{proof}

\begin{example}
    Consider two matrix algebras $A$, and $B$.
    \begin{align*}
        A = \bigoplus ^N_{i=1} M_{n_i}(\mathbb{C}) \;\;\; B = \bigoplus ^M_{j=1} M_{m_j}(\mathbb{C})
    \end{align*}
    Let $\hat{A} \simeq \hat{B}$ that implies $N=M$ and define $E$ with $A$ acting by block-diagonal
    matrices on the first tensor and B acting in the same way on the second tensor. Define $F$ vice versa.
    \begin{align*}
        E:= \bigoplus _{i=1}^N \mathbb{C}^{n_i} \otimes \mathbb{C}^{m_i} \;\;\;
        F:= \bigoplus _{i=1}^N \mathbb{C}^{m_i} \otimes \mathbb{C}^{n_i}
    \end{align*}
    Then we calculate the Kasparov product.
    \begin{align*}
        E \otimes _B F &\simeq \bigoplus _{i=1}^N (\mathbb{C}^{n_i}\otimes\mathbb{C}^{m_i})
            \otimes _{M_{m_i}(\mathbb{C})} (\mathbb{C}^{m_i}\otimes\mathbb{C}^{n_i}) \\
                       &\simeq \bigoplus _{i=1}^N \mathbb{C}^{n_i}\otimes
                       \left(\mathbb{C}^{m_i}\otimes _{M_{m_i}(\mathbb{C})}\mathbb{C}^{m_i}\right)
                        \oplus \mathbb{C}^{n_i} \\
                       &\simeq \bigoplus _{i=1}^N \mathbb{C}^{m_i}\otimes\mathbb{C}^{n_i} \simeq A
    \end{align*}
    and from $F \otimes _A E \simeq B$.
\end{example}

We conclude that.
\begin{itemize}
    \item There is a duality between finite spaces and Morita equivalence classes of matrix algebras.
    \item By replacing $*$-homomorphism $A\rightarrow B$ with Hilbert bimodules $(A,B)$ we introduce
        a richer structure of morphism between matrix algebras.
\end{itemize}


\end{document}
