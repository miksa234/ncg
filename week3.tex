\documentclass[a4paper]{article}

\usepackage[T1]{fontenc}
\usepackage[utf8]{inputenc}

\usepackage{mathptmx}

\usepackage{subcaption}
\usepackage[shortlabels]{enumitem}
\usepackage{amsmath,amssymb}
\usepackage{amsthm}
\usepackage{bbm}
\usepackage{graphicx}
\usepackage[colorlinks=true,naturalnames=true,plainpages=false,pdfpagelabels=true]{hyperref}
\usepackage[parfill]{parskip}

\theoremstyle{definition}
\newtheorem{definition}{Definition}

\theoremstyle{definition}
\newtheorem{question}{Question}

\theoremstyle{definition}
\newtheorem{example}{Example}

\theoremstyle{theorem}
\newtheorem{theorem}{Theorem}

\theoremstyle{theorem}
\newtheorem{exercise}{Exercise}

\theoremstyle{theorem}
\newtheorem{lemma}{Lemma}

\theoremstyle{definition}
\newtheorem{solution}{Solution}

\newtheorem*{idea}{Proof Idea}


\title{Notes on \\ Noncommutative Geometry and Particle Physics}
\author{Popovic Milutin}
\date{Week 2: 26.02 - 4.03}

\begin{document}

\maketitle
\tableofcontents
\section{Excurse to Group Theory and Lie Groups}
\subsection{Groups and Representations}
    \begin{definition}
        A Group $G$ is a set with a binary operation on $G$ satisfying.
        \begin{enumerate}
        \item $f, g \in G$ we have $fg = h \in G$.
        \item $f(gh) = (fg) h$
        \item  $\exists\ e \in G\ \forall f\in G$ with $ef=fe=f$
        \item $\forall f \in G\ \exists\ f^{-1}\in G$ with $ff^{-1}=f^{-1}f=e$
        \end{enumerate}
    \end{definition}

    \begin{definition}
        A Representation of a Group $G$ is a mapping, $D$ of elements of $G$ onto a set of \textit{linear
        operators} such that:
        \begin{enumerate}
            \item $D(e) = 1$, $1$ is the identity operator in the space on which linear operators act
            \item $D(g_1)D(g_2) = D(g_1g_2)$, the mapping is linear in group the group operation
        \end{enumerate}
    \end{definition}

    Just by looking at symmetries of a Group we can find a nice representation, and if the group is finite we
    can even find a matrix representation (Cheyley's Theorem). We all ready know a lot about linear algebra
    which will then allow us to study these Groups very thoroughly and derive physical properties with
    minimal information.


\subsection{Lie Groups}
    Group elements now depend \textit{smoothly} on a set \textit{continuous parameters} $g(\alpha) \in G$.
    We are looking at continuous symmetries, e.g. a Sphere in $\mathbb{R}^3$ can be rotated in any direction
    without changing. The collection of rotations forms a Lie group because the group elements are smoothly
    differentiable.

\subsubsection{Generators}
    We parameterize $g(\alpha)|_{\alpha=0} = e$ and we assume that near the identity element, the group
    elements can be described by a finite set of elements $\alpha_a$ for $a = 1,..,N$. For a representation
    $D$ of this group, linear operators need to be parametrized the same way:
    \begin{align}
        D(\alpha)|_{\alpha=0} = 1
    \end{align}

    Because of the smoothness and continuity we can Taylor expand a representation near the identity:
    \begin{align}
        D(\alpha) &= 1 + id\alpha_a X_a + \cdots && \\
        \text{with}&\;\; X_a = -i \frac{\partial D(\alpha)}{\partial \alpha_a}\bigg\arrowvert _{\alpha=0}
        && \text{\footnote{Einstein Summation Convention, summation over repeated indices}}
    \end{align}

    We call $X_a$ the \textit{generators of the group}.
    \begin{itemize}
        \item If the parametrization is \textit{parsimonious}\footnote{parsimonious -
            All parameters are needed to distinguish between group elements} then all
            of $X_a$ will be independent.

        \item If the representation is unitary then $X_a$ will be \textit{hermitian}, because of the
            $i$ in the definition.

        \item Sophus Lie showed how to derive generators without representations.
    \end{itemize}

    Now let us go in some fixed infinitesimal direction from the identity.
    \begin{align}
        D(d\alpha) = 1+ id\alpha _a X_a
    \end{align}
    Because of the group property of closure with respect to the group operation we can raise $D(d\alpha)$
    to a large power and still get a group element.
    \begin{align}
        D(\alpha) = \lim_{k\rightarrow \infty}(1+i\frac{\alpha_a X_a}{k})^k = e^{i\alpha_a X_a}
    \end{align}
    This is called the \textit{exponential parameterization}. Looking at the expression we see that
    group elements can be expressed in terms of generators, and generators form a vector space.
    They are often referred to any element in the real linear space spanned by $X_a's$.

\subsubsection{Lie Algebras}
    Let us consider a parameter family of group elements created by one generator $X_a$:
    \begin{align}
        U(\lambda) = e^{i\lambda \alpha _a X_a}
    \end{align}
    We know for that for the same generator the group multiplication is linear meaning:
    \begin{align}
        U(\lambda _1)U(\lambda _2) = U(\lambda_1 + \lambda_2)
    \end{align}
    But if we multiply elements generated by two different generators the general case is
    \begin{align}
        e^{i\alpha_a X_a} e^{i\beta_b X_b} \neq  e^{i (\alpha _a + \beta_b) X_a}
    \end{align}
    Yet because the exponentials are a representation of a group, and a group has closure under
    group operation we know the above needs to be true for some $\delta _a$
    \begin{align}
        e^{i\alpha_a X_a} e^{i\beta_b X_b} = e^{i \delta _a X_a}
    \end{align}
    To further examine the exponent we rewrite the expression and Taylor expand $ln(1+K)$
    to the second of $K = e^{i\alpha_a X_a} e^{i\beta_b X_b} -1$
    \begin{align*}
        i\delta _a X_a =& ln(1 + K) = K - \frac{K^2}{2} + \cdots \\
        \text{and}\;\;\; K =&\ e^{i\alpha_a X_a} e^{i\beta_b X_b} -1 \\
          =&\ (1 + i\alpha _a X_a - \frac{1}{2}(\alpha _a X_a)^2 + \cdots) \\
          \cdot&\ (1 + i\beta _b X_b - \frac{1}{2}(\beta _b X_b)^2 + \cdots) -1 \\
          =&\ i\alpha _a X_a + i\beta _b X_b - \alpha_a X_a \beta _b X_b \\
          -&\ \frac{1}{2}(\alpha _a X_a)^2 - \frac{1}{2}(\beta _b X_b)^2 + \cdots
    \end{align*}
    So:
    \begin{align*}
        i\delta _a X_a =&\ i\alpha _a X_a + i\beta _b X_b - \alpha_a X_a \beta _b X_b \\
          -&\ \frac{1}{2}(\alpha _a X_a)^2 - \frac{1}{2}(\beta _b X_b)^2 \\
          +&\ \frac{1}{2}(\ai\alpha _a X_a + i\beta _b X_b - \alpha_a X_a \beta _b X_b)^2 \\
          =&\ i\alpha _a X_a + i\beta _b X_b - \alpha_a X_a \beta _b X_b \\
          -&\ \frac{1}{2}(\alpha _a X_a)^2 - \frac{1}{2}(\beta _b X_b)^2  \\
          +&\ \frac{1}{2}(\alpha _a X_a)^2 + \frac{1}{2}(\beta _b X_b)^2 \\
          +& \frac{1}{2}\alpha _a X_a \beta _b X_b + \frac{1}{2}\beta _b X_b \alpha _a X_a
    \end{align*}
    Because $X$'s are linear operators $\alpha _a X_a \beta _b X_b \neq \beta _b X_b \alpha _a X_a$.
    These generators form an \textit{algebra under commutation} and we get
    \begin{align*}
        i\delta _a X_a =&\ i\alpha _a X_a + i\beta _b X_b - \alpha_a X_a \beta _b X_b \\
                    -&\ \frac{1}{2}[\alpha _a X_a, \beta _b X_b] + \cdots
    \end{align*}
    Thus rewriting the equation gives us
    \begin{align*}
        [\alpha _a X_a, \beta _b X_b] = -2i(\delta _c -\alpha _c -\beta _c) X_c \cdots \equiv i\gamma _c X_c
    \end{align*}
    Because this is true for all $\alpha$ and $\beta$, and considering the group closure, there exists some
    \textit{real} $f_{abc}$ called the \textit{structure constant} satisfying.
    \begin{equation}
        \gamma _c = \alpha _a \beta _b f_{abc}
    \end{equation}
    Which is the same as.
    \begin{equation}
        [X_a, X_b] = i f_{abc} X_c
    \end{equation}
    This is called the \textit{Lie algebra of a group}
    \newline
    \newline
    So $f$ is antisymmetric because $[A, B] = -[B, A]$, which means $f_{abc} = -f_{bac}$.
    \newline
    And $\delta$ can now be written as
    \begin{equation}
        \delta _a = \alpha _a + \beta _a - \frac{1}{2} \gamma _a \cdots
    \end{equation}
    Just by following the properties of Lie Groups (dependence on parameters and smoothness) in a fixed
    direction near die identity to find physical statements. E.g.
    $[\hat{r}_i, \hat{p}_j] = i \hslash \delta _{ij}$ tells us that we can't know the position
    and the momentum of a particle exactly at a given time.




\end{document}
