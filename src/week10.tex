\documentclass[a4paper]{article}

\usepackage[T1]{fontenc}
\usepackage[utf8]{inputenc}

\usepackage{mathptmx}

\usepackage{subcaption}
\usepackage[shortlabels]{enumitem}
\usepackage{amssymb}
\usepackage{amsthm}
\usepackage{mathtools}
\usepackage{bbm}
\usepackage{graphicx}
\usepackage[colorlinks=true,naturalnames=true,plainpages=false,pdfpagelabels=true]{hyperref}
\usepackage[parfill]{parskip}

\usepackage{tikz}
\usetikzlibrary{patterns,decorations.pathmorphing,positioning}

\usepackage[framemethod=TikZ]{mdframed}

\tikzstyle{titlered} =
    [draw=black, thick, fill=white,%
        text=black, rectangle,
        right, minimum height=.7cm]

\newcounter{exercise}

\renewcommand*\theexercise{Exercise~\arabic{exercise}}

\makeatletter
\mdfdefinestyle{exercisestyle}{%
    outerlinewidth=1em,%
    outerlinecolor=white,%
    leftmargin=-1em,%
    rightmargin=-1em,%
    middlelinewidth=1.2pt,%
    roundcorner=5pt,%
    linecolor=black,%
    backgroundcolor=blue!5,
    innertopmargin=1.2\baselineskip,
    skipabove={\dimexpr0.5\baselineskip+\topskip\relax},
    skipbelow={-1em},
    needspace=3\baselineskip,
    frametitlefont=\sffamily\bfseries,
    settings={\global\stepcounter{exercise}},
    singleextra={%
        \node[titlered,xshift=1cm] at (P-|O) %
            {~\mdf@frametitlefont{\theexercise}~};},%
    firstextra={%
            \node[titlered,xshift=1cm] at (P-|O) %
                    {~\mdf@frametitlefont{\theexercise}~};},
}
\makeatother

\newenvironment{MyExercise}%
{\begin{mdframed}[style=exercisestyle]}{\end{mdframed}}

\theoremstyle{definition}
\newtheorem{definition}{Definition}

\theoremstyle{definition}
\newtheorem{question}{Question}

\theoremstyle{definition}
\newtheorem{example}{Example}

\theoremstyle{theorem}
\newtheorem{theorem}{Theorem}

\theoremstyle{theorem}
\newtheorem{lemma}{Lemma}


\theoremstyle{theorem}
\newtheorem{proposition}{Proposition}

\newtheorem*{idea}{Proof Idea}


\title{University of Vienna\\ Faculty of Physics\\ \vspace{1.25cm}
Notes on\\ Noncommutative Geometry and Particle Phyiscs}
\author{Milutin Popovic \\ Supervisor: Dr. Lisa
Glaser}
\date{Week 8: 8.05 - 18.05}

\begin{document}

    \maketitle
    \tableofcontents
    \newpage


\section{Spectral Action of the Fluctuated Dirac Operator}
\begin{proposition}
    The spectral action of the almost commutative manifold $M$ with $\dim(M)
    =4$ with a fluctuated Dirac operator is.
    \begin{align}
        \text{Tr}(f\frac{D_\omega}{\Lambda}) \sim \int_M \mathcal{L}(g_{\mu\nu},
         B_\mu, \Phi) \sqrt{g}\ d^4x + O(\Lambda^{-1})
    \end{align}
    with
    \begin{align}
        \mathcal{L}(g_{\mu\nu}, B_\mu, \Phi) =
        N\mathcal{L}_M(g_{\mu\nu})
        \mathcal{L}_B(B_\mu)+
        \mathcal{L}_\phi(g_{\mu\nu}, B_\mu, \Phi)
    \end{align}
    where $N=4$ and $\mathcal{L}_M$ is the Lagrangian of the spectral triple
    $(C^\infty(M) , L^2(S), D_M)$
    \begin{align}\label{lagr}
        \mathcal{L}_M(g_{\mu\nu}) := \frac{f_4 \Lambda ^4}{2\pi^2} -
        \frac{f_2 \Lambda^2}{24\pi ^2}s - \frac{f(0)}{320\pi^2} C_{\mu\nu
        \varrho \sigma}C^{\mu\nu \varrho \sigma}.
    \end{align}
    Here $C^{\mu\nu \varrho \sigma}$ is defined in terms of the Riemannian
    curvature tensor $R_{\mu\nu \varrho \sigma}$ and the Ricci tensor
    $R_{\nu\sigma} = g^{\mu\varrho} R_{\mu\nu \varrho\sigma}$.


    Furthermore $\mathcal{L}_B$ describes the kinetic term of the gauge field
    \begin{align}
        \mathcal{L}_B(B_\mu) := \frac{f(0)}{24\pi^2}
        \text{Tr}(F_{\mu\nu}F^{\mu\nu}).
    \end{align}
    Last $\mathcal{L}_\phi$ is the scalar-field Lagrangian with a boundary
    term.
    \begin{align}
        \mathcal{L}_\phi(g_{\mu\nu}, B_\mu, \Phi) :=
        &-\frac{2f_2\Lambda^2}{4\pi^2}\text{Tr}(\Phi^2) + \frac{f(0)}{8\pi^2}
        \text{Tr}(\Phi^4) + \frac{f(0)}{24\pi^2} \Delta(\text{Tr}(\Phi^2))\\
        &+ \frac{f(0)}{48\pi^2}s\text{Tr}(\Phi^2)
        \frac{f(0)}{8\pi^2}\text{Tr}((D_\mu \Phi)(D^\mu \Phi)).
    \end{align}
\end{proposition}
\begin{proof}
     The dimension of our manifold $M$ is $\dim(M) = \text{Tr}(id) =4 $. Let us
     take a $x \in M$, we have an asymtotic expansion of
     $\text{Tr}(f(\frac{D_\omega}{\Lambda}))$ as $\Lambda \rightarrow \infty$
     \begin{align}
         \text{Tr}(f(\frac{D_\omega}{\Lambda})) \simeq& \ 2f_4 \Lambda ^4
         a_0(D_\omega ^2)+ 2f_2\Lambda^2 a_2(D_\omega^2) \\&+ f(0) a_4(D_\omega^4)
         +O(\Lambda^{-1}).
     \end{align}
     Note that the heat kernel coefficients are zero for uneven $k$,
     furthermore they are dependent on the fluctuated Dirac operator
     $D_\omega$. We can rewrite the heat kernel coefficients in terms of $D_M$,
     for the first two we note that $N:= \text{Tr}\mathbbm{1_{H_F}})$
     \begin{align}
         a_0(D_\omega^2) &= Na_0(D_M^2)\\
         a_2(D_\omega^2 &= Na_2(D_M^2) - \frac{1}{4\pi^2}\int_M
         \text{Tr}(\Phi^2)\sqrt{g}d^4x
     \end{align}
     For $a_4$ we need to extend in terms of coefficients of $F$, look week9.pdf
     for the standard version,
     \begin{align}
         &\frac{1}{360}\text{Tr}(60sF)= -\frac{1}{6}S(Ns + 4
         \text{Tr}(\Phi^2))\\
        \nonumber\\
         &F^2 = \frac{1}{16}s^2\otimes 1 + 1\otimes \Phi^4 - \frac{1}{4}
         \gamma^\mu\gamma^\nu \gamma^\varrho\gamma^\sigma F_{\mu\nu}F^{\mu\nu}+\\
         &\;\;\;\;\;\;\;+\gamma^\mu\gamma^\nu\otimes(D_\mu\Phi)(D_\nu
         \Phi)+\frac{1}{2}s\otimes \Phi^2 + \ \text{traceless terms}\\
         \nonumber\\
         &\frac{1}{360}\text{Tr}(180F^2) = \frac{1}{8}s^2N + 2\text{Tr}(\Phi^4)
         + \text{Tr}(F_{\mu\nu}F^{\mu\nu}) +\\
         &\;\;\;\;\;\;\;+2\text{Tr}((D_\mu\Phi)(D^\mu\Phi))
         + s\text{Tr}(\Phi^2)\\
         \nonumber\\
         &\frac{1}{360}\text{Tr}(-60\DeltaF)=
         \frac{1}{6}\Delta(Ns+4\text{Tr}(\Phi^2)).
     \end{align}
     Now for the cross terms of $\Omega_{\mu\nu}^E\Omega^{E\mu\nu}$  the trace
     vanishes because of the anti-symmetric properties of the Riemannian
     Cruvature Tensor
     \begin{align}
         \Omega_{\mu\nu}^E\Omega^{E\mu\nu} = \Omega_{\mu\nu}^S\Omega^{S\mu\nu}
         \otimes 1 - 1\otimes F_{\mu\nu}F^{\mu\nu} + 2i\Omega_{\mu\nu}^S
         \otimes F^{\mu\nu}
     \end{align}
     the trace  of the cross term vanishes because
     \begin{align}
         \text{Tr}(\Omega^{S}_{\mu\nu} = \frac{1}{4}
         R_{\mu\nu\varrho\sigma}\text{Tr}(\gamma^\mu\gamma^\nu) = \frac{1}{4}
         R_{\mu\nu\varrho\sigma}g^{\mu\nu} =0
     \end{align}
     and the trace of the whole term is
     \begin{align}
         \frac{1}{360}\text{Tr}(30\Omega^E_{\mu\nu}\Omega^{E\mu\nu}) =
         \frac{N}{24}R_{\mu\nu\varrho\sigma}R^{\mu\nu\varrho\sigma}
         -\frac{1}{3}\text{Tr}(F_{\mu\nu}F^{\mu\nu}).
     \end{align}
     Plugging the results into $a_4$ and simplifying we can write
     \begin{align}
         a_4(x, D_\omega^4) &= Na_4(x, D_M^2) + \frac{1}{4\pi^2}\bigg(\frac{1}{12} s
         \text{Tr}(\Phi^2) + \frac{1}{2}\text{Tr}(\Phi^4) \\
         &+ \frac{1}{4}
         \text{Tr}((D_\mu\Phi)(D^\mu \Phi)) + \frac{1}{6}
         \Delta\text{Tr}(\Phi^2) + \frac{1}{6}
         \text{Tr}(F_{\mu\nu}F^{\mu\nu})\bigg)
     \end{align}
     The only thing left is to plug in the heat kernel coefficients into the
     heat kernel expansion above.
\end{proof}

\section{Fermionic Action}
\begin{definition}
    The fermionic action is defined by
    \begin{align}
        S_f[\omega, \psi] = (J\tilde{\psi}, D_\omega \tilde{\psi})
    \end{align}
    with $\tilde{\psi} \in H_{cl}^+ := \{\tilde{\psi}: \psi \in H^+\}$.
    $H_{cl}^+$ is the set of Grassmann variables in $H$ in the +1-eigenspace
    of the grading $\gamma$.
\end{definition}

\begin{align}
    M\times F_{ED} := \left(C^\infty(M,\mathbb{C}^2),\ L^2(S)\otimes
    \mathbb{C}^4,\
    D_M\otimes 1 +\gamma _M \otimes D_F;\; J_M\otimes J_F,\ \gamma_M\otimes
    \gamma _F\right)
\end{align}




\end{document}
