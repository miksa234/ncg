\vspace*{\fill}
\begin{abstract}
    Noncommutative geometry is a branch of mathematics that has deep
    connections to applications in physics. From reconstructing the theory of
    electrodynamics with minimal coupling to gravity, to deriving the full
    Lagrangian of the standard model and predicting the Higgs mass.  One of
    the reasons for this is the natural existence of a nontrivial gauge group
    of a mathematical structure called the spectral triple, which encodes
    (classical) geometrical data intro algebraic data. Altogether this thesis
    is based on literature work, mostly from Walter D. Suijlekom's book
    `\textit{Noncommutative Geometry and Particle Physics}' \cite{ncgwalter}.
    We summarize enough information to both establish the basic backbone of
    noncommutative geometry and to further out derive the Lagrangian of
    electrodynamics.
\end{abstract}
\vspace*{\fill}
