\section{Introduction}
Noncommutative geometry is a branch of mathematics that incorporates many
different mathematical fields, e.g. Functional analysis, K-Theory,
Differential Geometry, Representation Theory and many more. The origins can
be dated back to the 1940s where two Russian mathematicians Gelfand and
Naimark proved a theorem that connects (in the sense of duality) (classical)
geometry and algebras. From the beginning it was obvious that noncommutative
geometry has physical applications, explicitly with gauge theories, since a
nontrivial gauge group arises naturally from the main structure of
noncommutative geometry called the spectral triple. We will naturally use
this property to present how to derive the Lagrangian of electrodynamics
\ref{sec:5}, and additionally get a purely gravitational Lagrangian.
In regards to this to get to the action principles in terms of geometrical
invariants, a method called the heat kernel expansion is used.

The aim of this thesis is to give a basic foundation of noncommutative
geometry and to present a physical application which can be derived from this
theory. Additionally we emphasize that this thesis is only literature work,
where chapters \ref{sec:1}-\ref{sec:3} and \ref{sec:5} are from
the work of Walter D. Suijlekom's book \cite{ncgwalter} and chapter
\ref{sec:4} from D.V. Vassilevich's paper \cite{heatkernel}.

\textbf{NOW:CHAPTER OVERVIEW}
