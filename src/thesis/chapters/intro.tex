\section{Introduction}
Noncommutative geometry is a branch of mathematics that incorporates many
different mathematical fields, e.g. Functional analysis, K-Theory,
Differential Geometry, Representation Theory and many more. The origins can
be dated back to the 1940s where two Russian mathematicians Gelfand and
Naimark proved a theorem that connects (in the sense of duality) (classical)
geometry and algebras. From the beginning it was obvious that noncommutative
geometry has physical applications, explicitly with gauge theories. A
nontrivial gauge group arises naturally from the main structure of
noncommutative geometry called the spectral triple. We will naturally use
this property to present how to derive the Lagrangian of electrodynamics
\ref{sec:5}, and additionally get a purely gravitational Lagrangian.
In regards to this, to get to the action principles in terms of geometrical
invariants, a method called the heat kernel expansion is used.

The aim of this thesis is to give a basic foundation of noncommutative
geometry and to present a physical application which can be derived from this
theory. Additionally we emphasize that this thesis is only literature work,
where chapters \ref{sec:1}, \ref{sec:2}, \ref{sec:3}, \ref{sec:5} and
\ref{sec:6} are from the work of Walter D. Suijlekom's book
`\textit{Noncommutative Geometry and Particle Physics}' \cite{ncgwalter} and
chapter \ref{sec:4} from D.V. Vassilevich's paper \cite{heatkernel}.

The prominent structure of noncommutative geometry is the spectral triple.
The most basic form of a spectral triple consists of a unital $C^*$ algebra
$A$ acting on a Hilbertspace $H$. Together with a self-adjoint operator $D$ in
$H$, with specific conditions coinciding with the Dirac operator on
a Riemannian spin$^c$ manifold which square is the Laplacian (up to a scalar
term).

The structure of the thesis is based on first getting the background
knowledge of noncommutative geometry and the heat kernel expansion. Then by
combining this insight we work out the Lagrangian of electrodynamics. Thereby
the first two chapters \ref{sec:1} and \ref{sec:2} go through the basic
version of noncommutative geometry, in the sense of finite discrete spaces,
finite spectral triples. It is important to understand these basics, since
they build up the ground work for constructing the almost commutative
manifold of electrodynamics, that is the Two-Point space $F_X$. Additionally
the notion of equivalence relations between spectral triples, called Morita
equivalence is introduced.

The next chapter \ref{sec:3} extends the finite spectral triple with a real
structure, called the real finite spectral triple, we also examine Morita
equivalence within this extension.

Chapter \ref{sec:4} explains the heat kernel and leads off to the heat kernel
expansion, where the famous heat kernel coefficients arise. Hereof we
calculate the heat kernel coefficients, which become important when
calculating the Lagrangian of the almost commutative manifold of
electrodynamics. We again atone, that this chapter is based on Vassilevich's
paper \cite{heatkernel}.

In the last two chapters \ref{sec:5} and \ref{sec:6} we go over the ideas and
the process of constructing the almost commutative manifold. With this
information we can calculate the action principles corresponding to the
almostcommutative manifold, that will give rise to the Lagrangian of
electrodynamics and an additional purely gravitational Lagrangian.
