\subsection{Excurse}
\textbf{Manifold:} A topological space that is locally Euclidean.
\newline
\textbf{Riemannian Manifold:}A Manifold equipped with a Riemannian
Metric, a
symmetric bilinear form on Vector Fields $\Gamma(TM)$
\begin{align}
    &g: \Gamma(TM) \times \Gamma(TM) \rightarrow C(M) \\
    \text{with}& \nonumber\\
    &g(X, Y) \in \mathbb{R} \;\;\; \text{if $X, Y \in \mathbb{R}$}\\
    &\text{$g$ is $C(M)$-bilinear } \forall f\in C(M):\;\; g(fX, Y) =
    g(X,
    fY) = fg(X,Y)\\
    &g(X,X) \begin{cases}\geq 0  \;\;\; \forall X \\ = 0 \;\;\; \forall X
        =0
    \end{cases}
\end{align}
$g$ on $M$ gives rise to a distance function on $M$
\begin{align}
    d_g(x, y) = \inf_\gamma \left\{\int_0^1(\dot{\gamma}(t),
    \dot{\gamma}(t))dt;\;\; \gamma(0) = x, \gamma(1) = y \right\}
\end{align}
Riemannian Manifold is called spin$^c$ if there exists a vector bundle $S
\rightarrow M$ with an algebra bundle isomorphism
\begin{align}
    \mathbb{C}\text{I}(TM) &\simeq \text{End}(S)\;\;\; &\text{($dim(M)$
    even)}\\
    \mathbb{C}\text{I}(TM)^\circ &\simeq \text{End}(S)\;\;\;
    &\text{($dim(M)$ odd)}\\
\end{align}
$(M,S)$ is called the \textbf{spin$^c$ structure on $M$}.
\newline
$S$ is called the \textbf{spinor Bundle}.
\newline
$\Gamma(S)$ are the \textbf{spinors}.

Riemannian spin$^c$ Manifold is called spin if there exists an
anti-unitary
operator $J_M:\Gamma(S) \rightarrow \Gamma(S)$ such that:
\begin{enumerate}
    \item $J_M$ commutes with the action of real-valued  continuous
        functions
        on $\Gamma(S)$.
    \item $J_M$ commutes with $\text{Cliff}^-(M)$ (even case)\\
    $J_M$ commutes with $\text{Cliff}^-(M)^\circ$ (odd case)
\end{enumerate}
$(S, J_M)$ is called the \textbf{spin Structure on $M$}
\newline
$J_M$ is called the \textbf{charge conjugation}.

\subsection{Operators of Laplace Type}
Let $M$ be a $n$ dimensional compact Riemannian manifold with $\partial M = 0$.
Then consider a vector bundle $V$ over $M$ (i.e. there is a vector space to
each point on $M$), so we can define smooth functions. We want to look at
arbitrary differential operators $D$ of Laplace type on $V$, they have the general
from
\begin{align}
    D = -(g^{\mu\nu} \partial_\mu\partial_\nu + a^\sigma\partial_\sigma +b)
\end{align}
where $a^\sigma, b$ are matrix valued functions on $M$ and $g^{\mu\nu}$ is the
inverse metric on $M$. There is a unique connection on $V$ and a unique
endomorphism (matrix valued function) $E$ on $V$, then we can rewrite $D$ in
terms of $E$ and covariant derivatives
\begin{align}
    D = -(g^{\mu\nu} \nabla_\mu \nabla_\nu +E)
\end{align}
Where the covariant derivative consists of $\nabla = \nabla^{[R]} +\omega$ the
standard Riemannian covariant derivative $\nabla^{[R]}$ and a "gauge" bundle
$\omega$ (fluctuations). WE can write $E$ and $\omega$ in terms of geometrical
identities
\begin{align}
    \omega_\delta &= \frac{1}{2}g_{\nu\delta}(a^\nu
    +g^{\mu\sigma}\Gamma^\nu_{\mu\sigma}I_V)\\
    E &= b - g^{\nu\mu}(\partial_\mu \omega_\nu + \omega_\nu \omega_\mu -
    \omega_\sigma \Gamma^\sigma_{\nu\mu})
\end{align}
where $I_V$ is the identity in $V$ and the Christoffel symbol
\begin{align}
    \Gamma^\sigma_{\mu\nu} = g^{\sigma\varrho} \frac{1}{2} (\partial_\mu
    g_{\nu\varrho} + \partial_\nu g_{\mu\varrho} - \partial_\varrho g_{\mu\nu})
\end{align}
Furthermore we remind ourselves of the Riemmanian curvature tensor, Ricci
Tensor and the Scalar curavture.
\begin{align}
    R^\mu_{\nu\varrho\sigma} &= \partial_\sigma \Gamma^{\mu}_{\nu\varrho}
    -\partial_\varrho \Gamma^\mu_{\nu\sigma}
    \Gamma^{\lambda}_{\nu\varrho}\Gamma^{\mu}_{\lambda\sigma}
    \Gamma^{\lambda}_{\nu\sigma}\Gamma^{\mu}_{\lambda\varrho}\\
    R_{\mu\nu} &:= R^{\sigma}_{\mu\nu\sigma}\\
    R &:= R^\mu_{\ \mu}
\end{align}

The we let $\{e_1, \dots, e_n\}$ be the local orthonormal frame of
$TM$(tangent bundle $M$), which will be noted with flat indices $i,j,k,l
\in\{1,\dots, n\}$, we use $e^k_\mu, e^\nu_j$ to transform between flat indices
and curved indices $\mu, \nu, \varrho$.
\begin{align}
    e^\mu_j e^\nu_k g_{\mu\nu} &= \delta_{jk}\\
    e^\mu_j e^\nu_k \delta^{jk} &= g^{\mu\nu} \\
    e^j_\mu e^\mu_k  &= \delta^j_k
\end{align}

The Riemannian part of the covariant derivative contains the standard
Levi-Civita connection, so that for a $v_\nu$ we write
\begin{align}
    \nabla_\mu^{[R]} v_\nu = \partial_\mu v_\nu -
    \Gamma^{\varrho}_{\mu\nu}v_\varrho.
\end{align}
The extended covariant derivative reads then
\begin{align}
    \nabla_\mu v^j = \partial_\mu v^j + \sigma^{jk}_\mu v_k.
\end{align}
the condition $\nabla_\mu e^k_\nu = 0$ gives us the general connection
\begin{align}
    \sigma^{kl}_\mu = e^\nu_l\Gamma^{\varrho}_{\mu\nu}e^k_\varrho - e^\nu_l
    \partial_\mu e^k_\nu
\end{align}
The we may define the field strength $\Omega_{\mu\nu}$ of the connection $\omega$
\begin{align}
    \Omega_{\mu\nu} = \partial_\mu \omega_\nu -\partial_\nu \omega_\mu
    +\omega_\mu \omega_\nu -\omega_\nu\omega_\mu.
\end{align}
If we apply the covariant derivative on $\Omega$ we get
\begin{align}
    \nabla_\varrho\Omega_{\mu\nu} = \partial_\varrho \Omega_{\mu\nu} -
    \Gamma^{\sigma}_{\varrho \mu} \Omega_{\sigma\mu} + [\omega_\varrho,
    \Omega_{\mu\nu}]
\end{align}
