\subsection{Excurse}
\textbf{Manifold:} A topological space that is locally Euclidean.
\newline
\textbf{Riemannian Manifold:}A Manifold equipped with a Riemannian
Metric, a
symmetric bilinear form on Vector Fields $\Gamma(TM)$
\begin{align}
    &g: \Gamma(TM) \times \Gamma(TM) \rightarrow C(M) \\
    \text{with}& \nonumber\\
    &g(X, Y) \in \mathbb{R} \;\;\; \text{if $X, Y \in \mathbb{R}$}\\
    &\text{$g$ is $C(M)$-bilinear } \forall f\in C(M):\;\; g(fX, Y) =
    g(X,
    fY) = fg(X,Y)\\
    &g(X,X) \begin{cases}\geq 0  \;\;\; \forall X \\ = 0 \;\;\; \forall X
        =0
    \end{cases}
\end{align}
$g$ on $M$ gives rise to a distance function on $M$
\begin{align}
    d_g(x, y) = \inf_\gamma \left\{\int_0^1(\dot{\gamma}(t),
    \dot{\gamma}(t))dt;\;\; \gamma(0) = x, \gamma(1) = y \right\}
\end{align}
Riemannian Manifold is called spin$^c$ if there exists a vector bundle $S
\rightarrow M$ with an algebra bundle isomorphism
\begin{align}
    \mathbb{C}\text{I}(TM) &\simeq \text{End}(S)\;\;\; &\text{($dim(M)$
    even)}\\
    \mathbb{C}\text{I}(TM)^\circ &\simeq \text{End}(S)\;\;\;
    &\text{($dim(M)$ odd)}\\
\end{align}
$(M,S)$ is called the \textbf{spin$^c$ structure on $M$}.
\newline
$S$ is called the \textbf{spinor Bundle}.
\newline
$\Gamma(S)$ are the \textbf{spinors}.

Riemannian spin$^c$ Manifold is called spin if there exists an
anti-unitary
operator $J_M:\Gamma(S) \rightarrow \Gamma(S)$ such that:
\begin{enumerate}
    \item $J_M$ commutes with the action of real-valued  continuous
        functions
        on $\Gamma(S)$.
    \item $J_M$ commutes with $\text{Cliff}^-(M)$ (even case)\\
    $J_M$ commutes with $\text{Cliff}^-(M)^\circ$ (odd case)
\end{enumerate}
$(S, J_M)$ is called the \textbf{spin Structure on $M$}
\newline
$J_M$ is called the \textbf{charge conjugation}.
