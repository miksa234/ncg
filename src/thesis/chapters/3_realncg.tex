\subsection{Finite Real Noncommutative Spaces\label{sec:3}}
\subsubsection{Finite Real Spectral Triples}
In this chapter we supplement the finite spectral triples with a \textit{real
structure}. We additionally require a symmetry condition that that $H$ is an
$A$-$A$-bimodule rather than only a $A$-left module. This ansatz has tight
bounds with physical properties such as charge conjugation, into which we will
dive in deeper in later chapters. In regards to this we will need to set a basis
of definitions to get an overview.
First we introduce a $\mathbb{Z}_2$-grading $\gamma$ with the following
properties
\begin{align}
    \gamma ^* &= \gamma, \\
    \gamma ^2 &= 1, \\
    \gamma D &= - D \gamma,\\
    \gamma a &= a \gamma, \;\;\;\; a\in A.
\end{align}
Then we can define a finite real spectral triple.
\begin{mydefinition}
    A \textit{finite real spectral triple} is given by a finite spectral
    triple $(A, H, D)$ and a anti-unitary operator $J:H\rightarrow H$ called
    the \textit{real structure}, such that
    \begin{align}
        a^\circ := J\ a^*\ J^{-1},
    \end{align}
    is a right representation of $A$ on $H$, that is $(ab)^\circ = b^\circ
    a^\circ$. With two requirements
    \begin{align}
        &[a, b^\circ] = 0,\\
        &[[D, a],\ b^\circ] = 0.
    \end{align}
    The two properties are called the \textit{commutant property}, they
    require that the left action of an element in $A$ and $\Omega _D^1(A)$ commutes with the right
    action on $A$.
\end{mydefinition}
\begin{mydefinition}
    The $KO$-dimension of a real spectral triple is determined by the sings
    $\epsilon, \epsilon ' ,\epsilon '' \in \{-1, 1\}$ appearing in
    \begin{align}
        J^2 &= \epsilon, \\
        J\ D &= \epsilon \ D\ J,\\
        J\ \gamma &= \epsilon''\ \gamma\ J.
    \end{align}
\end{mydefinition}
\begin{table}[h!]
    \centering
    \caption{$KO$-dimension $k$ modulo $8$ of a real spectral triple}
    \begin{tabular}{ c | c c c c c c c c}
        \hline
        $k$        & 0 & 1 & 2 & 3 & 4 & 5 & 6 & 7 \\
        \hline
     $\epsilon$    & 1 & 1 & -1 & -1 & -1 & -1 & 1 & 1 \\
     $\epsilon '$  & 1 & -1 & 1 & 1 & 1 & -1 & 1 & 1 \\
     $\epsilon ''$ & 1 &  & -1 &  & 1 &  & -1 &  \\
     \hline
    \end{tabular}
\end{table}
\noindent
Even thought the KO-dimension of a real spectral triple is important, we will
not be doing in-depth introduction of the KO-dimension, for this we reference
again to \cite{ncgwalter}.

\begin{mydefinition}
An opposite-algebra $A^\circ$ of a $A$ is defined to be equal to $A$ as a
vector space with the opposite product
\begin{align}
    &a\circ b := ba\\
    &\Rightarrow a^\circ = Ja^* J^{-1},
\end{align}
which defines the left representation of $A^\circ$ on $H$
\end{mydefinition}


%------------EXAMPLE EXERCISE
Let us examine an example of a matrix algebra $M_N(\mathbb{C})$ acting on
$H=M_N(\mathbb{C})$ by left matrix multiplication with the Hilbert Schmidt
inner product.
\begin{align}
    \langle a , b \rangle = \text{Tr}(a^* b).
\end{align}
We can define $\gamma (a) = a$ and $J(a) = a^*$ with $a\in H$.  Since $D$
must be odd with respect to $\gamma$ it vanishes identically.  Furthermore we
know the multiplicity space is $V_i = \mathbb{C}^{m_i}$, and also we know
that for $T\in H$ and$a\in A'$ to work we need $a\ T=T\ a$. Thus by laws of
matrix multiplication we need $A' \simeq \bigoplus _i M_{m_i}(\mathbb{C})$. For
this to work we naturally need $H = \bigoplus_i \mathbb{C}^{n_i} \otimes
\mathbb{C}^{m_i}$.  Hence the right action of $M_N(\mathbb{C})$ on $H =
M_N(\mathbb{C})$ as defined by $a \mapsto a^\circ$ is given by right matrix
multiplication
\begin{align}
    a^\circ \xi = J a^* J^{-1}\xi = Ja^* \xi^* = J\xi a=\xi^* a
\end{align}

%------------EXAMPLE EXERCISE

\begin{mydefinition}
    We call $\xi \in H$ \textbf{cyclic vector} in $A$ if:
    \begin{align}
        A\xi := { a\xi:\;\; a\in A} = H
    \end{align}
    We call $\xi \in H$ \textbf{separating vector} in $A$ if:
    \begin{align}
        a\xi = 0\;\; \Rightarrow \;\; a=0;\;\;\; a\in A
    \end{align}
\end{mydefinition}
%------------------- EXERCISE
Suppose $(A, H, D = 0)$ is a finite spectral triple such that $H$ possesses a
cyclic and separating vector for $A$ and let
\begin{align}
    J: H \rightarrow H
\end{align}
be the operator in $S = J \Delta ^{1/2}$ with $\Delta = S^*S$. By composition
$S(a\xi) = a*\xi$ this is literally anti-linearity, then $S(a \xi) = a* \xi$
defines a anti-linear operator. Furthermore the operator $S$ is invertible
because, if a $\xi \in H$ is cyclic then we have $S(A\xi) = A^*\xi = A\xi =
H$. Vice versa the same has to work for $S^{-1}$, otherwise $\xi$ wouldn't
exist. And hence $S^{-1}(A^*\xi) = S^{-1}(H) = H$. Additionally $J$ is
anti-unitary because firstly, $S$ is bijective thus $\Delta ^{1/2}$ and $J$ need to be bijective.
Also have $J = S \Delta^{-1/2}$ and $\Delta^* = \Delta$, so for a $\xi _1 ,
\xi _2 \in H$ we can write
\begin{align}
    \langle J \xi _1 , J \xi _2 \rangle  &= \langle  J^*J\xi_1 , \xi_2\rangle ^* =\nonumber\\
    &= \langle (\Delta ^{-1/2})^* S^* S \Delta ^{-1/2} \xi_1, \xi_2\rangle ^* =\nonumber \\
    &= \langle (\Delta^{-1/2})^* \Delta \Delta^{-1/2} \xi_1, \xi_2\rangle ^* =\nonumber\\
    &= \langle \Delta^{-1/2} \Delta^{1/2}\Delta^{1/2} \Delta^{-1/2} \xi_1, \xi_2\rangle ^*
    =\nonumber\\
    &= \langle \xi _1, \xi_2\rangle ^* = \langle \xi_2 , \xi_1\rangle ,
\end{align}
which concludes the anti-unitarity by definition.
%------------------- EXERCISE
\subsubsection{Morphisms Between Finite Real Spectral Triples}
Like the unitary equivalence relation for finite spectral triples, we can it
extend it to finite real spectral triples.
\begin{mydefinition}
    We call two finite real spectral triples $(A_1, H_1 ,D_1 ; J_1 , \gamma_1)$
    and $(A_2, H_2, D_2; J_2, \gamma _2)$ unitarily equivalent if $A_1 =
    A_2$ and if there exists a unitary operator $U: H_1 \rightarrow H_2$ such
    that
    \begin{align}
        U\ \pi_1(a)\ U^* &= \pi _2(a),\\
        U\ D_1\ U^* &= D_2,\\
        U \gamma _1\ U^*  &= \gamma _2,\\
        U\ J_1\ U^* &= J_2.
    \end{align}
\end{mydefinition}
\begin{mydefinition}
    Let $E$ be a $B$-$A$ bimodule. The \textit{conjugate Module} $E^\circ$ is
    given by the $A$-$B$-bimodule.
    \begin{align}
        E^\circ = \{\bar{e} : e\in E\},
    \end{align}
    with
    \begin{align}
    a \cdot \bar{e} \cdot b = b^*\ \bar{e}\ a^*, \;\;\;\; \forall a\in A, b \in
        B.
    \end{align}
\end{mydefinition}
We bear in mind that $E^\circ$ is not a Hilbert bimodule for $(A, B)$ because
it doesn't have a natural $B$-valued inner product. But there is a $A$-valued
inner product on the left $A$-module $E^\circ$ with
\begin{align}
    \langle \bar{e}_1, \bar{e}_2 \rangle = \langle e_2 , e_1 \rangle,
    \;\;\;\; e_1, e_2 \in E.
\end{align}
And linearity in $A$ by the terms
\begin{align}
    \langle a\ \bar{e}_1, \bar{e}_2 \rangle = a \langle \bar{e}_1, \bar{e}_2
    \rangle, \;\;\;\; \forall a \in A.
\end{align}

%------------- EXERCISE
It turns out that $E^\circ$ is a Hilbert bimodule
of $(B^{\circ}, A^{\circ})$. A straightforward calculation of the properties of the Hilbert bimodule and its $B^{\circ}$
valued inner product gives the results. For $\bar{e}_1, \bar{e}_2 \in E^{\circ}$ and $a^\circ \in A,
b^\circ \in B$ we write
\begin{align}
    \langle\bar{e}_1, a^\circ \bar{e}_2\rangle &= \langle\bar{e}_1, Ja^*J^{-1}
    \bar{e}_2\rangle=\nonumber\\
    &= \langle\bar{e}_1 , J a^* e_2\rangle \nonumber \\
    &= \langle J^{-1} e_1, a^* e_2\rangle \nonumber\\
    & = \langle a^* e_1, e_2\rangle= \langle J^{-1}(a^\circ)^* J e_1, e_2\rangle  \nonumber\\
    & = \langle J^{-1} (a^\circ)^* \bar{e}_1, e_2\rangle \nonumber\\
    & = \langle (a^\circ)^* \bar{e}_1 , \bar{e}_2\rangle.
\end{align}
Next for $\langle\bar{e}_1, \bar{e}_2 b^\circ\rangle = \langle\bar{e}_1,
\bar{e_2}\rangle b^\circ$ we obtain
\begin{align}
    \langle\bar{e}_1, \bar{e}_2 b^\circ\rangle  &= \langle\bar{e}_1, \bar{e}_2 Jb^*J^{-1}\rangle
    \nonumber\\
    &= \langle\bar{e}_1, \bar{e_2}\rangle Jb^*J^{-1} \nonumber \\
    &= \langle\bar{e}_1, \bar{e}_2\rangle b^\circ.
\end{align}
Additionally we get
\begin{align}
    (\langle\bar{e}_1, \bar{e}_2)\rangle_{E^\circ})^* &= (\langle e_2, e_1\rangle_E)^*\nonumber\\
                                          &= \langle e_1, e_2\rangle_E^* \nonumber\\
                                          &= \langle\bar{e}_2, \bar{e}_2\rangle_{E^\circ}.
\end{align}
And finally we have
\begin{align}
    \langle\bar{e}, \bar{e}\rangle = \langle e, e\rangle \geq 0
\end{align}
%------------- EXERCISE

Given the results thus far, given a Hilbert bimodule $E$ for $(B, A)$ one can
construct a spectral triple $(B, H', D'; J', \gamma ')$ from $(A, H, D; J,
\gamma)$. For $H'$ we make a $\mathbb{C}$-valued inner product on $H'$ by combining
the $A$ valued inner product on $E$ and $E^\circ$ with the
$\mathbb{C}$-valued inner product on $H$ by defining
\begin{align}
    H' := E\otimes _A H \otimes _A E^\circ.
\end{align}
Then the action of $B$ on $H'$ takes the following form
\begin{align}
    b(e_2 \otimes \xi \otimes \bar{e}_2 ) = (be_1) \otimes \xi \otimes
    \bar{e}_2.
\end{align}
The right action of $B$ on $H'$ defined by action on the right components of
$E^\circ$ is
\begin{align}
    J'(e_1 \otimes \xi \otimes \bar{e}_2) = e_2 \otimes J \xi \otimes
    \bar{e}_1,
\end{align}
where $b^\circ = J' b^* (J')^{-1}$ and $b^* \in B$ is the action on $H'$.
Hence the connection reads
\begin{align}
    &\nabla: E \rightarrow E\otimes _A \Omega _D ^1(A) \\
    &\bar{\nabla}:E^\circ \rightarrow \Omega _D^1(A) \otimes _A E^\circ,
\end{align}
which gives the Dirac operator on $H' = E \otimes _A H \otimes _A
E^\circ$ as
\begin{align}
    D'(e_1 \otimes \xi \otimes \bar{e}_2) = (\nabla e_1) \xi \otimes
    \bar{e_2}+ e_1 \otimes D\xi \otimes \bar{e}_2 + e_1 \otimes
    \xi(\bar{\nabla}\bar{e}_2).
\end{align}
And the right action of $\omega \in \Omega _D ^1(A)$ on $\xi \in H$ is
defined by
\begin{align}
    \xi \mapsto \epsilon' J \omega ^* J^{-1}\xi.
\end{align}
Finally for the grading one obtains
\begin{align}
    \gamma ' = 1 \otimes \gamma \otimes 1.
\end{align}

Summarizing we can write down the following theorem
\begin{mytheorem}
    Suppose $(A, H, D; J, \gamma)$ is a finite spectral triple of
    $KO$-dimension $k$, let $\nabla$ be a connection satisfying the
    compatibility condition (same as with finite spectral triples).
    Then $(B, H',D'; J', \gamma')$ is a finite spectral triple of
    $KO$-Dimension $k$. ($H', D', J', \gamma'$)
\end{mytheorem}

\begin{proof}
    The only thing left is to check is, if the $KO$-dimension is preserved.
    That is one needs to check if if the $\epsilon$'s are the same.
    \begin{align}
        &(J')^2 = 1 \otimes J^2 \otimes 1 = \epsilon,\\
        &J' \gamma '= \epsilon ''\gamma'J'.
    \end{align}
    Lastly for $\epsilon '$ one obtains
    \begin{align}
        J'D'(e_1 \otimes \xi \otimes \bar{e}_2)&=J'\big((\nabla e_1) \xi \otimes
        \bar{e_2} + e_1 \otimes D\xi \otimes \bar{e}_2 + e_1 \otimes \xi (\tau
        \nabla e_2)\big)\nonumber \\
        &= \epsilon' D'\left(e_2 \otimes J\xi \otimes \bar{e}_2\right)\nonumber\\
        &= \epsilon' D'J'\left(e_1 \otimes \xi \bar{e}_2\right)
    \end{align}
\end{proof}

Let us take a look at $\nabla : E \Rightarrow E \otimes _A \Omega _d^1 (A)$,
the right connection on $E$ and consider the following anti-linear map
\begin{align}
    \tau : E \otimes_A \Omega _D^1 (A) &\rightarrow \Omega _D^1 (A) \otimes_A E^\circ\\
            e \otimes \omega &\mapsto -\omega ^* \otimes \bar{e}.
\end{align}
Interestingly the map $\bar{\nabla} : E^\circ \rightarrow \Omega _D^1(A) \otimes E^\circ$
with $\bar{\nabla}(\bar{e}) = \tau \circ \nabla(e)$ is a left connection, that means
show that it satisfied the left Leibniz rule, for one
\begin{align}
    \tau \circ \nabla(ae) = \bar{\nabla}(a\bar{e}) = \bar{\nabla}(a^*
    \bar{e}).
\end{align}
And for two
\begin{align}
    \tau \circ \nabla(ae) &= \tau(\nabla(e)a) + \tau \circ(e \otimes
     d(a))\nonumber \\
     &=a^*\bar{\nabla}(\bar{e}) - d(a)^* \otimes \bar{e}. \nonumber\\
     &= a^*\bar{\nabla}(\bar{e}) + d(a^*) \otimes \bar{e}.
\end{align}

