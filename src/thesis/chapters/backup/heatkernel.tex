\subsection{Heat Kernel Expansion}
\subsubsection{The Heat Kernel}
The heat kernel $K(t; x, y; D)$ is the fundamental solution of the heat
equation
\begin{align}
    (\partial _t + D_x)K(t;x, y;D) =0,
\end{align}
which depends on the operator $D$ of Laplacian type.

For a flat manifold $M = \mathbb{R}^n$ and $D = D_0 := -\Delta_\mu\Delta^\mu +m^2$ the
Laplacian with a mass term and the initial condition
\begin{align}
    K(0;x,y;D) = \delta(x,y),
\end{align}
takes the form of the standard fundamental solution
\begin{align}\label{eq:standard}
    K(t;x,y;D_0) = (4\pi t)^{-n/2}\exp\left(-\frac{(x-y)^2}{4t}-tm^2\right).
\end{align}
Let us consider now a more general operator $D$ with a potential term or a
gauge field, the heat kernel reads then
\begin{align}
    K(t;x,y;D) = \langle x|e^{-tD}|y\rangle.
\end{align}
We can expand the heat kernel in $t$, still having a
singularity from the equation \eqref{eq:standard} as $t \rightarrow 0$ thus the
expansion reads
\begin{align}
    K(t;x,y;D) = K(t;x,y;D_0)\left(1 + tb_2(x,y) + t^2b_4(x,y) + \dots
    \right),
\end{align}
where $b_k(x,y)$ become regular as $y \rightarrow x$. These coefficients are called the heat
kernel coefficients.
%%----------------------- KANN WEGGELASSEN WERDEN
%\newline
%\textbf{KANN WEGELASSEN WERDEN BIS ZUM NÄCHSTEN KAPITEL}
%Let's turn our attention to a propagator $D^{-1}(x,y)$ defined through the
%heat kernel, with an integral representation
%\begin{align}
%    D^{-1} (x,y) = \int_0^\infty dt K(t;x,y;D).
%\end{align}
%If we assume the heat kernel vanishes for $t\rightarrow \infty$, we can
%integrate formally to get
%\begin{align}
%    D^{-1}(x,y) \simeq
%    2(4\pi)^{-n/2}\sum_{j=0}\left(\frac{|x-y|}{2m}\right)^{-\frac{n}{2}+j+1}
%    K_{-\frac{n}{2}+j+1}(|x-y|m)b_{2j}(x,y),
%\end{align}
%where $b_0 = 1$ and $K_\nu (z)$ is the Bessel function
%\begin{align}
%    K_\nu(z) = \frac{1}{\pi} \int_0^\pi \cos(\nu\tau-z\sin(\tau))d\tau.
%\end{align}
%The Bessel function solves the following differential equation
%\begin{align}
%    z^2 \frac{d^2K}{dz^2} + z \frac{dK}{dz} + (z^2 - \nu^2)=0.
%\end{align}
%By looking at an integral approximation for the propagator we conclude that
%the singularities of $D^{-1}$ coincide with the singularities of the heat
%kernel coefficients. Thus we can say, that a generating functional in terms of
%$\det(D)$ is called the one-loop effective action (quantum field theory)
%\begin{align}
%    W = \frac{1}{2}\ln(\det D).
%\end{align}
%We have a direct relation with one-loop effective action $W$ and the
%heat kernel. Furthermore notice that for each eigenvalue $\lambda >0$ of $D$
%we can write the identity.
%\begin{align}
%    \ln \lambda  = -\int_0^\infty \frac{e^{-t\lambda}}{t}dt
%\end{align}
%This expression is correct up to an infinite constant which does not depend
%on the eigenvalue $\lambda$, thus we can ignore it. By substituting
%$\ln(\det D) = \text{Tr}(\ln D)$ we can rewrite the one-loop effective action
%$W$ into
%\begin{align}
%    W = -\frac{1}{2} \int_0^\infty dt \frac{K(t, D)}{t},
%\end{align}
%where
%\begin{align}
%    K(t, D) = \text{Tr}(e^{-tD}) = \int d^n x \sqrt{g}K(t;x,x;D).
%\end{align}
%The problem now is that the integral of $W$ is divergent at both limits. Yet
%the divergences at $t\rightarrow \infty$ are caused by $\lambda \leq 0$ of $D$
%(infrared divergences) and can be ignored. The divergences at $t\rightarrow 0$
%are cutoff at $t=\Lambda^{-2}$, simply written as
%\begin{align}
%    W_\Lambda = -\frac{1}{2} \int_{\Lambda^{-2}}^\infty dt \frac{K(t, D)}{t}.
%\end{align}
%We can calculate $W_\Lambda$ up to an order of $\lambda ^0$
%\begin{align}
%    W_\Lambda &= -(4\pi)^{-n/2} \int d^n x\sqrt{g}\bigg(
%    \sum_{2(j+l)<n}\Lambda^{n-2j-2l}b_{2j}(x,x) \frac{(-m^2)^l l!}{n-2j-2l} +\\
%    &+ \sum_{2(j+l) =n }\ln(\Lambda) (-m^2)^l l! b_{2j}(x,x)
%    \mathcal{O}(\lambda^0) \bigg)
%\end{align}
%There is an divergence at $b_2(x,x)$ for $k\leq n$. Computing the limit
%$\Lambda \rightarrow \infty$ we get
%\begin{align}
%    -\frac{1}{2}(4\pi)^{n/2}m^n\int d^n x\sqrt{g} \sum_{2j>n}
%    \frac{b_{2j}(x,x)}{m^{2j}}\Gamma(2j-n),
%\end{align}
%where $\Gamma$ stands for the gamma function.
%%----------------------- KANN WEGGELASSEN WERDEN


\subsubsection{Spectral Functions}
Manifolds $M$ with a disappearing boundary condition for the operator $e^{-tD}$ for $t>0$ is a
trace class operator on $L^2(V)$. Meaning for any smooth function $f$ on $M$
we can define
\begin{align}
    K(t,f,D) := \text{Tr}_{L^2}(fe^{-tD}),
\end{align}
or alternately write an integral representation
\begin{align}
    K(t, f, D) = \int_M d^n x \sqrt{g} \text{Tr}_V(K(t;x,x;D)f(x)),
\end{align}
in the regular limit $y\rightarrow y$. We can write the Heat Kernel in terms
of the spectrum of $D$. So for an orthonormal basis $\{\phi_\lambda\}$ of
eigenfunctions for $D$, which corresponds to the eigenvalue $\lambda$, we
can rewrite the heat kernel into
\begin{align}
    K(t;x,y;D) = \sum_\lambda \phi^\dagger_\lambda(x)
    \phi_\lambda(y)e^{-t\lambda}.
\end{align}
An asymptotic expansion as $t \rightarrow 0$ for the trace is then
\begin{align}
    \text{Tr}_{L^2}(fe^{-tD}) \simeq \sum_{k\geq 0}t^{(k-n)/2}a_k(f,D),
\end{align}
where
\begin{align}
    a_k(f,D) = (4\pi)^{-n/2} \int_M d^4x \sqrt{g} b_k(x,x) f(x).
\end{align}
\subsubsection{General Formulae}
Let us summarize what we have obtained in the last chapter, we considered a
compact Riemannian manifold $M$ without boundary condition, a vector bundle
$V$ over $M$ to define functions which carry discrete (spin or gauge)
indices, an operator $D$ of Laplace type over $V$ and smooth function $f$ on
$M$.

There is an asymptotic expansion where the heat kernel coefficients with an
odd index $k=2j+1$ vanish $a_{2j+1}(f,D) = 0$. On the other hand coefficients
with an even index are locally computable in terms of geometric invariants
\begin{align}
    a_k(f,D) &= \text{Tr}_V\left(\int_M d^n x\sqrt{g}(f(x)a_k(x;D)\right)
    =\nonumber\\
    &=\sum_I \text{Tr}_V\left(\int_M d^nx \sqrt{g}(fu^I
    \mathcal{A}^I_k(D))\right).
\end{align}
We denote $\mathcal{A}^I_k$ as all possible independent invariants of
dimension $k$, and $u^I$ are constants. The invariants are constructed from
$E, \Omega, R_{\mu\nu\varrho\sigma}$ and their derivatives If $E$ has
dimension two, then the derivative has dimension one. So if $k=2$ there are
only two independent invariants, $E$ and $R$. This corresponds to the
statement $a_{2j+1}=0$.

If we consider $M = M_1 \times M_2$ with coordinates $x_1$ and $x_2$ and a
decomposed Laplace style operator $D = D_1 \otimes 1 + 1 \otimes D_2$ we can
separate functions acting on operators and on coordinates linearly by the
following
\begin{align}
    e^{-tD} &= e^{-tD_1} \otimes e^{-tD_2},\\
    f(x_1, x_2) &= f_1(x_1)f_2(x_2),
\end{align}
thus the heat kernel coefficients are separated by
\begin{align}
    a_k(x;D) &= \sum_{p+q=k} a_p(x_1; D_1)a_q(x_2;D_2)
\end{align}
If we know the eigenvalues of $D_1$ are known,  $l^2, l\in \mathbb{Z}$, we
can obtain the heat kernel asymmetries with the Poisson summation formula
giving us an approximation in the order of $e^{-1/t}$
\begin{align}
    K(t, D_1) &= \sum_{l\in\mathbb{Z}} e^{-tl^2} = \sqrt{\frac{\pi}{t}}
    \sum_{l\in\mathbb{Z}} e^{-\frac{\pi^2l^2}{t}} = \nonumber \\
    &\simeq \sqrt{\frac{\pi}{t}} + \mathcal{O}(e^{-1/t}).
\end{align}
The exponentially small terms have no effect on the heat kernel
coefficients and that the only nonzero coefficient is $a_0(1, D_1) =
\sqrt{\pi}$, therefore the heat coefficients can be written as
\begin{align}
    a_k(f(x^2), D) = \sqrt{\pi}\int_{M_2}
    d^{n-1}x\sqrt{g}\sum_I\text{Tr}_V\left(f(x^2)u^I_{(n-1)}
    \mathcal{A}^I_n(D_2)\right).
\end{align}

Because all of the geometric invariants  associated with $D$ are in the $D_2$
part, they are independent of $x_1$. Ultimately meaning we are free to choose
$M_1$. For $M_1 = S^1$ with $x\in (0, 2\pi)$ and $D_1=-\partial_{x_1}^2$
we can rewrite the heat kernel coefficients into
\begin{align}
    a_k(f(x_2), D) &= \int_{S^1\times M_2}d^nx \sqrt{g} \sum_I
    \text{Tr}_V(f(x_2) u_{(n)}^I \mathcal{A}^I_k(D_2))= \nonumber\\
    &= 2\pi \int_{M_2} d^nx\sqrt{g} \sum_I\text{Tr}_V(f(x_2) u_{(n)}^I
    \mathcal{A}^I_k(D_2)).
\end{align}
Computing the two equations above we see that
\begin{align}
    u_{(n)}^I = \sqrt{4\pi} u^I_{(n+1)}
\end{align}

\subsubsection{Heat Kernel Coefficients}
To calculate the heat kernel coefficients we need the following variational
equations
\begin{align}
    &\frac{d}{d\varepsilon}\bigg|_{\varepsilon=0}a_k(1, e^{-2\varepsilon f}D) =
    (n-k) a_k(f, D),\label{eq:var1}\\
    &\frac{d}{d\varepsilon}\bigg|_{\varepsilon=0}a_k(1, D-\varepsilon F) =
    a_{k-2}(F,D),\label{eq:var2}\\
    &\frac{d}{d\varepsilon}\bigg|_{\varepsilon=0}a_k(e^{-2\varepsilon f}F,
    e^{-2\varepsilon f}D) =
    0\label{eq:var3}.
\end{align}
Let us explain the equations above. To get the first equation \eqref{eq:var1}
we differentiate \begin{align}
    \frac{d}{d\varepsilon}\bigg|_{\varepsilon=0} \text{Tr}(\exp(-e^{-2\varepsilon
    f}tD) = \text{Tr}(2ftDe^{-tD}) = -2t\frac{d}{dt}\text{Tr}(fe^{-tD}))
\end{align}
then we expand both sides in $t$ and get \eqref{eq:var1}. Equation \eqref{eq:var2} is derived similarly.

For equation \eqref{eq:var3} we consider the following operator
\begin{align}
    D(\varepsilon,\delta) = e^{-2\varepsilon f}(D-\delta F)
\end{align}
for $k=n$ we use equation \eqref{eq:var1} and we get
\begin{align}
    \frac{d}{d\varepsilon}\bigg|_{\varepsilon=0}a_n(1,D(\varepsilon,\delta))
    =0,
\end{align}
then we take the variation in terms of $\delta$, evaluated at $\delta =0$ and
swap the differentiation, allowed by theorem of Schwarz
\begin{align}
    0 &=
    \frac{d}{d\delta}\bigg|_{\delta=0}\frac{d}{d\varepsilon}\bigg|_{\varepsilon=0}a_n(1,
    D(\varepsilon,\delta)) =\nonumber\\
      &=\frac{d}{d\varepsilon}\bigg|_{\varepsilon=0}\frac{d}{d\delta}\bigg|_{\delta=0}a_n(1,
    D(\varepsilon,\delta)) =\nonumber\\
      &=a_{n-2} ( e^{-2\varepsilon f}F, e^{-2\varepsilon f}D),
\end{align}
which gives us equation \eqref{eq:var3}.

Now that we have established the ground basis, we can calculate the constants
$u^I$, and by that the first three heat kernel coefficients read
\begin{align}
    a_0(f, D) &= (4\pi)^{-n/2}\int_Md^n x\sqrt{g} \text{Tr}_V(a_0 f),\\
    a_2(f, D) &= (4\pi)^{-n/2}\frac{1}{6}\int_Md^n
    x\sqrt{g}\text{Tr}_V)(f\alpha _1 E+\alpha _2 R),\\
    a_4(f, D) &= (4\pi)^{-n/2}\frac{1}{360}\int_Md^n
    x\sqrt{g}\text{Tr}_V(f(\alpha_3 E_{,kk} + \alpha_4\ R\ E + \alpha_5 E^2
    \alpha_6 R_{,kk} + \nonumber\\
    &+\alpha_7 R^2 + \alpha_8 R_{ij}R_{ij} + \alpha_9
    R_{ijkl}R_{ijkl} +\alpha_{10} \Omega_{ij}\Omega{ij})),
\end{align}
where the comma subscript $,$ denotes the derivative and constants $\alpha_I$
do not depend on the dimension of the Manifold and we can compute them with
our variational identities.

The first coefficient $\alpha_0$ can be read from the heat kernel expansion of
the Laplacian on $S^1$ (above), $\alpha_0 = 1$. For $\alpha_1$ we use
\eqref{eq:var2}, the coefficient $k = 2$ is
\begin{align}
    \frac{1}{6} \int_M d^n x\sqrt{g} \text{Tr}_V(\alpha_1F) = \int_M d^n
    x\sqrt{g} \text{Tr}_V(F),
\end{align}
which means $\alpha_1 = 6$. Looking at the coefficient $k=4$ we have
\begin{align}
    \frac{1}{360}\int_Md^n x \sqrt{g}\text{Tr}_V(\alpha_4\ F\ R + 2\alpha_5\ F\ E)
    = \frac{1}{6} \int_Md^n x\sqrt{g}\text{Tr}_V(\alpha_1\ F\ E + \alpha_2\ F\ R),
\end{align}
thus $\alpha_4 = 60\alpha_2$ and $\alpha_5 = 180$.

By applying  \eqref{eq:var3} to $n=4$ we get
\begin{align}
    \frac{d}{d\varepsilon}|_{\varepsilon=0} a_2(e^{-2\varepsilon f}F,
    e^{-2\varepsilon f}D) = 0.
\end{align}
Collecting the terms with $\text{Tr}_V(\int_Md^nx\sqrt{g}(Ff_{,jj}))$ we
obtain $\alpha_1 = 6\alpha_2$, that is $\alpha_2 = 1$, so $\alpha_4 = 60$.

Now we let $M=M_1\times M_2$ and split $D = -\Delta_1 -\Delta_2$, where
$\Delta_{1/2}$ are Laplacians for $M_1, M_2$. This allows us to decompose the heat
kernel coefficient for $k=4$ into
\begin{align}
    a_4(1,-\Delta_1-\Delta_2) =& a_4(1, -\Delta_1) a_0(1,
    -\Delta_2)\nonumber+ \\
                               &+a_2(1,-\Delta_1) a_2(1,-\Delta_2)\nonumber \\
                               &+ a_0(1,-\Delta_1)a_4(1,-\Delta_2),
\end{align}
with $E=0$ and $\Omega =0$ and by calculating the terms with $R_1R_2$  (scalar
curvature of $M_{1/2}$) we obtain $\frac{2}{360}\alpha_7 =
(\frac{\alpha_2}{6})^2$, thus $\alpha_7 = 5$.

For $n=6$ we get
\begin{align}
    0 &= \text{Tr}_V(\int_Md^nx\sqrt{g}
    (F(-2\alpha_3-10\alpha_4+4\alpha_5)f_{,kk}E +\nonumber\\
    &+(2\alpha_3 + 10\alpha_6)f_{,iijj}+\nonumber\\
    &+(2\alpha_4 -2\alpha_6 - 20\alpha_7 -2\alpha_8)f_{,ii}R\nonumber\\
    &+(-8\alpha_8 -8\alpha_6)f_{,ij}R_{ij}))
\end{align}
we obtain $\alpha_3 = 60$, $\alpha_6=12$, $\alpha_8 = -2$ and $\alpha_9 = 2$

To get $\alpha_{10}$ we use the Gauss-Bonnet theorem, ultimately giving us
$\alpha_{10}=30$. We leave out this lengthy calculation and refer to
\cite{heatkernel} for further reading.

Let us summarize our calculations which ultimately give us the following heat kernel
coefficients
\begin{align}
    \alpha_0(f, D) &= (4\pi)^{-n/2}\int_M d^n x \sqrt{g} \text{Tr}_V(f),\\
    \alpha_2(f, D) &= (4\pi)^{-n/2}\frac{1}{6}\int_M d^n x \sqrt{g}
    \text{Tr}_V(f(6E+R)),\\
    \alpha_4(f, D) &= (4\pi)^{-n/2}\frac{1}{360}\int_M d^n x \sqrt{g}
    \text{Tr}_V(f(60E_{,kk}+60RE+ 180E^2 +\\
    &+12R_{,kk} + 5R^2 - 2 R_{ij}R_{ij}
    2R_{ijkl}R_{ijkl} +30\Omega_{ij}\Omega_{ij})).
\end{align}

