\subsection{Almost-commutative Manifold\label{sec:4}}
\subsubsection{Two-Point Space}
One of the basics forms of noncommutative space is the Two-Point space $X
:= \{x, y\}$. The Two-Point space can be represented by the following spectral triple
\begin{align}
    F_X := (C(X) = \mathbb{C}^2, H_F, D_F; J_F, \gamma _f).
\end{align}
Three properties of $F_X$ stand out. First of all the action of $C(X)$ on
$H_F$ is faithful for $dim(H_F) \geq 2$, thus a simple choice for the
Hilbertspace can be made, for instance $H_F = \mathbb{C}^2$. Furthermore
$\gamma_F$ is the $\mathbb{Z}_2$ grading, which allows for a decomposition of
$H_F$ into
\begin{align}
   H_F = H_F^+ \otimes H_F^- = \mathbb{C} \otimes \mathbb{C},
\end{align}
where
\begin{align}
    H_F^\pm = \{\psi \in H_F |\; \gamma_F\psi = \pm \psi\},
\end{align}
are two eigenspaces. And lastly the Dirac operator $D_F$ lets us
interchange between the two eigenspaces $H_F^\pm$,
\begin{align}
    D_F =
        \begin{pmatrix}0 & t \\ \bar{t} & 0\end{pmatrix},  \;\;\;\;\;
            \text{with} \;\; t\in\mathbb{C}.
\end{align}

The Two-Point space $F_X$ can only have a real structure if the Dirac
operator vanishes, i.e. $D_F = 0$. In that case the KO-dimension is 0,
2 or 6. To elaborate further, we draw the only two diagram representations of
$F_X$ at $\underbrace{\mathbb{C} \oplus \mathbb{C}}_{C(X)}$ on
$\underbrace{\mathbb{C} \oplus\mathbb{C}}_{H_F}$, which are
\begin{figure}[h!] \centering
\begin{tikzpicture}[
    dot/.style = {draw, circle, inner sep=0.06cm},
    no/.style = {},
    ]
    \node[no](a) at (0,0) [label=left:$\textbf{1}^\circ$] {};
    \node[no](b) at (0, -1) [label=left:$\textbf{1}^\circ$] {};
    \node[no](c) at (1, 0.5) [label=above:$\textbf{1}$] {};
    \node[no](d) at (2, 0.5) [label=above:$\textbf{1}$] {};
    \node[dot](d0) at (2,0) [] {};
    \node[dot](d0) at (1,-1) [] {};

    \node[no](a1) at (6,0) [label=left:$\textbf{1}^\circ$] {};
    \node[no](b2) at (6, -1) [label=left:$\textbf{1}^\circ$] {};
    \node[no](c2) at (7, 0.5) [label=above:$\textbf{1}$] {};
    \node[no](d2) at (8, 0.5) [label=above:$\textbf{1}$] {};
    \node[dot](d0) at (7,0) [] {};
    \node[dot](d0) at (8,-1) [] {};
    \end{tikzpicture}
    \caption{Two diagram representations of $F_X$}
\end{figure}\newline
If the Two-Point space $F_X$ would be a real spectral triple then $D_F$ can
only go vertically or horizontally. This would mean that $D_F$ vanishes.
As for the KO-dimension The diagram on the left has KO-dimension 2 and 6, the diagram on the
right 0 and 4. Yet KO-dimension 4 is ruled out because
$dim(H_F^\pm) = 1$ (Lemma 3.8 in \cite{ncgwalter}) , which ultimately means $J_F^2 = -1$ is
not allowed.
\subsubsection{Product Space}
By Extending the Two-Point space with a four dimensional Riemannian spin
manifold, we get an almost commutative manifold $M\times F_X$, given by
\begin{align}
    M\times F_X = \big(C^\infty(M, \mathbb{C}^2), L^2(S)\otimes \mathbb{C}^2,
    D_M\otimes 1 ; J_M\otimes J_F, \gamma_M \otimes \gamma_F\big),
\end{align}
where
\begin{align}
   C^\infty(M, \mathbb{C}^2) \simeq C^\infty(M) \oplus  C^\infty(M).
\end{align}
According to Gelfand duality the algebra $C^\infty(M, \mathbb{C}^2)$ of the
spectral triple corresponds to the space
\begin{align}
    N:= M\otimes X.
\end{align}
Keep in mind that we still need to find an appropriate real structure on the
Riemannian spin manifold, $J_M$. Furthermore the total Hilbertspace can be
decomposed into $H = L^2(S) \oplus L^2(S)$, such that for $\underbrace{a,b\in
C^\infty(M)}_{(a, b) \in C^\infty(N)}$ and $\underbrace{\psi, \phi \in
L^2(S)}_{(\psi, \phi) \in H}$ we have
\begin{align}
    (a, b)(\psi, \phi) = (a\psi, b\phi).
\end{align}
Along with the decomposition of the total Hilbertspace a
distance formula on $M\times F_X$ can be considered with
\begin{align}\label{eq:commutator inequality}
    d_{D_F}(x,y) = \sup\left\{  |a(x) - a(y)|:a\in A_F, ||[D_F, a]|| \leq
    1 \right\}.
\end{align}
To calculate the distance between two points on the Two-Point space $X= \{x,
y\}$, between $x$ and $y$, we consider an $a \in \mathbb{C}^2 = C(X)$, which is
specified by two complex numbers $a(x)$ and $a(y)$. Then we simplify the
commutator inequality in \eqref{eq:commutator inequality}
\begin{align}
    &||[D_F , a]|| = ||(a(y) - a(x))\begin{pmatrix}0 &t\\\bar{t} &0
    \end{pmatrix}|| \leq 1,\\
    &\Leftrightarrow |a(y) - a(x)|\leq \frac{1}{|t|}.
\end{align}
The supremum then gives us the distance
\begin{align}
    d_{D_F} (x,y) = \frac{1}{|t|}.
\end{align}
An interesting observation here is that, if the Riemannian spin manifold can be
represented by a real spectral triple then a real structure $J_M$ exists,
along the lines it follows that $t=0$ and the distance becomes infinite. This is a
purely mathematical observation and has no physical meaning.

We can also construct a distance formula on $N$ (in reference to a point $p
\in M$) between two points on $N=M\times X$, $(p, x)$ and $(p,y)$. Then an $a
\in C^\infty(N)$ is determined by $a_x(p):=a(p, x)$ and $a_y(p):=a(p, y)$.
The distance between these two points is
\begin{align}
    d_{D_F\otimes 1}(n_1, n_2) =  \sup \left\{ |a(n_1) - a(n_2)|: a\in
    A, ||[D\otimes 1, a]||\right\}.
\end{align}
On the other hand if we consider $n_1 = (p,x)$ and $n_2 = (q, x)$
for $p,q \in M$ then
\begin{align}
    d_{D_M \otimes 1} (n_1, n_2) = |a_x(p) - a_x(q)| \;\;\;\text{for}\;\;
    a_x\in
    C^\infty(M) \;\; \text{with} \;\; ||[D\otimes 1, a_x]|| \leq 1
\end{align}
The distance formula turns to out to be the geodesic distance formula
\begin{align}
    d_{D_M\otimes1}(n_1, n_2) = d_g(p, q),
\end{align}
which is to be expected since we are only looking at the manifold.
However if $n_1 = (p, x)$ and $n_2 = (q, y)$ then the two conditions are
\begin{align}
    &||[D_M, a_x]|| \leq 1, \;\;\; \text{and}\\
    &||[D_M, a_y|| \leq 1.
\end{align}
These conditions have no restriction which results in the distance being
infinite! And $N = M\times X$ is given by two disjoint copies of M  which are
separated by infinite distance

The distance is only finite if $[D_F, a] < 1$. In this case the commutator
generates a scalar field and the finiteness of the distance is
related to the existence of scalar fields.

\subsubsection{$U(1)$ Gauge Group}
To get a insight into the physical properties of the almost commutative
manifold $M\times F_X$, that is to calculate the spectral action, we need to
determine the corresponding Gauge group.
For this we set of with simple definitions and important propositions to
help us break down and search for the gauge group of the Two-Point $F_X$
space which we then extend to $M\times F_X$. We will only be diving
superficially into this chapter, for further reading we refer to
\cite{ncgwalter}.
\begin{mydefinition}
Gauge Group of a real spectral triple is given by
\begin{align}
    \mathfrak{B}(A, H; J) := \{ U = uJuJ^{-1} | u\in U(A)\}.
\end{align}
\end{mydefinition}
\begin{mydefinition}
    A *-automorphism of a *-algebra $A$ is a linear invertible
    map
    \begin{align}
        &\alpha:A \rightarrow A,\;\;\; \text{with}\\
        \nonumber\\
        &\alpha(ab) = \alpha(a)\alpha(b),\\
        &\alpha(a)^* = \alpha(a^*).
    \end{align}
    The \textbf{Group of automorphisms of the *-Algebra $A$} is denoted by
    $(A)$.\newline
    The automorphism $\alpha$ is called \textbf{inner} if
    \begin{align}
        \alpha(a) = u a u^* \;\;\; \text{for} \;\; U(A),
    \end{align}
    where $U(A)$ is
    \begin{align}
        U(A) = \{ u\in A|\;\; uu^* = u^*u=1\}. \;\;\;
        \text{(unitary)}
    \end{align}
\end{mydefinition}
The Gauge group of $F_X$ is given by the quotient $U(A)/U(A_J)$.
To get a nontrivial Gauge group so we need to choose a $U(A_J) \neq
U(A)$ and $U((A_F)_{J_F}) \neq U(A_F)$.
We consider our Two-Point space $F_X$ to be equipped with a real structure,
which means the operator vanishes, and the spectral triple representation is
\begin{align}
    F_X := \left(\mathbb{C}^2,\mathbb{C}^2, D_F =\begin{pmatrix}
        0&0\\0&0\end{pmatrix}; J_f =\begin{pmatrix}
    0&C\\C&0\end{pmatrix},
            \gamma_F = \begin{pmatrix}1&0\\0&-1\end{pmatrix}\right).
\end{align}
Here $C$ is the complex conjugation, and $F_X$ is a real even finite
spectral triple (space) of KO-dimension 6.

\begin{myproposition}
The Gauge group of the Two-Point space $\mathfrak{B}(F_X)$ is $U(1)$.
\end{myproposition}
\begin{proof}
    Note that $U(A_F) = U(1) \times U(1)$. We need to show that $U(A_F) \cap
    U(A_F)_{J_F}) \simeq U(1)$, such that $\mathfrak{B}(F) \simeq U(1)$. So
    for an element $a \in \mathbb{C}^2$ to be in $(A_F)_{J_F}$, it has to
    satisfy $J_F a^* J_F = a$,
    \begin{align}
        J_F a^* J^{-1} =
        \begin{pmatrix}0&C\\C&0\end{pmatrix}
            \begin{pmatrix}\bar{a}_1&0\\0&\bar{a}_2\end{pmatrix}
        \begin{pmatrix}0&C\\C&0\end{pmatrix}
            =
            \begin{pmatrix}a_2&0\\0&a_1\end{pmatrix}.
    \end{align}
    This can only be the case if $a_1 = a_2$. So we have
    $(A_F)_{J_F} \simeq \mathbb{C}$, whose unitary elements
    from $U(1)$ are contained in the diagonal subgroup of
    $U(A_F)$.
\end{proof}

An arbitrary hermitian field $A_\mu = -ia\partial _\mu b$  is given by
two $U(1)$ Gauge fields $X_\mu^1, X_\mu^2 \in C^\infty(M, \mathbb{R})$.
However $A_\mu$ appears in combination $A_\mu - J_F A_\mu J_F^{-1}$:
\begin{align}
 A_\mu - J_F A_\mu J_F^{-1} =
    \begin{pmatrix}X_\mu^1&0\\0&X_\mu^2 \end{pmatrix}
        -
    \begin{pmatrix}X_\mu^2&0\\0&X_\mu^1 \end{pmatrix}
        =:
    \begin{pmatrix}Y_\mu&0\\0&-Y_\mu \end{pmatrix}
    = Y_\mu \otimes \gamma _F,
\end{align}
where $Y_\mu$ the $U(1)$ Gauge field is defined as
\begin{align}
    Y_\mu := X_\mu^1 - X_\mu^2 \in C^\infty(M, \mathbb{R}) = C^\infty(M,
    i\ u(1)).
\end{align}

\begin{myproposition}
    The inner fluctuations of the almost-commutative manifold $M\times
    F_X$ are parameterized by a $U(1)$-gauge field $Y_\mu$ as
    \begin{align}
        D \mapsto D' = D + \gamma ^\mu Y_\mu \otimes \gamma_F
    \end{align}
    The action of the gauge group $\mathfrak{B}(M\times F_X) \simeq
    C^\infty (M, U(1))$ on $D'$ is implemented by
    \begin{align}
        Y_\mu \mapsto Y_\mu - i\ u\partial_\mu u^*; \;\;\;\;\; (u\in
        \mathfrak{B}(M\times F_X)).
    \end{align}
\end{myproposition}

