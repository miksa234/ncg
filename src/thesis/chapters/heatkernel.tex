
\subsection{Heat Kernel Expansion}
\subsubsection{The Heat Kernel}
The heat kernel $K(t; x, y; D)$ is the fundamental solution of the heat
equation. It depends on the operator $D$ of Laplacian type.
\begin{align}
    (\partial _t + D_x)K(t;x, y;D) =0
\end{align}
For a flat manifold $M = \mathbb{R}^n$ and $D = D_0 := -\Delta_\mu\Delta^\mu +m^2$ the
Laplacian with a mass term and the initial condition
\begin{align}
    K(0;x,y;D) = \delta(x,y)
\end{align}
we have the standard fundamental solution
\begin{align}\label{eq:standard}
    K(t;x,y;D_0) = (4\pi t)^{-n/2}\exp\left(-\frac{(x-y)^2}{4t}-tm^2\right)
\end{align}
Let us consider now a more general operator $D$ with a potential term or a
guage field, the heat kernel reads then
\begin{align}
    K(t;x,y;D) = \langle x|e^{-tD}|y\rangle.
\end{align}
We can expand it it in terms of $D_0$ and we still have the
singularity from the equation \ref{eq:standard} as $t\rightarrow 0$ thus the
expansion gives
\begin{align}
    K(t;x,y;D) = K(t;x,y;D_0)\left(1 + tb_2(x,y) + t^2b_4(x,y) + \dots \right)
\end{align}
where $b_k(x,y)$ are regular in $y \rightarrow x$. They are called the heat
kernel coefficients.

\subsubsection{Example}
Now let us consider a propagator $D^{-1}(x,y)$ defined through the heat kernel
in an integral representation
\begin{align}
    D^{-1} (x,y) = \int_0^\infty dt K(t;x,y;D).
\end{align}
We can integrate the expression formally if we assume the heat kernel vanishes
for $t\rightarrow \infty$ we get
\begin{align}
    D^{-1}(x,y) \simeq
    2(4\pi)^{-n/2}\sum_{j=0}\left(\frac{|x-y|}{2m}\right)^{-\frac{n}{2}+j+1}
    K_{-\frac{n}{2}+j+1}(|x-y|m)b_{2j}(x,y).
\end{align}
where $b_0 = 1$ and $K_\nu (z)$ is the Bessel function
\begin{align}
    K_\nu(z) = \frac{1}{\pi} \int_0^\pi cos(\nu\tau-z\sin(\tau))d\tau
\end{align}
it solves the differential equation
\begin{align}
    z^2 \frac{d^2K}{dz^2} + z \frac{dK}{dz} + (z^2 - \nu^2)=0.
\end{align}
By looking at integral approximation of the propagator we conclude
that the singularities of $D^{-1}$ coincide with the singularities of the heat
kernel coefficients.
We consider now a generating functional in terns of $\det(D)$ which is called
the one-loop effective action (quantum fields theory)
\begin{align}
    W = \frac{1}{2}\ln(\det D)
\end{align}
we can relate $W$ with the heat kernel. For each eigenvalue $\lambda >0$ of $D$
we can write the identity.
\begin{align}
    \ln \lambda  = -\int_0^\infty \frac{e^{-t\lambda}}{t}dt
\end{align}
This expression is correct up to an infinite constant which does not depent on
$\lambda$, because of this we can ignore it. Further more we use $\ln(\det D) =
\text{Tr}(\ln D)$ and therefor we can write for $W$
\begin{align}
    W = -\frac{1}{2} \int_0^\infty dt \frac{K(t, D)}{t}
\end{align}
where
\begin{align}
    K(t, D) = \text{Tr}(e^{-tD}) = \int d^n x \sqrt{g}K(t;x,x;D).
\end{align}
The problem is now that the integral of $W$ is divergent at both limits. Yet
the divergences at $t\rightarrow \infty$ are caused by $\lambda \leq 0$ of $D$
(infrared divergences) and are just ignored. The divergences at $t\rightarrow 0$
are cutoff at $t=\Lambda^{-2}$, thus we write
\begin{align}
    W_\Lambda = -\frac{1}{2} \int_{\Lambda^{-2}}^\infty dt \frac{K(t, D)}{t}.
\end{align}
We can calculate $W_\Lambda$ at up to an order of $\lambda ^0$
\begin{align}
    W_\Lambda &= -(4\pi)^{-n/2} \int d^n x\sqrt{g}\bigg(
    \sum_{2(j+l)<n}\Lambda^{n-2j-2l}b_{2j}(x,x) \frac{(-m^2)^l l!}{n-2j-2l} +\\
    &+ \sum_{2(j+l) =n }\ln(\Lambda) (-m^2)^l l! b_{2j}(x,x)
    \mathcal{O}(\lambda^0) \bigg)
\end{align}
There is an divergence at $b_2(x,x)$ with $k\leq n$. Now we compute the limit
$\Lambda \rightarrow \infty$
\begin{align}
    -\frac{1}{2}(4\pi)^{n/2}m^n\int d^n x\sqrt{g} \sum_{2j>n}
    \frac{b_{2j}(x,x)}{m^{2j}}\Gamma(2j-n)
\end{align}
here $\Gamma$ is the gamma function.
\subsubsection{Differential Geometry and Operators of Laplace Type}
Let $M$ be a $n$ dimensional compact Riemannian manifold with $\partial M = 0$.
Then consider a vector bundle $V$ over $M$ (i.e. there is a vector space to
each point on $M$), so we can define smooth functions. We want to look at
arbitrary differential operators $D$ of Laplace type on $V$, they have the general
from
\begin{align}
    D = -(g^{\mu\nu} \partial_\mu\partial_\nu + a^\sigma\partial_\sigma +b)
\end{align}
where $a^\sigma, b$ are matrix valued functions on $M$ and $g^{\mu\nu}$ is the
inverse metric on $M$. There is a unique connection on $V$ and a unique
endomorphism (matrix valued function) $E$ on $V$, then we can rewrite $D$ in
terms of $E$ and covariant derivatives
\begin{align}
    D = -(g^{\mu\nu} \nabla_\mu \nabla_\nu +E)
\end{align}
Where the covariant derivative consists of $\nabla = \nabla^{[R]} +\omega$ the
standard Riemannian covariant derivative $\nabla^{[R]}$ and a "gauge" bundle
$\omega$ (fluctuations). WE can write $E$ and $\omega$ in terms of geometrical
identities
\begin{align}
    \omega_\delta &= \frac{1}{2}g_{\nu\delta}(a^\nu
    +g^{\mu\sigma}\Gamma^\nu_{\mu\sigma}I_V)\\
    E &= b - g^{\nu\mu}(\partial_\mu \omega_\nu + \omega_\nu \omega_\mu -
    \omega_\sigma \Gamma^\sigma_{\nu\mu})
\end{align}
where $I_V$ is the identity in $V$ and the Christoffel symbol
\begin{align}
    \Gamma^\sigma_{\mu\nu} = g^{\sigma\varrho} \frac{1}{2} (\partial_\mu
    g_{\nu\varrho} + \partial_\nu g_{\mu\varrho} - \partial_\varrho g_{\mu\nu})
\end{align}
Furthermore we remind ourselves of the Riemmanian curvature tensor, Ricci
Tensor and the Scalar curavture.
\begin{align}
    R^\mu_{\nu\varrho\sigma} &= \partial_\sigma \Gamma^{\mu}_{\nu\varrho}
    -\partial_\varrho \Gamma^\mu_{\nu\sigma}
    \Gamma^{\lambda}_{\nu\varrho}\Gamma^{\mu}_{\lambda\sigma}
    \Gamma^{\lambda}_{\nu\sigma}\Gamma^{\mu}_{\lambda\varrho}\\
    R_{\mu\nu} &:= R^{\sigma}_{\mu\nu\sigma}\\
    R &:= R^\mu_{\ \mu}
\end{align}

The we let $\{e_1, \dots, e_n\}$ be the local orthonormal frame of
$TM$(tangent bundle $M$), which will be noted with flat indices $i,j,k,l
\in\{1,\dots, n\}$, we use $e^k_\mu, e^\nu_j$ to transform between flat indices
and curved indices $\mu, \nu, \varrho$.
\begin{align}
    e^\mu_j e^\nu_k g_{\mu\nu} &= \delta_{jk}\\
    e^\mu_j e^\nu_k \delta^{jk} &= g^{\mu\nu} \\
    e^j_\mu e^\mu_k  &= \delta^j_k
\end{align}

The Riemannian part of the covariant derivative contains the standard
Levi-Civita connection, so that for a $v_\nu$ we write
\begin{align}
    \nabla_\mu^{[R]} v_\nu = \partial_\mu v_\nu -
    \Gamma^{\varrho}_{\mu\nu}v_\varrho.
\end{align}
The extended covariant derivative reads then
\begin{align}
    \nabla_\mu v^j = \partial_\mu v^j + \sigma^{jk}_\mu v_k.
\end{align}
the condition $\nabla_\mu e^k_\nu = 0$ gives us the general connection
\begin{align}
    \sigma^{kl}_\mu = e^\nu_l\Gamma^{\varrho}_{\mu\nu}e^k_\varrho - e^\nu_l
    \partial_\mu e^k_\nu
\end{align}
The we may define the field strength $\Omega_{\mu\nu}$ of the connection $\omega$
\begin{align}
    \Omega_{\mu\nu} = \partial_\mu \omega_\nu -\partial_\nu \omega_\mu
    +\omega_\mu \omega_\nu -\omega_\nu\omega_\mu.
\end{align}
If we apply the covariant derivative on $\Omega$ we get
\begin{align}
    \nabla_\varrho\Omega_{\mu\nu} = \partial_\varrho \Omega_{\mu\nu} -
    \Gamma^{\sigma}_{\varrho \mu} \Omega_{\sigma\mu} + [\omega_\varrho,
    \Omega_{\mu\nu}]
\end{align}

\subsubsection{Spectral Functions}
Manifolds without $M$ boundary condition for the operator $e^{-tD}$ for $t>0$ is a
trace class operator on $L^2(V)$, this means that for any smooth function $f$
on $M$ we can define
\begin{align}
    K(t,f,D) = \text{Tr}_{L^2}(fe^{-tD})
\end{align}
and we can rewrite
\begin{align}
    K(t, f, D) = \int_M d^n x \sqrt{g} \text{Tr}_V(K(t;x,x;D)f(x)).
\end{align}
in terms of the Heat kernel $K(t;x,y;D)$ in the regular limit $y\rightarrow y$.
We can write the Heat Kernel in terms of the spectrum of $D$. Say
$\{\phi_\lambda\}$ is a ONB of eigenfunctions of $D$ corresponding to the
eigenvalue $\lambda$
\begin{align}
    K(t;x,y;D) = \sum_\lambda \phi^\dagger_\lambda(x)
    \phi_\lambda(y)e^{-t\lambda}.
\end{align}
We have an asymtotic expansion at $t \rightarrow 0$  for the trace
\begin{align}
    Tr_{L^2}(fe^{-tD}) \simeq \sum_{k\geq 0}t^{(k-n)/2}a_k(f,D).
\end{align}
where
\begin{align}
    a_k(f,D) = (4\pi)^{-n/2} \int_M d^4x \sqrt{g} b_k(x,x) f(x)
\end{align}
\subsubsection{General Formulae}
We consider a compact Riemmanian Manifold $M$ without boundary condition, a
vector bundle $V$ over $M$ to define functions which carry discrete (spin or
gauge) indices. An Laplace style operator $D$ over $V$ and smooth function $f$
on $M$. There is an asymtotic expansion where the heat kernel coefficients
\begin{enumerate}
    \item with odd index $k=2j+1$ vanish
        $a_{2j+1}(f,D) = 0$
    \item with even index are locally computable in terms of geometric
        invariants
\end{enumerate}
\begin{align}
    a_k(f,D) &= \text{Tr}_V\left(\int_M d^n x\sqrt{g}(f(x)a_k(x;D)\right) =\\
    &=\sum_I \text{Tr}_V\left(\int_M d^nx \sqrt{g}(fu^I \mathcal{A}^I_k(D))\right)
\end{align}
here $\mathcal{K}^I_k$ are all possible independent invariants of dimension
$k$, constructed from $E, \Omega, R_{\mu\nu\varrho\sigma}$ and their
derivatives, $u^I$ are some constants.

If $E$ has dimension two, then the derivative has dimension one. So if $k=2$
there are only two independent invariants, $E$ and $R$. This corresponds to the
statement $a_{2j+1}=0$.

If we consider $M = M_1 \times M_2$ with coordinates $x_1$ and $x_2$ and a
decomposed Laplace style operator $D = D_1 \otimes 1 + 1 \otimes D_2$ we can
separate everything, i.e.
\begin{align}
    e^{-tD} &= e^{-tD_1} \otimes e^{-tD_2}\\
    f(x_1, x_2) &= f_1(x_1)f_2(x_2)\\
    a_k(x;D) &= \sum_{p+q=k} a_p(x_1; D_1)a_q(x_2;D_2)
\end{align}
Say the spectrum of $D_1$ is known, $l^2, l\in \mathbb{Z}$. We obtain the heat
kernel asymmetries with the Poisson Summation formula
\begin{align}
    K(t, D_1) &= \sum_{l\in\mathbb{Z}} e^{-tl^2} = \sqrt{\frac{\pi}{t}}
    \sum_{l\in\mathbb{Z}} e^{-\frac{\pi^2l^2}{t}} = \\
    &\simeq \sqrt{\frac{\pi}{t}} + \mathcal{O}(e^{-1/t}).
\end{align}
Note that the exponentially small terms have no effect on the heat kernel
coefficients and that the only nonzero coefficient is $a_0(1, D_1) =
\sqrt{\pi}$. Therefore we can write
\begin{align}
    a_k(f(x^2), D) = \sqrt{\pi}\int_{M_2}
    d^{n-1}x\sqrt{g}\sum_I\text{Tr}_V\left(f(x^2)u^I_{(n-1)}
    \mathcal{A}^I_n(D_2)\right).
\end{align}

On the other had all geometric invariants associated with $D$ are in the $D_2$
part. Thus all invariants are independent of $x_1$, so we can choose for $M_1$.
Say $M_1 = S^1$ with $x\in (0, 2\pi)$ and $D_1=-\partial_{x_1}^2$ we may
rewrite the heat kernel coefficients in
\begin{align}
    a_k(f(x_2), D) &= \int_{S^1\times M_2}d^nx \sqrt{g} \sum_I
    \text{Tr}_V(f(x_2) u_{(n)}^I \mathcal{A}^I_k(D_2))=\\
    &= 2\pi \int_{M_2} d^nx\sqrt{g} \sum_I\text{Tr}_V(f(x_2) u_{(n)}^I
    \mathcal{A}^I_k(D_2)).
\end{align}
Computing the two equations above we see that
\begin{align}
    u_{(n)}^I = \sqrt{4\pi} u^I_{(n+1)}
\end{align}

\subsubsection{Heat Kernel Coefficients}
To calculate the heat kernel coefficients we need the following variational
equations
\begin{align}
    &\frac{d}{d\varepsilon}|_{\varepsilon=0}a_k(1, e^{-2\varepsilon f}D) =
    (n-k) a_k(f, D),\label{eq:var1}\\
    &\frac{d}{d\varepsilon}|_{\varepsilon=0}a_k(1, D-\varepsilon F) =
    a_{k-2}(F,D),\label{eq:var2}\\
    &\frac{d}{d\varepsilon}|_{\varepsilon=0}a_k(e^{-2\varepsilon f}F,
    e^{-2\varepsilon f}D) =
    0\label{eq:var3}.
\end{align}
To prove the equation \ref{eq:var1} we differentiate
\begin{align}
    \frac{d}{d\varepsilon}|_{\varepsilon=0} \text{Tr}(\exp(-e^{-2\varepsilon
    f}tD) = \text{Tr}(2ftDe^{-tD}) = -2t\frac{d}{dt}\text{Tr}(fe^{-tD}))
\end{align}
then we expand both sides in $t$ and get \ref{eq:var1}. Equation \ref{eq:var2}
is derived similarly. For equation \ref{eq:var3} we consider the following
operator
\begin{align}
    D(\varepsilon,\delta) = e^{-2\varepsilon f}(D-\delta F)
\end{align}
for $k=n$ we use equation \ref{eq:var1} and we get
\begin{align}
    \frac{d}{d\varepsilon}|_{\varepsilon=0}a_n(1,D(\varepsilon,\delta)) =0
\end{align}
then we take the variation in terms of $\delta$, evaluated at $\delta =0$ and
swap the differentiation, allowed by theorem of Schwarz
\begin{align}
    0 &=
    \frac{d}{d\delta}|_{\delta=0}\frac{d}{d\varepsilon}|_{\varepsilon=0}a_n(1,
    D(\varepsilon,\delta)) =
    \frac{d}{d\varepsilon}|_{\varepsilon=0}\frac{d}{d\delta}|_{\delta=0}a_n(1,
    D(\varepsilon,\delta)) =\\
    &=a_{n-2} ( e^{-2\varepsilon f}F, e^{-2\varepsilon f}D)
\end{align}
which proves equation \ref{eq:var3}. With this we calculate the constants $u^I$
and we can write the first three heat kernel coefficients as
\begin{align}
    a_0(f, D) &= (4\pi)^{-n/2}\int_Md^n x\sqrt{g} \text{Tr}_V(a_0 f)\\
    a_2(f, D) &= (4\pi)^{-n/2}\frac{1}{6}\int_Md^n
    x\sqrt{g}\text{Tr}_V)(f\alpha _1 E+\alpha _2 R)\\
    a_4(f, D) &= (4\pi)^{-n/2}\frac{1}{360}\int_Md^n
    x\sqrt{g}\text{Tr}_V(f(\alpha_3 E_{,kk} + \alpha_4 RE + \alpha_5 E^2
    \alpha_6 R_{,kk} + \\
    &+\alpha_7 R^2 + \alpha_8 R_{ij}R_{ij} + \alpha_9
    R_{ijkl}R_{ijkl} +\alpha_{10} \Omega_{ij}\Omega{ij})).
\end{align}
The constants $\alpha_I$ do not depend on the dimension $n$ of the Manifold and
we can compute them with our variational identities.

The first coefficient $\alpha_0$ can be seen from the heat kernel expanion of
the Laplacian on $S^1$ (above), $\alpha_0 = 1$. For $\alpha_1$ we use
\ref{eq:var2}, for $k = 2$
\begin{align}
    \frac{1}{6} \int_M d^n x\sqrt{g} \text{Tr}_V(\alpha_1F) = \int_M d^n
    x\sqrt{g} \text{Tr}_V(F),
\end{align}
thus we conclude that $\alpha_1 = 6$. Now we take $k=4$
\begin{align}
    \frac{1}{360}\int_Md^n x \sqrt{g}\text{Tr}_V(\alpha_4 F R + 2\alpha_5 F E)
    = \frac{1}{6} \int_Md^n x\sqrt{g}\text{Tr}_V(\alpha_1 FE + \alpha_2 FR),
\end{align}
thus $\alpha_4 = 60\alpha_2$ and $\alpha_5 = 180$.

Furthermore we apply \ref{eq:var3} to $n=4$
\begin{align}
    \frac{d}{d\varepsilon}|_{\varepsilon=0} a_2(e^{-2\varepsilon f}F,
    e^{-2\varepsilon f}D) = 0.
\end{align}
By collecting the terms with $\text{Tr}_V(\int_Md^nx\sqrt{g}(Ff_{,jj}))$ we
obtain $\alpha_1 = 6\alpha_2$, that is $\alpha_2 = 1$, so $\alpha_4 = 60$.

Now we let $M=M_1\times M_2$ and split $D = -\Delta_1 -\Delta_2$, where
$\Delta_{1/2}$ are Laplacians for $M_1, M_2$, then we can decompose the heat
kernel coefficients for $k=4$
\begin{align}
    a_4(1,-\Delta_1-\Delta_2) =& a_4(1, -\Delta_1) a_0(1, -\Delta_2)
    +a_2(1,-\Delta_1) a_2(1,-\Delta_2) \\&+ a_0(1,-\Delta_1) a_4(1,-\Delta_2)
\end{align}
with $E=0$ and $\Omega =0$ and by calculating the terms with $R_1R_2$  (scalar
curvature of $M_{1/2}$) we obtain $\frac{2}{360}\alpha_7 =
(\frac{\alpha_2}{6})^2$, thus $\alpha_7 = 5$.

For $n=6$ we get
\begin{align}
    0 &= \text{Tr}_V(\int_Md^nx\sqrt{g}
    (F(-2\alpha_3-10\alpha_4+4\alpha_5)f_{,kk}E +\\
    &+(2\alpha_3 + 10\alpha_6)f_{,iijj}+\\
    &+(2\alpha_4 -2\alpha_6 - 20\alpha_7 -2\alpha_8)f_{,ii}R\\
    &+(-8\alpha_8 -8\alpha_6)f_{,ij}R_{ij}))
\end{align}
we obtain $\alpha_3 = 60$, $\alpha_6=12$, $\alpha_8 = -2$ and $\alpha_9 = 2$

For $\alpha_{10}$ we use the Gauss-Bonnet theorem to get $\alpha_{10}=30$,
which is left out because it is a lengthy computation.

Summarizing we get for the heat kernel coefficients
\begin{align}
    \alpha_0(f, D) &= (4\pi)^{-n/2}\int_M d^n x \sqrt{g} \text{Tr}_V(f)\\
    \alpha_2(f, D) &= (4\pi)^{-n/2}\frac{1}{6}\int_M d^n x \sqrt{g}
    \text{Tr}_V(f(6E+R))\\
    \alpha_4(f, D) &= (4\pi)^{-n/2}\frac{1}{360}\int_M d^n x \sqrt{g}
    \text{Tr}_V(f(60E_{,kk}+60RE+ 180E^2 +\\
    &+12R_{,kk} + 5R^2 - 2 R_{ij}R_{ij}
    2R_{ijkl}R_{ijkl} +30\Omega_{ij}\Omega_{ij}))\\
\end{align}

