\subsection{Noncommutative Geometry of Electrodynamics\label{sec:5}}
In this chapter we go through a derivation Electrodynamics with
the almost commutative manifold $M\times F_X$ and the abelian gauge group
$U(1)$. The conclusion is an unified description of gravity and
electrodynamics although in the classical level.

The almost commutative Manifold $M\times F_X$ outlines a local gauge group
$U(1)$. The inner fluctuations of the Dirac operator relate to $Y_\mu$ the
gauge field of $U(1)$. According to the setup we ultimately arrive at two
serious problems.

First of all the operator $D_F$, in the Two-Point space $F_X$, must vanish
such that a real structure can exists. However this implies that the electrons
are massless.

The second problem arises when looking at the Euclidean action for a free
Dirac field
\begin{align}
    S = - \int i \bar{\psi}(\gamma ^\mu\partial _\mu - m) \psi d^4x,
\end{align}
where $\psi,\ \bar{\psi}$ must be considered as two independent variables.
This means that the fermionic action $S_f$ needs two independent Dirac spinors.
Let us try and construct two independent Dirac spinors with our data, first
take a look at the decomposition of the basis and of the total
Hilbertspace $H = L^2(S) \otimes H_F$. For the orthonormal basis of $H_F$ we
can write $\{e, \bar{e}\}$ , where $\{e\}$ is the orthonormal basis of
$H_F^+$ and $\{\bar{e}\}$ the orthonormal basis of $H_F^-$. Accompanied with
the real structure we arrive at the following relations
\begin{align}
    J_F e &= \bar{e} \;\;\;\;\;\; J_F \bar{e} = e, \\
    \gamma_F e &= e  \;\;\;\;\;\;   \gamma_F \bar{e} = \bar{e}.
\end{align}
Along with the decomposition of $L^2(S) = L^2(S)^+ \oplus L^2(S)^-$ and  $\gamma = \gamma _M
\otimes \gamma _F$ we can obtain the positive eigenspace
\begin{align}
    H^+ = L^2(S)^+ \otimes H_F^+ \oplus L(S)^- \otimes H_F^-.
\end{align}
So, for an $\xi \in H^+$ we can write
\begin{align}
    \xi = \psi _L \otimes e + \psi _R \otimes \bar{e},
\end{align}
where $\psi_L \in L^2(S)^+$ and $\psi _R \in L^2(S)^-$ are the two Wheyl
spinors. We denote that $\xi$ is only determined by one Dirac spinor $\psi :=
\psi_L + \psi _R$. Since \textbf{we require two independent spinors}, our
conclusion is that the definition of the fermionic action gives too much
restrictions to the Two-Point space $F_X$.
\subsubsection{The Finite Space}
To solve the two problems we simply enlarge (double) the Hilbertspace. This
is visualized by introducing multiplicities in Krajewski Diagrams
\cite{ncgwalter} which will also allow us to choose a nonzero Dirac operator
that will connect the two vertices and preserve real structure making our
particles massive and bringing anti-particles into the mix.

We start of with the same algebra $C^\infty(M, \mathbb{C}^2)$, corresponding
to space $N= M\times X$. The Hilbertspace describes four particles, meaning
it has four orthonormal basis elements. It describes \textbf{left handed
electrons} and \textbf{right handed positrons}. This way we have
$\{ \underbrace{e_R, e_L}_{\text{left-handed}}, \underbrace{\bar{e}_R,
\bar{e}_L}_{\text{right-handed}}\}$ an orthonormal basis for $H_F =
\mathbb{C}^4$. Accompanied with the real structure $J_F$ allowing us to
interchange particles with antiparticles by the following equations
\begin{align}
    &J_F e_R = \bar{e}_R, \\
    &J_F e_L = \bar{e_L}, \\
    \nonumber \\
    &\gamma _F e_R = -e_R,\\
    &\gamma_F e_L = e_L,
\end{align}
where $J_F$ and $\gamma_F$ have to following properties
\begin{align}
    &J_F^2 = 1,\\
    & J_F \gamma_F  = - \gamma_F J_F.
\end{align}
By the means of $\gamma_F$ we have two options to decompose the total
Hilbertspace $H$, firstly into
\begin{align}
    H_F = \underbrace{H_F^+}_{\text{ONB } \{e_L, \bar{e}_L\}}
    \oplus \underbrace{H_F^-}_{\text{ONB } \{e_R, \bar{e}_R\}},
\end{align}
or alternatively into the eigenspace of particles and their
antiparticles (electrons and positrons) which is preferred in literature and
which will be used further out
\begin{align}
    H_F = \underbrace{H_{e}}_{\text{ONB } \{e_L, e_R\}} \oplus
    \underbrace{H_{\bar{e}}}_{\text{ONB } \{\bar{e}_L, \bar{e}_R\}},
\end{align}
the shortening `ONB' means orthonormal basis.

The action of $a \in A = \mathbb{C}^2$ on $H$ with respect to the ONB
$\{e_L, e_R, \bar{e}_L, \bar{e}_R\}$ is represented by
\begin{align}\label{eq:leftrightrepr}
    a =
    (a_1 , a_2 ) \mapsto
        \begin{pmatrix}
            a_1 &0 &0 &0\\
             0&a_1 &0 &0\\
            0 &0 &a_2 &0\\
            0 &0 &0 &a_2\\
        \end{pmatrix}
\end{align}
Do note that this action commutes wit the grading and that $[a, b^\circ] = 0$
with $b:= J_F b^*J_F$ because both the left and the right action are given by
diagonal matrices according to equation \eqref{eq:leftrightrepr}. Furthermore
note that we are still left with $D_F = 0$ and the following spectral triple
\begin{align}\label{eq:fedfail}
        \left( \mathbb{C}^2, \mathbb{C}^2, D_F=0; J_F =
        \begin{pmatrix}
            0 & C \\ C &0
        \end{pmatrix},
        \gamma _F =
        \begin{pmatrix}
            1 & 0 \\ 0 &-1
        \end{pmatrix}
        \right).
    \end{align}
It can be represented in the following Krajewski diagram,
with two nodes of multiplicity two bellow
    \begin{figure}[H] \centering
    \begin{tikzpicture}[
        dot/.style = {draw, circle, inner sep=0.06cm},
        bigdot/.style = {draw, circle, inner sep=0.09cm},
        no/.style = {},
        ]
        \node[no](a) at (0,0) [label=left:$\textbf{1}^\circ$] {};
        \node[no](b) at (0, -1) [label=left:$\textbf{1}^\circ$] {};
        \node[no](c) at (0.5, 0.5) [label=above:$\textbf{1}$] {};
        \node[no](d) at (1.5, 0.5) [label=above:$\textbf{1}$] {};
        \node[dot](d0) at (1.5,0) [] {};
        \node[dot](d0) at (0.5,-1) [] {};
        \node[bigdot](d0) at (1.5,0) [] {};
        \node[bigdot](d0) at (0.5,-1) [] {};
        \end{tikzpicture}
        \caption{Krajewski diagram of the spectral triple from equation \ref{eq:fedfail}}
    \end{figure}
\subsubsection{A noncommutative Finite Dirac Operator}
To extend our spectral triple with a non-zero Operator, we need to take a
closer look at the Krajewski diagram above. Notice that edges only exist
between multiple vertices, meaning we can construct a Dirac operator mapping
between the two vertices. The operator can be represented by the following matrix
\begin{align}\label{eq:feddirac}
    D_F =
    \begin{pmatrix}
    0 & d & 0 & 0 \\
    \bar{d} & 0 & 0 & 0 \\
    0 & 0 & 0 & \bar{d} \\
    0 & 0 & d & 0
    \end{pmatrix}
\end{align}
We can now define the finite space $F_{ED}$.
\begin{align}
    F_{ED} := (\mathbb{C}^2, \mathbb{C}^4, D_F; J_F, \gamma_F)
\end{align}
where $J_F$ and $\gamma_F$ are as in equation \eqref{eq:fedfail} and $D_F$
from equation \eqref{eq:feddirac}.

\subsubsection{Almost commutative Manifold of Electrodynamics}
The almost commutative manifold $M\times F_{ED}$ has KO-dimension 2, and is
represented by the following spectral triple
\begin{align}\label{eq:almost commutative manifold}
    M\times F_{ED} := \big(C^\infty(M,\mathbb{C}^2),\ L^2(S)\otimes
    \mathbb{C}^4,\
    D_M\otimes 1 +\gamma _M \otimes D_F;\; J_M\otimes J_F,\ \gamma_M\otimes
    \gamma _F\big).
\end{align}
The algebra didn't change, thus we can decompose it like before
\begin{align}
    C^\infty(M, \mathbb{C}^2) = C^\infty (M) \oplus C^\infty (M).
\end{align}
As for the Hilbertspace, we can decomposition it in the following way
\begin{align}
    H = (L^2(S) \otimes H_e ) \oplus (L^2(S) \otimes H_{\bar{e}}).
\end{align}
Note that the one component of the algebra is acting on $L^2(S) \otimes H_e$,
and the other one acting on $L^2(S) \otimes H_{\bar{e}}$. In other words the components of
the decomposition of both the algebra and the Hilbertspace match by the action of
the algebra.

The derivation of the gauge theory is the same for $F_{ED}$ as for the
Two-Point space $F_X$. We have $\mathfrak{B}(F) \simeq U(1)$ and for an
arbitrary gauge field $B_\mu = A_\mu - J_F A_\mu J_F^{-1}$ we can write
\begin{align} \label{field}
    B_\mu =
    \begin{pmatrix}
        Y_\mu & 0 & 0 & 0 \\
        0 & Y_\mu& 0 & 0 \\
        0 & 0 & Y_\mu& 0 \\
        0 & 0 & 0 & Y_\mu
    \end{pmatrix} \;\;\;\;\;\ \text{for} \;\;\ Y_\mu (x) \in \mathbb{R}.
\end{align}
There is one single $U(1)$ gauge field $Y_\mu$, carrying the action of the
gauge group
\begin{align}
   \text{$\mathfrak{B}$}(M\times F_{ED}) \simeq C^\infty(M, U(1))
\end{align}

The space $N = M\times X$ consists of two copies of $M$.
If $D_F = 0$ we have infinite distance between the two copies, yet now we have
adjusted the spectral triple to have a nonzero Dirac operator. The new
Dirac operator still has a commuting relation with the algebra $[D_F, a] = 0$
$\forall a \in A$, and we should note that the distance between the two
copies of $M$ is still infinite. This is purely an mathematically abstract
observation and doesn't affect physical results.

\subsubsection{Spectral Action}
In this chapter we bring all our results together to establish an
Action functional to describe a physical system. It turns out that
the Lagrangian of the almost commutative manifold $M\times F_{ED}$
corresponds to the Lagrangian of Electrodynamics on a curved
background manifold (+ gravitational Lagrangian), consisting of the spectral
action $S_b$ (bosonic) and of the fermionic action $S_f$.

The simplest spectral action of a spectral triple $(A, H, D)$ is given by the
trace of a function of $D$. We also consider inner fluctuations of the Dirac
operator
\begin{align}
    D_\omega = D + \omega + \varepsilon' J\omega J^{-1},
\end{align}
where $\omega = \omega ^* \in \Omega_D^1(A)$.
\begin{mydefinition}
    Let $f:\mathbb{R} \rightarrow \mathbb{R}$ be a suitable function
    \textbf{positive and even}. The spectral action is then
    \begin{align}
        S_b [\omega] := \text{Tr}\big(f(\frac{D_\omega}{\Lambda})\big)
    \end{align}
    where $\Lambda$ is a real cutoff parameter. The minimal condition on $f$
    is that $f(\frac{D_\omega}{\Lambda})$ is  a trace class operator. A trace
    class operator is a compact operator with a well defined finite trace
    independent of the basis. The subscript $b$ in $S_b$ stands for bosonic,
    because in physical applications $\omega$ will describe bosonic fields.

    In addition to the bosonic action $S_b$, we can define a topological spectral
    action $S_{top}$. Leaning on the grading $\gamma$ the topological spectral action is
    \begin{align}
        S_{\text{top}}[\omega] := \text{Tr}(\gamma\
        f(\frac{D_\omega}{\Lambda})).
    \end{align}
\end{mydefinition}
\begin{mydefinition}\label{def:fermionic action}
    The fermionic action is defined by
    \begin{align}
        S_f[\omega, \psi] = (J\tilde{\psi}, D_\omega \tilde{\psi})
    \end{align}
    with $\tilde{\psi} \in H_{cl}^+ := \{\tilde{\psi}: \psi \in H^+\}$, where
    $H_{cl}^+$ is a set of Grassmann variables in $H$ in the $+1$-eigenspace
    of the grading $\gamma$.
\end{mydefinition}

%---------------------- APPENDIX ?????????????--------------------
Grassmann variables are a set of Basis vectors of a vector space, they
form a unital algebra over a vector field $V$, where the generators are
anti commuting, that is for Grassmann variables $\theta _i, \theta _j$  we have
\begin{align}
    &\theta _i \theta _j = -\theta _j \theta _i, \\
    &\theta _i x = x\theta _j \;\;\;\; x\in V, \\
    &(\theta_i)^2 = 0 \;\;\; (\theta _i \theta _i = -\theta _i \theta _i).
\end{align}
%---------------------- APPENDIX ?????????????--------------------
\begin{myproposition}
    The spectral action of the almost commutative manifold $M$ with $\dim(M)
    =4$ with a fluctuated Dirac operator is
    \begin{align}
        \text{Tr}(f\frac{D_\omega}{\Lambda}) \sim \int_M \mathcal{L}(g_{\mu\nu},
         B_\mu, \Phi) \sqrt{g}\ d^4x + O(\Lambda^{-1}),
    \end{align}
    where
    \begin{align}
        \mathcal{L}(g_{\mu\nu}, B_\mu, \Phi) =
        N\mathcal{L}_M(g_{\mu\nu})
        \mathcal{L}_B(B_\mu)+
        \mathcal{L}_\phi(g_{\mu\nu}, B_\mu, \Phi).
    \end{align}
    The Lagrangian $\mathcal{L}_M$ is of the spectral triple $(C^\infty(M) ,
    L^2(S), D_M)$, represented by the following term
    \begin{align}\label{lagr}
        \mathcal{L}_M(g_{\mu\nu}) := \frac{f_4 \Lambda ^4}{2\pi^2} -
        \frac{f_2 \Lambda^2}{24\pi ^2}s - \frac{f(0)}{320\pi^2} C_{\mu\nu
        \varrho \sigma}C^{\mu\nu \varrho \sigma},
    \end{align}
    where $C^{\mu\nu \varrho \sigma}$ is the Weyl tensor defined in terms of the Riemannian
    curvature tensor $R_{\mu\nu \varrho \sigma}$ and the Ricci tensor
    $R_{\nu\sigma} = g^{\mu\varrho} R_{\mu\nu \varrho\sigma}$ such that
    \begin{align}
        C^{\mu\nu\varrho\sigma}C_{\mu\nu\varrho\sigma}=
        R_{\mu\nu\varrho\sigma}R^{\mu\nu\varrho\sigma} -
        2R_{\nu\sigma}R^{\nu\sigma} + \frac{1}{2}s^2.
    \end{align}
    The kinetic term of the gauge field is described by the Lagrangian
    $\mathcal{L}_B$, which takes the following shape
    \begin{align}
        \mathcal{L}_B(B_\mu) := \frac{f(0)}{24\pi^2}
        \text{Tr}(F_{\mu\nu}F^{\mu\nu}).
    \end{align}
    Lastly $\mathcal{L}_\phi$ is the scalar-field Lagrangian with a boundary
    term, given by
    \begin{align}
        \mathcal{L}_\phi(g_{\mu\nu}, B_\mu, \Phi) :=
        &-\frac{2f_2\Lambda^2}{4\pi^2}\text{Tr}(\Phi^2) + \frac{f(0)}{8\pi^2}
        \text{Tr}(\Phi^4) + \frac{f(0)}{24\pi^2}
        \Delta(\text{Tr}(\Phi^2))\nonumber\\
        &+ \frac{f(0)}{48\pi^2}s\text{Tr}(\Phi^2)
        \frac{f(0)}{8\pi^2}\text{Tr}((D_\mu \Phi)(D^\mu \Phi)).
    \end{align}
\end{myproposition}
\begin{proof}
     The dimension of the manifold $M$ is $\dim(M) = \text{Tr}(id) =4$. For
     an $x \in M$, we have an asymptotic expansion of the term
     $\text{Tr}(f(\frac{D_\omega}{\Lambda}))$ as $\Lambda$ goes to infinity,
     which can be written as
     \begin{align}
         \text{Tr}(f(\frac{D_\omega}{\Lambda})) \simeq& \ 2f_4 \Lambda ^4
         a_0(D_\omega ^2)+ 2f_2\Lambda^2 a_2(D_\omega^2)\nonumber \\&+ f(0) a_4(D_\omega^4)
         +O(\Lambda^{-1}).\label{eq:trheatkernel}
     \end{align}
     We have to note here that the heat kernel coefficients are zero for uneven $k$,
     and they are dependent on the fluctuated Dirac operator
     $D_\omega$. We can rewrite the heat kernel coefficients in terms of $D_M$,
     for the first two terms $a_0$ and $a_2$ we use $N:=
     \text{Tr}(\mathbbm{1}_{H_F})$ and one obtains
     \begin{align}
         a_0(D_\omega^2) &= Na_0(D_M^2),\\
         a_2(D_\omega^2) &= Na_2(D_M^2) - \frac{1}{4\pi^2}\int_M
         \text{Tr}(\Phi^2)\sqrt{g}d^4x.
     \end{align}
     For $a_4$ we extend in terms of coefficients of $F$ from equation
     \eqref{eq: a_4}
     \begin{align}
         &\frac{1}{360}\text{Tr}(60RE)= -\frac{1}{6}S(NR + 4
         \text{Tr}(\Phi^2))\\
        \nonumber\\
         &E^2 = \frac{1}{16}R^2\otimes 1 + 1\otimes \Phi^4 - \frac{1}{4}
         \gamma^\mu\gamma^\nu \gamma^\varrho\gamma^\sigma
         F_{\mu\nu}F^{\mu\nu}+\nonumber\\
         &\;\;\;\;\;\;\;+\gamma^\mu\gamma^\nu\otimes(D_\mu\Phi)(D_\nu
         \Phi)+\frac{1}{2}s\otimes \Phi^2 + \ \text{traceless terms},\\
         \nonumber\\
         &\frac{1}{360}\text{Tr}(180E^2) = \frac{1}{8}R^2N + 2\text{Tr}(\Phi^4)
         + \text{Tr}(F_{\mu\nu}F^{\mu\nu}) +\nonumber\\
         &\;\;\;\;\;\;\;+2\text{Tr}((D_\mu\Phi)(D^\mu\Phi))
         + s\text{Tr}(\Phi^2)\\
         \nonumber\\
         &\frac{1}{360}\text{Tr}(-60\Delta E)=
         \frac{1}{6}\Delta(NR+4\text{Tr}(\Phi^2)).
     \end{align}
     The cross terms of the trace in $\Omega_{\mu\nu}^E\Omega^{E\mu\nu}$
     vanishes because of the antisymmetric property of the Riemannian
     curvature tensor, reading
     \begin{align}
         \Omega_{\mu\nu}^E\Omega^{E\mu\nu} = \Omega_{\mu\nu}^S\Omega^{S\mu\nu}
         \otimes 1 - 1\otimes F_{\mu\nu}F^{\mu\nu} + 2i\Omega_{\mu\nu}^S
         \otimes F^{\mu\nu}.
     \end{align}
     The trace  of the cross term $\Omega^{S}_{\mu\nu}$ vanishes because
     \begin{align}
         \text{Tr}(\Omega^{S}_{\mu\nu}) = \frac{1}{4}
         R_{\mu\nu\varrho\sigma}\text{Tr}(\gamma^\mu\gamma^\nu) = \frac{1}{4}
         R_{\mu\nu\varrho\sigma}g^{\mu\nu} =0,
     \end{align}
     then the trace of the whole term is given by
     \begin{align}
         \frac{1}{360}\text{Tr}(30\Omega^E_{\mu\nu}\Omega^{E\mu\nu}) =
         \frac{N}{24}R_{\mu\nu\varrho\sigma}R^{\mu\nu\varrho\sigma}
         -\frac{1}{3}\text{Tr}(F_{\mu\nu}F^{\mu\nu}).
     \end{align}
     Finally plugging the results into the coefficient $a_4$ and simplifying
     one gets
     \begin{align}
         a_4(x, D_\omega^4) &= Na_4(x, D_M^2) + \frac{1}{4\pi^2}\bigg(\frac{1}{12} s
         \text{Tr}(\Phi^2) + \frac{1}{2}\text{Tr}(\Phi^4) \nonumber \\
         &+ \frac{1}{4}
         \text{Tr}((D_\mu\Phi)(D^\mu \Phi)) + \frac{1}{6}
         \Delta\text{Tr}(\Phi^2) + \frac{1}{6}
         \text{Tr}(F_{\mu\nu}F^{\mu\nu})\bigg).
     \end{align}
     The only thing left is to substitute the heat kernel coefficients into the
     heat kernel expansion in equation \eqref{eq:trheatkernel}.
\end{proof}

\subsubsection{Fermionic Action}
We remind ourselves the definition of the fermionic action in definition
\ref{def:fermionic action} and the manifold we are dealing with in equation
\eqref{eq:almost commutative manifold}. The Hilbertspace $H_F$ is separated
into the particle-antiparticle states with ONB $\{e_R, e_L, \bar{e}_R,
\bar{e}_L\}$. The orthonormal basis of $H_F^+$ is $\{e_L, \bar{e}_R\}$ and
consequently for $H_F^-$, $\{e_R, \bar{e}_L\}$. The decomposition of a spinor
$\psi \in L^2(S)$ in each of the eigenspaces $H_F^\pm$ is $\psi = \psi_R+
\psi_L$. Meaning for an arbitrary $\psi \in H^+$ we can write
\begin{align}
    \psi = \chi_R \otimes e_R + \chi_L \otimes e_L + \psi_L \otimes
    \bar{e}_R+
    \psi_R \otimes \bar{e}_L,
\end{align}
where $\chi_L, \psi_L \in L^2(S)^+$ and $\chi_R, \psi_R \in L^2(S)^-$.

Since the fermionic action yields too much restriction on $F_{ED}$ (modified
Two-Point space $F_X$) one redefines it by taking into account the fluctuated Dirac
operator
\begin{align}
    D_\omega = D_M \otimes i + \gamma^\mu \otimes B_\mu + \gamma_M \otimes
    D_F.
\end{align}
The Fermionic Action is
\begin{align}
S_F = (J\tilde{\xi}, D_\omega\tilde{\xi})
\end{align}
for a $\xi \in H^+$. Then the straight forward calculation gives \begin{align}
    \frac{1}{2}(J\tilde{\xi}, D_\omega\tilde{\xi})
        &=\frac{1}{2}(J\tilde{\xi}, (D_M \otimes
        i)\tilde{\xi})\label{eq:fermionic1}\\
        &+\frac{1}{2}(J\tilde{\xi}, (\gamma^\mu \otimes B_\mu)
        \tilde{\xi})\label{eq:fermionic2}\\
        &+\frac{1}{2}(J\tilde{\xi}, (\gamma_M\otimes
        D_F)\tilde{\xi})\label{eq:fermionic3},
\end{align}
(note that we add the constant $\frac{1}{2}$ to the action).
For the term in \eqref{eq:fermionic1} we calculate
\begin{align}
    \frac{1}{2}(J\tilde{\xi}, (D_M\otimes 1)\tilde{\xi}) &=
    \frac{1}{2}(J_M\tilde{\chi}_R,D_M\tilde{\psi}_L)+\nonumber
    \frac{1}{2}(J_M\tilde{\chi}_L,D_M\tilde{\psi}_R)+
    \\&+\frac{1}{2}(J_M\tilde{\psi}_L,D_M\tilde{\psi}_R)+\nonumber
    \frac{1}{2}(J_M\tilde{\chi}_R,D_M\tilde{\chi}_L)\\
    &= (J_M\tilde{\chi},D_M\tilde{\chi}).
\end{align}
For the term in \eqref{eq:fermionic2} we have
\begin{align}
    \frac{1}{2}(J\tilde{\xi}, (\gamma^\mu \otimes B_\mu)\tilde{\xi})&=
    -\frac{1}{2}(J_M\tilde{\chi}_R, \gamma^\mu Y_\mu\tilde{\psi}_R)
    -\frac{1}{2}(J_M\tilde{\chi}_L, \gamma^\mu Y_\mu\tilde{\psi}_R)+\nonumber\\
    &+\frac{1}{2}(J_M\tilde{\psi}_L, \gamma^\mu Y_\mu\tilde{\chi}_R)+
    \frac{1}{2}(J_M\tilde{\psi}_R, \gamma^\mu Y_\mu\tilde{\chi}_L)=\nonumber\\
    &= -(J_M\tilde{\chi}, \gamma^\mu Y_\mu\tilde{\psi}).
\end{align}
And for \eqref{eq:fermionic3} we can write
\begin{align}
    \frac{1}{2}(J\tilde{\xi}, (\gamma_M\otimes D_F)\tilde{\xi})&=
    +\frac{1}{2}(J_M\tilde{\chi}_R, d\gamma_M\tilde{\chi}_R)
    +\frac{1}{2}(J_M\tilde{\chi}_L, \bar{d}\gamma_M\tilde{\chi}_L)+\nonumber\\
    &+\frac{1}{2}(J_M\tilde{\chi}_L, \bar{d}\gamma_M\tilde{\chi}_L)
    +\frac{1}{2}(J_M\tilde{\chi}_R, d\gamma_M\tilde{\chi}_R)=\nonumber\\
    &= i(J_M\tilde{\chi}, m\tilde{\psi}).
\end{align}
A small problem arises, we obtain a complex mass parameter $d$, but we can
write $d:=im$ for $m\in \mathbb{R}$, which stands for the real mass.

Finally the fermionic action of $M\times F_{ED}$ takes the form
    \begin{align}
        S_f = -i\big(J_M\tilde{\chi}, \gamma(\nabla^S_\mu - i\Gamma_\mu)
        \tilde{\Psi}\big) + \big(S_M\tilde{\chi}_L, \bar{d}\tilde{\psi}_L\big) -
        \big(J_M\tilde{\chi}_R, d \tilde{\psi}_R\big).
    \end{align}
Ultimately we arrive at the full Lagrangian of the almost commutative
manifold $M\times F_{ED}$, which is the sum of the purely gravitational
Lagrangian
\begin{align}\label{eq:final1}
        \mathcal{L}_{grav}(g_{\mu\nu})=4\mathcal{L}_M(g_{\mu\nu})+
        \mathcal{L}_\phi (g_{\mu\nu}),
    \end{align}
and the Lagrangian of electrodynamics
\begin{align}\label{eq:final2}
        \mathcal{L}_{ED} = -i\bigg\langle
        J_M\tilde{\chi},\big(\gamma^\mu(\nabla^S_\mu - iY_\mu) -m\big)\tilde{\psi})
        \bigg\rangle
        +\frac{f(0)}{6\pi^2} Y_{\mu\nu}Y^{\mu\nu}.
    \end{align}

