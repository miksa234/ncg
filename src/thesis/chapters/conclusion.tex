\section{Conclusion}
We conclude that the framework of noncommutative geometry can fully describe
the physics of electrodynamics. This is done by introducing the spectral and
fermionic action principles of the almost commutative manifold $M \times F_{ED}$
constructed from a four dimensional Riemannian spin manifold and a
modification of the two point space $F_X$. By going through rough
calculations of the heat kernel coefficients to describe the Lagrangian in
terms of geometrical invariants we finally arrive at the Lagrangians in
equations \eqref{eq:final1} and \eqref{eq:final2}.

With a similar complex ansatz Walter D. Suijlekom describes in his book
\cite{ncgwalter} how to figure out a specific version of a spectral triple
corresponding the almost commutative manifold which delivers the physics of
the full Standard Model and with this information accurately calculating the
mass of the Higgs boson. Moreover he describes more accurately the
correspondence of the gauge theory of an almost commutative manifold, which
brings this noncommutative geometry to the interest of physicists in the first place.
