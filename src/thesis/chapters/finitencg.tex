\subsection{Finite Spectral Triples}
\subsubsection{Metric on Finite Discrete Spaces}
We can describe our finite discrete space $X$ by a structure space $\hat{A}$
of a matrix algebra $A$. To establish a distance between two points in $X$ (as
we would in a metric space) we use an array $\{d_{ij}\}_{i, j \in X}$ of
\textit{real non-negative} entries in $X$ such that
\begin{itemize}
    \item $d_{ij} = d_{ji}$             Symmetric
    \item $d_{ij} \leq d_{ik} d_{kj}$       Triangle Inequality
    \item $d_{ij} = 0$ for $i=j$
\end{itemize}

In the commutative case, the algebra $A$ is commutative and can describe the
metric on $X$ in terms of algebraic data.
\begin{mytheorem}
    Let $d_{ij}$ be a metric on $X$ a finite discrete space with $N$ points, $A = \mathbb{C}^N$
    with elements $a = (a(i))_{i=1}^N$ such that $\hat{A} \simeq X$. Then there exists a
    representation $\pi$ of $A$ on a finite-dimensional inner product space $H$ and a symmetric
    operator $D$ on $H$ such that
    \begin{equation}
        d_{ij} = \sup_{a\in A}\bigg\{\big|a(i)-a(j)\big| : |\big|\big[D,
            \pi(a)]\big|\big| \leq 1\bigg\}
    \end{equation}
\end{mytheorem}

\begin{proof}
    We claim that this would follow from the equality:
    \begin{equation}
        \big|\big|[D, \pi(a)]\big|\big| = \max_{k\neq l}
        \bigg\{\frac{1}{d_{kl}}\big|a(k) - a(l)\big|\bigg\}
        \label{induction}
    \end{equation}
    This can be proven with induction. Let us set $N=2$,
    $H=\mathbb{C}^2$, $\pi:A\rightarrow L(H)$ and a hermitian matrix $D$.
    \begin{align}
        \pi(a) =
        \begin{pmatrix}
            a(1) & 0 \\
            0 & a(2)
        \end{pmatrix}
        \;\;\;\;
        D =
        \begin{pmatrix}
            0 & (d_{12})^{-1} \\
            (d_{21})^{-1} & 0
        \end{pmatrix}
    \end{align}
    Then we compute the commutator
    \begin{align}
        \big|\big|[D, \pi(a)]\big|\big| = (d_{12})^{-1} \big| a(1) - a(2)\big|
    \end{align}

    For the case $A=\mathbb{C}^3$, we have $H = (\mathbb{C}^2)^{\oplus 3} = H_2
    \oplus H_2^1 \oplus H_2^2$. The representation $\pi (a)$ reads
    \begin{align}
        \pi((a(1), a(2), a(3)) &=
        \begin{pmatrix}
            a(1) & 0 \\ 0 & a(2)
        \end{pmatrix} \oplus
        \begin{pmatrix}
            a(1) & 0 \\ 0 & a(3)
        \end{pmatrix} \oplus
        \begin{pmatrix}
            a(2) & 0 \\ 0 & a(2)
        \end{pmatrix} \nonumber  \\
                               & = \text{diag}\big(a(1), a(2), a(1), a(3), a(2),
                               a(3)\big)
    \end{align}
    And the operator $D$ takes the form
    \begin{align}
        D &=
        \begin{pmatrix}
            0 & x_1 \\ x_1 & 0
        \end{pmatrix} \oplus
        \begin{pmatrix}
            0 & x_2 \\ x_2 & 0
        \end{pmatrix} \oplus
        \begin{pmatrix}
            0 & x_3 \\ x_3 & 0
        \end{pmatrix} \nonumber \\
        &=
        \begin{pmatrix}
            0   & x_1 & 0 & 0 & 0 & 0 \\
            x_1 & 0   & 0 & 0 & 0 & 0 \\
            0   & 0   & 0 & x_2 & 0 & 0 \\
            0   & 0   & x_2 & 0 & 0 & 0 \\
            0   & 0   & 0 & 0 & 0 & x_3 \\
            0   & 0   & 0 & 0 & x_3 & 0 \\
        \end{pmatrix}.
    \end{align}
    Then the norm of the commutator would be the largest eigenvalue
    \begin{align}\label{eq:skew matrix}
        &\big|\big|[D, \pi(a)]\big|\big| = \big|\big|D\pi(a) - \pi(a)D\big|\big|,
    \end{align}
    where the matrix in the norm from equation \eqref{eq:skew matrix} is a
    skew symmetric matrix. Its eigenvalues are $i\lambda_1, i\lambda_2,
    i\lambda_3, i\lambda_4$. The $\lambda$'s are on the upper and lower
    diagonal. The matrix norm would be the maximum of the norm with the
    larges eigenvalues:
    \begin{align}
        \big|\big|[D, \pi(a)]\big|\big| = \max_{a\in A}\bigg\{&x_1\big|a(2)-a(1)\big|,\nonumber\\ &x_2\big|(a(3)-a(1))\big|,\nonumber\\
        &x_3\big|(a(3)-a(2))\big|\bigg\}.
    \end{align}
    Hence the metric turns out to be
    \begin{align}
        d =
        \begin{pmatrix}
            0 & a(1)-a(2) & a(1)-a(3)\\
            a(2)-a(1) & 0 & a(2)-a(3)\\
            a(3)-a(1) & a(3)-a(2) & 0
        \end{pmatrix}.
    \end{align}

    Suppose this holds for $N$ with $\pi_N$, $H_N = \mathbb{C}^N$ and $D_N$.
    Then it has to hold for $N+1$ with $H_{N+1} = H_{N} \oplus \bigoplus_{i=1}^N
    H_N^i$, since the representation reads
    \begin{align}
        \pi_{N+1}(a(1),\dots,a(N+1)) &= \pi_N(a(1),\dots,a(N))
        \oplus
        \begin{pmatrix}
            a(1) & 0 \\
            0   & a(N+1)
        \end{pmatrix} \oplus \nonumber\\
         &\oplus \cdots \oplus
        \begin{pmatrix}
            a(N) & 0 \\
            0  1 & a(N+1)
        \end{pmatrix}.
    \end{align}
    And the operator $D_{N+1}$ is
    \begin{align}
        D_{N+1} &= D_N
        \oplus
        \begin{pmatrix}
            0 & (d_{1(N+1)})^{-1} \\
            (d_{1(N+1)})^{-1}   & 0
        \end{pmatrix}\oplus \nonumber \\
                &\oplus \cdots \oplus
        \begin{pmatrix}
            0 & (d_{N(N+1)})^{-1} \\
            (d_{N(N+1)})^{-1}   & 0
        \end{pmatrix}.
    \end{align}
    From this follows equation \eqref{induction}.
    Hence we can continue the proof by setting for fixed $i, j$, $a(k) =
    d_{ik}$, which then gives $|a(i) - a(j)| = d_{ij}$ and thereby it follows
    that
    \begin{align}
        \frac{1}{d_{kl}} \big| a(k) - a(l) \big| =  \frac{1}{d_{kl}} \big|
        d_{ik} - d_{il} \big| \leq 1.
    \end{align}
\end{proof}

%---------------- EXERCISE
To get a better understanding of the results of the theorem let us compute a
metric on the space of three points given by $d_{ij} = \sup_{a\in A}\{|a(i) -
a(j)|: ||[D, \pi(a)]|| \leq 1\}$ for the set of data $A = \mathbb{C}^3$ acting
in the defining representation $H = \mathbb{C}^3$, and
\begin{align}
    D =
    \begin{pmatrix}
        0 & d^{-1} & 0 \\ d^{-1} & 0 & 0 \\ 0 & 0 & 0
    \end{pmatrix},
\end{align}
for some $d \in \mathbb{R}$.
From the data $A=\mathbb{C}^3$, $H=\mathbb{C}^3$ and $D$ we compute the
commutator
\begin{align}
    \big|\big|[D, \pi(a)]\big|\big| &= d^{-1}\left|\left|
\begin{pmatrix}
    0 & a(2)-a(1) & 0 \\
    -(a(2)-a(1)) & 0 & 0 \\
    0 & 0 & 0
\end{pmatrix} \right|\right|.
\end{align}
Hence the metric is
\begin{align}
d =
    \begin{pmatrix}
        0 & a(1)-a(2) & a(1)  \\
        a(2)-a(1) & 0 & a(2) \\
        -a(1) & -a(2) & 0
    \end{pmatrix}.
\end{align}
%---------------- EXERCISE

The translation of the metric on $X$ into algebraic data assumes commutativity
in $A$, this can be extended to a noncommutative matrix algebra, by the
following metric on a structure space $\hat{A}$ of a matrix algebra
$M_{n_i}(\mathbb{C}$
\begin{equation}
    d_{ij} = \sup_{a\in A}\big\{|\text{Tr}(a(i)) - \text{Tr}((a(j))|: ||[D,
    a]|| \leq 1\big\}.\label{eq:discretemetric}
\end{equation}
Equation \eqref{eq:discretemetric} is special case of the Connes' distance
formula on a structure space of $A$.

Finally we have all three ingredients to define a finite spectral triple, an
mathematical structure which encodes finite discrete geometry into algebraic data.
\begin{mydefinition}
    A \textit{finite spectral triple} is a tripe $(A, H, D)$, where $A$ is a unital $*$-algebra,
    faithfully represented on a finite-dimensional Hilbert space $H$, with a symmetric operator
    $D: H \rightarrow H$. (Note that $A$ is automatically a matrix algebra.)
\end{mydefinition}

\subsubsection{Properties of Matrix Algebras}
\begin{mylemma}
    If $A$ is a unital C* algebra acting faithfully on a finite
    dimensional Hilbert space, then $A$ is a matrix algebra of the Form:
    \begin{align}
        A \simeq \bigoplus _{i=1}^N M_{n_i}(\mathbb{C}).
    \end{align}
    The wording 'acting faithfully on a Hilbertspace' means that the
    $*$-representation is injective, or for a $*$-homomorphism that means
    one-to-one correspondence
\end{mylemma}
\begin{proof}
    Since $A$ acts faithfully on a Hilbert
    space, this means that $A$ is a $*$ subalgebra of a matrix algebra $L(H) = M_{\dim
    (H)}(\mathbb{C})$. Hence it follows, that $A$ is isomorphic to a matrix
    algebra.
\end{proof}

A simple illustration would be $A = M_n(\mathbb{C})$ for the algebra and
$H=\mathbb{C}^n$ for the Hilbertspace. Since $A$ acts on $H$ with matrix
multiplication and standard inner product and the operator $D$ on $H$ is a
hermitian $n\times n$ matrix.

\begin{mydefinition}
    Given an finite spectral triple $(A, H, D)$, the $A$-bimodule of
    Connes' differential one-forms is
    \begin{align}\label{eq:connesoneforms}
        \Omega _D ^1 (A) := \left\{ \sum _k a_k[D, b_k]: a_k, b_k \in A
        \right\}.
    \end{align}
\end{mydefinition}
Additionally there is a map $d:A\rightarrow \Omega _D ^1 (A)$, $d = [D,
\cdot]$, where $d$ is a derivation of the $*$-algebra in the sense that
\begin{align}
    d(a\ b) = d(a)b + ad(b), \\
    d(a^*) = -d(a)^*.
\end{align}
Since we have $d(\cdot) = [D, \cdot]$, we can easily check the above equations
\begin{align}
    d(a\ b) &= [D, a\ b] = [D, a]b + a[D,b]\nonumber\\
    &= d(a)\ b + a\ d(b).
\end{align}
And
\begin{align}
    d(a^*) &= [D, a^*] = Da^* - a^*D \nonumber\\
    &=-(D^*\ a - a\ D^*) = -[D^*, a] \nonumber\\
    &= -d(a)^*.
\end{align}
Furthermore $\Omega _D^1 (A)$ is an $A$-bimodule, which can be seen by
rewriting the defining equation \eqref{eq:connesoneforms} into
\begin{align}
    a\ (a_k[D, b_k])\ b &= a\ a_k(D\ b_k - b_k\ D)\ b = \nonumber\\
       &= a\ a_k(D\ b_k\ b - b_k\ D\ b)=\nonumber\\
       &= a\ a_k(D\ b_k\ b - b_k\ D\ b - b_k\
       b\ D +b_k\ b\ D)=
       \nonumber\\
       &= a\ a_k(D\ b_k\ b-b_k\ b\ D + b_k\ b\ D - b_k\ D\ b) = \nonumber \\
       &= a\ a_k [D, b_k\ b] + a\ a_k\ b [D, b]=\nonumber\\
       &= \sum _k\ a_k'\ [D, b_k']
\end{align}

\begin{mylemma}
    Let $\big(A, H, D\big) = \big(M_n(\mathbb{C}), \mathbb{C}^n, D\big)$, where
    $D$ is a hermitian $n\times n$ matrix. If $D$ is not a multiple of the
    identity then
    \begin{align}
        \Omega _D ^1 (A)  \simeq  M_n(\mathbb{C}) = A
    \end{align}
\end{mylemma}
\begin{proof}
    Assume $D = \sum _i \lambda _i e_{ii}$ is diagonal, $\lambda _i \in \mathbb{R}$ and
    $\{e_{ij}\}$ is the basis of $M_n(\mathbb{C})$. Then for fixed $i$, $j$ choose $k$
    such that $\lambda _k \neq \lambda _j$, hence we have
    \begin{align} \label{eq:basis}
        \left(\frac{1}{\lambda _k - \lambda _j} e_{ik}\right) [D, e_{kj}] =
        e_{ij},
    \end{align}
    for $e_{ij}\in \Omega _D ^1 (A)$ by the above definition
    \eqref{eq:connesoneforms}. Ultimately we have
    \begin{align}
        \Omega _D ^1
    (A) \subset L(\mathbb{C}^n) = H \simeq M_n(\mathbb{C}) = A
    \end{align}
\end{proof}

Consider an example
\begin{align}
     \left(A=\mathbb{C}^2, H=\mathbb{C}^2,
         D = \begin{pmatrix} 0 & \lambda \\ \bar{\lambda} & 0
 \end{pmatrix}\right)
\end{align}
with $\lambda \neq 0$. We can show that $\Omega _D^1(A)
\simeq M_2(\mathbb{C})$. The Hilbert Basis $D$ can be extended in terms of
the basis of $M_2(\mathbb{C})$, plugging this into equation
\eqref{eq:basis} will get us the same cyclic result and thus
$\Omega _D^1(A) \simeq M_2(\mathbb{C})$.

\subsubsection{Morphisms Between Finite Spectral Triples}
Next we will define an equivalence relation between finite spectral triples, called
spectral unitary equivalence. This equivalence relation is given by the unitarity of the
two matrix algebras themselves, and an additional map $U$ which allows us to associate
one operator to a second operator.
\begin{mydefinition}
    Two finite spectral tripes $(A_1, H_1, D_1)$ and $(A_2, H_2, D_2)$ are
    called unitary equivalent if $A_1 = A_2$ and there exists a map $U:\ H_1
    \rightarrow H_2$ that satisfies
    \begin{align}
        U\ \pi_1(a)\ U^* &= \pi_2(a)\;\;\;\; \text{with} \;\;\; a \in A_1,\\
        U\ D_1\ U^* &= D_2.
    \end{align}
\end{mydefinition}
Notice that for any such $U$ we have the relation $(A, H, D) \sim (A, H,\ UDU^*)$.
And hence
\begin{align}
    U\ D\ U^* = D + U[D,\ U^*],
\end{align}
are of the form of elements in $\Omega _D^1 (A)$.

%-------------- EXERCISE
To make it clear that the above definition is an equivalence relation between
finite spectral triples, we need to see if the relation satisfies
reflexivity, symmetry and transitivity. Let us look then at three spectral
triples $(A_1, H_1, D_1)$, $(A_2, H_2, D_2)$ and $(A_3, H_3, D_3)$.
For reflexivity $(A_1, H_1, D_1) \sim (A_1, H_1, D_1)$. So there
exists the unitary map $U: H_1 \rightarrow H_1$, which is the identity
and always exists. On the other hand the symmetry condition requires
\begin{align}
    (A_1, H_1, D_1) \sim (A_2, H_2, D_2) \Leftrightarrow
    (A_2, H_2, D_2) \sim (A_1, H_1, D_1).
\end{align}
Because $U$ is unitary we can rewrite for the representation for $A_1$
\begin{align}
    &U\pi_1(a)U^* = \pi_2(a) \;\;\; | \cdot U^*\boxdot U\nonumber \\
    &U^*U\pi_1(a)U^*U = \pi_1(a) = U^*\pi_2(a)U.
\end{align}
The same relation applies for the symmetric operator $D$.
Lastly for transitivity the condition is
\begin{align}
    (A_1, H_1, D_1) &\sim (A_2, H_2, D_2) \;\;\; \text{and} \;\;\;
    (A_2, H_2, D_2) \sim (A_3, H_3, D_3) \nonumber\\
    &\Rightarrow (A_1, H_1, D_1) \sim (A_3, H_3, D_3).
\end{align}
Therefore the two unitary maps $U_{12}:H_1 \rightarrow H_2$ and
$U_{23}: H_2 \rightarrow H_3$ are
\begin{align}
    U_{23}\ U_{12}\ \pi_1(a)\ U^*_{12}\ U^*_{23} &= U_{23}\
    \pi_2(a)\ U_{23}^*\nonumber \\
    &= \pi_3(a), \\
    U_{23}\ U_{12}\ D_1\ U^*_{12}\ U^*_{23} &= U_{23}\
    D_2 U_{23}^* \nonumber\\
    &= D_3.
\end{align}
%-------------- EXERCISE

In order to extend the equivalence relation we take a look at Morita
equivalence of Matrix Algebras.
\begin{mydefinition}
    Let $A$ be an algebra. We say that $I \subset A$, as a vector space, is a
    right(left) ideal if $a\ b \in I$ for $a \in A$ and $b\in I$ (or $b\ a \in
    I$, $b\in I$, $a\in A$). We call a left-right ideal simply an ideal.
\end{mydefinition}

Given a Hilbert bimodule $E \in KK_f(B, A)$ and $(A, H, D)$ we construct
a finite spectral triple on $B$, $(B, H', D')$
\begin{equation}
    H' = E \otimes _A H.
\end{equation}
We might define $D'$ with
\begin{align}
    D'(e \otimes \xi) = e\otimes D\xi
\end{align}
Although this would not satisfy the ideal defining the balanced tensor
product over $A$, which is generated by elements of the form
\begin{align}
    e\ a \otimes \xi - e\otimes a\ \xi, \;\;\;\;\; e\in E, a\in A, \xi \in
    H.
\end{align}
This inherits the left action on $B$ from $E$ and has a $\mathbb{C}$
valued inner product space. $B$ also satisfies the ideal
\begin{equation}
    D'(e\otimes \xi) = e \otimes D\ \xi + \nabla (e)\ \xi, \;\;\;\; e\in
    E, a\in A,
\end{equation}
where $\nabla$ is called the \textit{connection on the right A-module E}
associated with the  derivation $d=[D, \cdot]$. The connection needs to
satisfy the \textit{Leibniz Rule}
\begin{equation}
    \nabla(ae) = \nabla(e)a + e \otimes [D, a], \;\;\;\;\;  e\in E,\; a\in A.
\end{equation}
Hence $D'$ is well defined on $E \otimes _A H$
\begin{align}
    D'(e\ a \otimes \xi - e \otimes a\ \xi) &=  D'(e\ a \otimes \xi) - D'(e
    \otimes \xi) \nonumber\\
    &= e\ a\otimes D\ \xi + \nabla(a\ e)\ \xi - e \otimes D(a\ \xi) - \nabla
    (e)\ a\ \xi \nonumber\\
    &= 0.
\end{align}
With the information thus far we can prove the following theorem
\begin{mytheorem}
    If $(A, H, D)$ is a finite spectral triple and $E \in KK_f(B, A)$,
    then $(V, E\otimes _A H, D')$ is a finite spectral triple, provided that
    $\nabla$ satisfies the compatibility condition
    \begin{equation}
        \langle e_1, \nabla e_2 \rangle _E - \langle \nabla e_1, e_2
        \rangle _E = d\langle e_1, e_2 \rangle _E \;\;\;\; e_1, e_2 \in E
    \end{equation}
\end{mytheorem}
\begin{proof}
    The computation for $E\otimes _A H$ is above . The only thing left is to
    show is, that $D'$ is a symmetric operator. We can prove this by
    computing for $e_1, e_2 \in E$ and $\xi _1, \xi _2 \in H$ then
    \begin{align}
        \langle e_1 \otimes \xi _1, D'(e_2 \otimes \xi_2)\rangle _{E\otimes _A H} &=
        \langle \xi _1, \langle e_1, \nabla e_2\rangle _E\ \xi _2\rangle
        \langle \xi _1 , \langle e_1, e_2\rangle _E\ D\ \xi_2\rangle_H  \nonumber\\
        &= \langle \xi _1, \langle \nabla e_1, e_2\rangle _E\ \xi _2\rangle _H + \langle \xi _1, d\langle e_1, e_2\rangle  _E
        \ \xi _2\rangle _H \nonumber\\
        &+ \langle D\ \xi _1,\langle e_1, e_2\rangle _E\ \xi _2\rangle _H -
        \langle \xi _1, [D, \langle e_1, e_2\rangle _E]\ \xi
        _2 \rangle _H \nonumber\\
        &= \langle D'(e_1 \otimes \xi _1), e_2 \otimes \xi _2\rangle _{E \otimes _A H}
    \end{align}
\end{proof}

Let us examine the scenario where we consider the difference of connections $\nabla$ and
$\nabla'$ on a right $A$-module $E$. Since both connections need to satisfy
the Leibniz rule, the difference also should
    \begin{align}
        \nabla(ea)-\nabla'(ea)&=\nabla(e) + e\otimes[D, a]\nonumber\\
        &-(\nabla'(e)a + e\otimes[D',a])\nonumber\\
        &=\bar{\nabla}a + e\otimes(Da-aD-D'a+aD')\nonumber\\
        &=\bar{\nabla}a + e\otimes((D-D')a-a(D-D'))\nonumber\\
        &=\bar{\nabla}a + e\otimes[D', a]\nonumber\\
        &=\bar{\nabla}(ea).
    \end{align}
Therefore $\nabla-\nabla'$ is a right $A$-linear map
$E \rightarrow E\otimes _A \Omega _D^1(A)$.

To get a better grasp of the results let us construct a finite spectral
triple $(A, H', D')$ from $(A, H, D)$. The derivation $d(\cdot):A \rightarrow
A\otimes _A \Omega_D^1(A)=\Omega_D^1(A)$ is a connection on $A$, i.e.\ considered a
right $A$-module
\begin{align}
        \nabla(e \cdot a) =  d(a),
\end{align}
hence $A\otimes_A H\simeq H$. Next we can construct the operator $D'$
for the connection $d(\cdot)$
\begin{align}
    D'(a\xi) = a(D\xi) + (\nabla a) \xi = D(a\xi).
\end{align}
By using the identity element in the connection relation we conclude
\begin{align}
   \nabla (e\cdot a) = \nabla(e) a + 1 \otimes d(a)=d(a) \nabla(e) a,
\end{align}
thus any connection $\nabla: A\rightarrow A\otimes_A \Omega_D^1(A)$ is
given by
\begin{align}\label{eq: uniqueconnection}
    \nabla = d + \omega,
\end{align}
where $\omega \in \Omega_D^1(A)$. This becomes clear when looking at the
difference operator $D'$ with the connection on $A$, which is given by
\begin{align}
    D'(a\otimes \xi) &= D'(a \xi) = a(D\xi) + (\nabla a)\xi \nonumber \\
                     &=a(D\xi) + \nabla(e \cdot a) \xi \nonumber\\
                     &= D(a\xi) + \nabla(e) (a\xi),
\end{align}
hence any such connection is of the form as in equation \eqref{eq: unique
connection}

%\subsubsection{Graphing Finite Spectral Triples}
%\begin{mydefinition}
%    A \textit{graph} is a ordered pair $(\Gamma ^{(0)}, \Gamma ^{(1)})$.
%    Where $\Gamma ^{(0)}$ is the set of vertices (nodes) and $\Gamma ^{(1)}$
%    a set of pairs of vertices (edges)
%\end{mydefinition}
%\begin{figure}[h!]
%    \centering
%\begin{tikzpicture}[
%        mass/.style = {draw,circle, minimum size=0.2cm, inner sep=0pt, thick},
%        spring/.style = {decorate,decoration={zigzag, pre length=1cm,post length=1cm,segment length=5pt}},]
%        \node[mass] (m1) at (1,1.5) {};
%        \node[mass] (m2) at (-1,1.5) {};
%        \node[mass] (m3) at (0,0) {};
%
%        \draw (m1) -- (m2);
%        \draw (m1) -- (m3);
%        \draw (m2) -- (m3);
%    \end{tikzpicture}
%    \caption{A simple graph with three vertices and three edges}
%\end{figure}
%%\begin{MyExercise}
%%    \textbf{
%%    Show that any finite-dimensional faithful representation $H$ of a matrix
%%    algebra $A$ is completely reducible. To do that show that the complement
%%    $W^{\perp}$ of an $A$-submodule $W\subset H$ is also an $A$-submodule
%%    of $H$.
%%}\newline
%%
%%    $A\simeq \bigoplus_{i=1}^N M_{n_i}(\mathbb{C})$ is the matrix algebra
%%    then $H$ is a Hilbert $A$-bimodule and $W$ a submodule of $A$.
%%    Because we have $H = W \cup W^{\perp}$, then $W^{\perp}$ is naturally a
%%    $A$-submodule, because elements in $W^{\perp}$ need to satisfy the
%%    bimodularity.
%%\end{MyExercise}
%\begin{mydefinition}
%    A $\Lambda$-decorated graph is given by an ordered pair $(\Gamma,
%    \Lambda)$ of a finite graph $\Gamma$ and a set of positive integers
%    $\Lambda$ with the labeling
%    \begin{itemize}
%        \item of the vetices $v\in \Gamma ^{(0)}$ given by $n(\nu) \in
%            \Lambda$
%        \item of the edges $e = (\nu _1, \nu _2) \in \Gamma ^{(1)}$ by
%            operators
%            \begin{itemize}
%                \item $D_e: \mathbb{C}^{n(\nu _1)} \rightarrow
%                    \mathbb{C}^{n(\nu _2)}$
%                \item and $D_e^*: \mathbb{C}^{n(\nu _2)} \rightarrow
%                    \mathbb{C}^{n(\nu _1)}$ its conjugate traspose
%                    (pullback?)
%            \end{itemize}
%    \end{itemize}
%    such that
%    \begin{equation}
%        n(\Gamma ^{(0)}) = \Lambda
%    \end{equation}
%\end{mydefinition}
%\begin{question}
%    Would then $D_e$ be the pullback?
%\end{question}
%\begin{question}
%    These graphs are important in the next chapter I should look
%    into it more, I don't understand much here, specific
%    how to construct them with the abstraction of a spectral triple...
%\end{question}
%
%The operator $D_e$ between $\textbf{n}_i$ and $\textbf{n}_j$ add up to
%$D_{ij}$
%\begin{align}
%    D_{ij} = \sum\limits_{\substack{e = (\nu _1, \nu _2) \\ n(\nu _1) =
%    \textbf{n}_i \\ n(\nu _2) = \textbf{n}_j}} D_e
%\end{align}
%
%\begin{mytheorem}
%    There is a on to one correspondence between finite spectral triples
%    modulo unitary equivalence and $\Lambda$-decorated graphs, given by
%    associating a finite spectral triples $(A, H, D)$ to  a $\Lambda$ decorated
%    graph $(\Gamma, \Lambda)$ in the following way:
%    \begin{equation}
%        A = \bigoplus _{n\in \Lambda} M_n(\mathbb{C}); \;\;\;
%        H = \bigoplus _{\nu \in \Gamma ^{(0)}} \mathbb{C}^{n(\nu)}; \;\;\;
%        D = \sum _{e \in \Gamma ^{(1)}} D_e + D_e^*
%    \end{equation}
%\end{mytheorem}
%    \begin{figure}[h!]
%    \centering
%    \begin{tikzpicture}[
%        mass/.style = {draw,circle, minimum size=0.3cm, inner sep=0pt, thick},
%    ]
%
%    \node[mass, label={\textbf{n}}] (m1) at (1,0) {};
%    \draw (m1) to [out=330, in=210, looseness=25] node[above] {$D_e$} (m1);
%    \end{tikzpicture}
%    \caption{A $\Lambda$-decorated Graph of $(M_n(\mathbb{C}), \mathbb{C}^n,
%    D = D_e + D_e^*)$}
%\end{figure}
%
%%\begin{MyExercise}
%%    \textbf{
%%    Draw a $\Lambda$ decorated graph corresponding to the spectral triple
%%    $(A=\mathbb{C}^3, H=\mathbb{C}^3, D=\begin{pmatrix}0 & \lambda & 0\\
%%    \bar{\lambda} &0 &0 \\ 0&0&0\end{pmatrix})$
%%}\newline
%%
%%\centering
%%\begin{tikzpicture}[
%%        mass/.style = {draw,circle, minimum size=0.4cm, inner sep=0pt, thick},
%%        spring/.style = {decorate,decoration={zigzag, pre length=1cm,post length=1cm,segment length=5pt}},]
%%        \node[mass] (m1) at (-1,1.5) {\textbf{1}};
%%        \node[mass] (m2) at (1,1.5) {\textbf{2}};
%%        \node[mass] (m3) at (3,1.5) {\textbf{3}};
%%
%%        \draw[style=thick, -] (1.1,1.7) -- (-1.1,1.7);
%%        \draw[style=thick, -] (1.1,1.3) -- (-1.1,1.3);
%%    \end{tikzpicture}
%%    %    \captionof{figure}{Solution}
%%\end{MyExercise}
%%\begin{MyExercise}
%%    \textbf{
%%    Use $\Lambda$-decorated graphs to classify all finite spectral triples
%%    (modulo unitary equivalence) on the matrix algebra
%%    $A=\mathbb{C}\oplus M_2(\mathbb{C})$
%%}\newline
%%
%%    \centering
%%\begin{tikzpicture}[
%%        mass/.style = {draw,circle, minimum size=0.4cm, inner sep=0pt, thick},
%%        spring/.style = {decorate,decoration={zigzag, pre length=1cm,post length=1cm,segment length=5pt}},]
%%        \node[mass] (m1) at (-1,1) {\textbf{1}};
%%        \node[mass] (m2) at (1,1) {\textbf{2}};
%%        \node[mass] (m3) at (3,1) {\textbf{3}};
%%
%%        \node[mass] (m4) at (-1,0) {\textbf{1}};
%%        \node[mass] (m5) at (1,0) {\textbf{2}};
%%        \node[mass] (m6) at (3,0) {\textbf{3}};
%%
%%        \node[mass] (m7) at (-1,-1) {\textbf{1}};
%%        \node[mass] (m8) at (1,-1) {\textbf{2}};
%%        \node[mass] (m9) at (3,-1) {\textbf{3}};
%%
%%        \node[mass] (m10) at (-1,-2) {\textbf{1}};
%%        \node[mass] (m11) at (1,-2) {\textbf{2}};
%%        \node[mass] (m12) at (3,-2) {\textbf{3}};
%%
%%        \draw[style=thick, -] (1.1,0.2) -- (-1.1,0.2);
%%        \draw[style=thick, -] (1.1,-0.2) -- (-1.1,-0.2);
%%        \draw[style=thick, -] (m7) to [out=330, in=210, looseness=10] node[above] {} (m7);
%%        \draw[style=thick, -] (m10) -- (m11) ;
%%
%%\end{tikzpicture}
%%%    \captionof{figure}{Solution $A=M_3(\mathbb{C})$}
%%\end{MyExercise}
%\subsubsection{Graph Construction of Finite Spectral Triples}
%\textbf{Algebra:}We know if a acts on a finite dimensional Hilbert space then
%this C* algebra is isomorphic to a matrix algebra so $A \simeq
%\bigoplus_{i=1}^{N}M_{n_i}(\mathbb{C})$. Where $i\in
%\hat{A}$ represents an equivalence class and runs from $1$ to $N$,
%thus $\hat{A}\simeq\{1,\dots, N\}$. We label equivalence classes by
%$\textbf{n}_i$, then $\hat{A}\simeq\{\textbf{n}_1,\dots,\textbf{n}_N\}$.
%\newline
%
%\textbf{Hilbert Space:} Since every Hilbert space that acts faithfully on a
%C* algebra is completely reducible, it is isomorphic to the composition
%of irreducible representations. $H \simeq \bigoplus_{i=1}^N\mathbb{C}^{n_i}
%\otimes V_i$. Where all $V_i$'s are Vector spaces, their dimension is the
%multiplicity of the representation landed by $\textbf{n}_i$ to $V_i$ itself
%by the multiplicity space.
%\newline
%
%\textbf{Finite Dirac Operator:} $D_{ij}$ is connecting nodes $\textbf{n}_i$
%and $\textbf{n}_j$, with a symmetric map $D_{ij}:\mathbb{C}^{n_i}\otimes V_i
%\rightarrow \mathbb{C}^{n_j}\otimes V_j$
%\newline
%
%To draw a graph, draw nodes in position $\textbf{n}_i\in \hat{A}$.
%Multiple nodes at the same position represent multiplicities in $H$.
%Draw lines between nodes to represent $D_{ij}$.
%
%\begin{figure}[h!]
%    \centering
%\begin{tikzpicture}
%    \node[draw, label=above:{$\textbf{n}_1$},circle, thick] at (-3,0) {};
%    \node[label=above:{$\dots$}] at (-2,0) {};
%    \node[draw, label=above:{$\textbf{n}_i$},circle, thick] at (-1,0) {};
%    \node[label=above:{$\dots$}] at (0,0) {};
%    \node[draw, label=above:{$\textbf{n}_j$},circle, thick] at (1,0) {};
%    \node[draw, label=above:{},circle, thick, inner sep=0cm, minimum
%    size=0.2cm]  at (1,0) {};
%    \node[label=above:{$\dots$}] at (2,0) {};
%    \node[draw, label=above:{$\textbf{n}_N$},circle, thick] at (3,0) {};
%
%        \draw[style=thick, -] (-1,-0.2) -- (1,-0.2);
%        \draw[style=thick, -] (-1,0.2) -- (1,0.2);
%        \path[style=thick, -] (-1,-0.2) edge[bend right=15]
%        node[pos=0.5,below] {} (3,-0.2);
%    \end{tikzpicture}
%    \caption{Example}
%\end{figure}
