
\subsection{Noncommutative Geometric Spaces}
\subsubsection{Matrix Algebras and Finite Spaces}
\subsubsection{$*$-Algebra}
\begin{definition}
    A \textit{vector space} $A$ over $\mathbb{C}$ is called a \textit{complex, unital Algebra} if, \\
    $\forall a,b \in A$ :
    \begin{enumerate}
        \item
            $A \times A \rightarrow A$       \hspace{0.1\textwidth} \textit{bilinear} \\
            $(a, b)\ \mapsto \ a\cdot b$
        \item
            $1a = a1 =a$                     \hspace{0.08\textwidth}  \textit{unital} \\
    \end{enumerate}
\end{definition}

\begin{definition}
    A $*$-algebra is an algebra $A$ with a \textit{conjugate linear map (involution)} $*:A\ \rightarrow  A$,
    $\forall a, b \in A$ satisfying:
    \begin{enumerate}
        \item
            $(ab)^* = b^*a^*$         \hspace{0.05\textwidth} \textit{antidistributive}
        \item
            $(a^*)^* = a$                   \hspace{0.1\textwidth} \textit{closure}
    \end{enumerate}
\end{definition}
In the following all unital algebras are referred to as algebras.

\subsubsection{Functions on Discrete Spaces}
Let $X$ be a \textit{discretized topological} space with $N$ points.
Consider functions of a continuous $*$-algebra $C(X)$ assigning values to $\mathbb{C}$, for $f, g \in C(X)$,
$\lambda \in \mathbb{C}$ and $x \in X$ they provide the following structures:

\begin{itemize}
    \item \textit{pointwise linear} \\
      $(f + g)(x) = f(x) + g(x)$\\
      $(\lambda f)(x) = \lambda (f(x))$
    \item \textit{pointwise multiplication} \\
        $fg(x) = f(x)g(x)$ \hspace{0.1\textwidth} same as $(fg)(x) = f(x)g(x))$?
    \item \textit{pointwise involution} \\
        $f^*(x) = \overline{f(x)}$
\end{itemize}

\begin{question}
    Mathematical difference between Topological Discreet Spaces and just Discreet Spaces?
\end{question}

The author indicates that $\mathbb{C}$-valued functions on $X$ are automatically continuous.
\begin{idea}
    CAN WE USE THE METRIC? NO!
    We know that $X$ is a \textit{finite discrete space}, meaning in an $\epsilon$-$\delta$ approach
    for each $x \in X$ the only $y \in X$, that is small enough is $x$ by itself, which implies
    $\epsilon$ is always bigger than zero, thus every function $f:X\ \rightarrow\ \mathbb{C}$ is continuous.
\end{idea}

\subsubsection{Isomorphism Property}
Furthermore $C(X)$ $*$-algebra is \textit{isomorphic} to a $*$-algebra $\mathbb{C}^N$ with involution
($N$ number of points in $X$), written as $C(X) \simeq \mathbb{C}^N$.
A function $f:X\ \rightarrow\ \mathbb{C}$ can be represented with $N \times N$ diagonal matrices,
where the value $(ii)$ is the value of the function at the corresponding
$i$-th point ($i = 1,...,N$). The structure is preserved because of the definitions of
matrix multiplication and the hermitian conjugate of matrices.

\begin{question}
    Can isomorphisms between $C(X)$ and $\mathbb{C}^N$ be shown with matrix factorization?
\end{question}
    Isomorphisms are bijective preserve structure and don't lose physical information/

\subsubsection{Mapping Finite Discrete Spaces}

\begin{definition}
    A \textit{map} between finite discrete spaces $X_1$ and $X_2$ is a function $\phi:\ X_1 \rightarrow\ X_2$
\end{definition}

For every map between finite discrete spaces there exists a corresponding map \\
$\phi ^*:C(X_2)\ \rightarrow C(X_1)$, which `pulls back' values even if $\phi$ is not bijective.
Note that the pullback doesn't map points back, but maps functions on an $*$-algebra $C(X)$.


This map is called a pullback (or a $*$-homomorphism or a $*$-algebra map under pointwise product).
Under the pointwise product:
\begin{itemize}
    \item $\phi ^*(fg) = \phi ^*(f) \phi ^*(g)$
    \item $\phi ^*(\overline{f}) = \overline{\phi ^*(f)}$
    \item $\phi ^*(\lambda f + g) = \lambda \phi ^*(f) + \phi ^*(g)$
\end{itemize}

\begin{question}
    $\phi$ is in most cases not bijective, so how can we prove that there exists such a
    pullback for every map between discrete spaces which preserves information? For bijective
    it is given by its inverse, which by definition exists because $\phi$ is a map.
    Or I didn't understand this correctly?
\end{question}

\begin{MyExercise}
    \textbf{
    Show that $\phi :X_1\ \rightarrow \ X_2$ is injective (surjective) map of finite spaces iff
    $\phi ^* :C(X_2)\ \rightarrow \ C(X_1)$ is surjective (injective).
}\newline

    Consider $X_1$ with $n$ points and $X_2$ with $m$ points. Then there are three cases:
    \begin{enumerate}
        \item $n=m$ \\
            Obviously $\phi$ is bijective and $\phi ^*$ too.
        \item $n \rangle  m$ \\
            $\phi$ assigns $n$ points to $m$ points when $n \rangle  m$,
            which is by definition surjective. \\
            $\phi ^*$ assigns $m$ points to $n$ points when $n \rangle  m$,
            which is by definition injective. \\
        \item $n \langle  m $ \\
            analogous
    \end{enumerate}
\end{MyExercise}

\subsubsection{Matrix Algebras}
\begin{definition}
    A \textit{(complex) matrix algebra} A is a direct sum, for $n_i, N \in \mathbb{N}$.
    \begin{align}
        A = \bigoplus _{i=1}^{N} M_{n_i}(\mathbb{C})
    \end{align}
    The involution is the hermitian conjugate, a $*$ algebra with involution is referred to as
    a matrix algebra
\end{definition}

So from a topological discrete space $X$, we can construct a $*$-algebra $C(X)$ which is isomorphic
to a matrix algebra $A$. The question is can we construct $X$ given $A$? $A$ is a matrix algebra,
which are in most cases is not commutative, so the answer is generally no.

There are two options. We can restrict ourselves to commutative matrix algebras,
which are the vast minority and not physically interesting.
Or we can allow more morphisms(isomorphisms) between matrix algebras.

\begin{question}
    Why are non-commutative algebras not physically interesting?
    Maybe too far fetched,but because physical observables (QM-Operators) are not commutative?
\end{question}
Exactly.

\subsubsection{Finite Inner Product Spaces and Representations}
Until now we looked at a finite topological discrete space, moreover we can consider a
finite dimensional inner product space $H$ (finite Hilbert-spaces), with inner product
$(\cdot,\cdot)\rightarrow \mathbb{C}$. $L(H)$ is the $*$-algebra of operators on $H$
with product given by composition and involution given by the adjoint, $T \mapsto T^*$.
$L(H)$ is a \textit{normed vector space} with
\begin{align}
    &\|T\|^2 = \text{sup}_{h \in H}\{(Th,Th): (h,h) \leq 1\} \hspace{0.1\textwidth} T \in L(H) \\
    &\|T\| = \text{sup}\{\sqrt{\lambda}: \lambda \text{ eigenvalue of } T\}
\end{align}


\begin{definition}
    The \textit{representation} of a finite dimensional $*$-algebra A is a pair $(H, \pi)$.
    $H$ is a finite, dimensional inner product space and $\pi$ is a $*$-\textit{algebra map}
    \begin{align}
        \pi:A\ \rightarrow \ L(H)
    \end{align}
\end{definition}
\begin{definition}
    $(H, \pi)$ is called \textit{irreducible} if:
    \begin{itemize}
        \item $H \neq \emptyset$
        \item only $\emptyset$ or $H$ is invariant under the action of $A$ on $H$
    \end{itemize}
\end{definition}

Examples for reducible and irreducible representations
\begin{itemize}
    \item $A = M_n(\mathbb{C})$, representation $H=\mathbb{C}^n$, $A$ acts as matrix multiplication\\
            $H$ is irreducible.
    \item $A = M_n(\mathbb{C})$, representation $H=\mathbb{C}^n\oplus \mathbb{C}^n$, with $a \in A$ acting
        in block form \\ $\pi: a \mapsto \big(\begin{smallmatrix} a & 0\\ 0 & a \end{smallmatrix}\big)$ is
            reducible.
\end{itemize}

\begin{definition}
    Let $(H_1, \pi _1)$ and $(H_2, \pi _2)$ be representations of a $*$-algebra $A$. They are called
    \textit{unitary equivalent} if there exists a map $U: H_1 \rightarrow H_2$ such that.
    \begin{align}
        \pi _1(a) = U^* \pi _2(a) U
    \end{align}
\end{definition}

\begin{question}
    In matrix representation this is diagonalization condition? (unitary diagonalization)
\end{question}
Yes

\begin{definition}
    $A$ a $*$-algebra then, $\hat{A}$ is called the structure space of all \textit{unitary equivalence classes
    of irreducible representations of A}
\end{definition}

\begin{question}
    Gelfand duality and the spectrum of $\hat{A}$, examples Fourier-Transform and Laplace-Transform
    for simple spaces.
\end{question}
More on that in later chapters.

\begin{MyExercise}
    \textbf{
    Given $(H, \pi)$ of a $*$-algebra $A$, the \textbf{commutant} $\pi (A)'$ of $\pi (A)$ is defined as a set
    of operators in $L(H)$ that commute with all $\pi (a)$
    \begin{align}
        \pi (A)' = \{T \in L(H):\pi (a)T = T\pi (a) \;\;\; \forall a\in A\}
    \end{align}
    \begin{enumerate}
        \item Show that $\pi (A)'$ is a $*$-algebra.
        \item Show that a representation $(H, \pi)$ of $A$ is irreducible iff the commutant $\pi (A)'$
                consists of multiples of the identity
    \end{enumerate}
}

    1. To show that $\pi (A)'$ is a $*$-algebra we have to show that it is unital, associative and involute.
    And note that $\pi (a) \in L(H)\ \forall a \in A$.
    Unitarity is given by the unital operator of the $*$-algebra of operators $L(H)$, which exists by definition
    because H is a inner product space. Associativity is given by $*$-algebra of $L(H)$, $L(H) \times L(H) \mapsto L(H)$,
    which is associative by definition. Involutnes is also given by the $*$-algebra $L(H)$
    with a map $*: L(H) \mapsto L(H)$ only for $T$ that commute with $\pi (a)$.
    \\
    2.?
\end{MyExercise}

\begin{MyExercise}
    \textbf{
    \begin{enumerate}
        \item If $A$ is a unital $*$-algebra, show that the $n \times n$ matrices $M_n(A)$ with entries
            in $A$ form a unital $*$-algebra.
        \item Let $\pi :A\ \rightarrow \ L(H)$ be a representation of a $*$-algebra $A$ and set
            $H^n = H \oplus ... \oplus H$, $n$ times. Show that $\tilde{\pi}:M_n(A) \rightarrow L(H^n)$
            of $M_n(A)$ with\\
            $\tilde{\pi}((a_{ij})) = (\tilde{\pi}(a_{ij})) \in M_n(A)$.
        \item Let $\tilde{\pi}:M_n(A) \rightarrow L(H^n)$ be a $*$ algebra representation of $M_n(A)$.
            Show that $\pi: A \rightarrow L(H^n)$ is a representation of $A$.
    \end{enumerate}
}
    1. We know $A$ is a $*$ algebra. Unitary operaton in $M_n(A)$ is given by the identity Matrix, which
    has to exists because every entry in $M_n(A)$ has to behave like in $A$. Associativity is given by
    matrix multiplication. Involutnes is given by the conjugate transpose.\\
    2. $A \simeq M_n(A)$ and $H \simeq H^n$ meaning $\tilde{\pi}$ is a valid reducible representation.\\
    3. $\tilde{\pi}$ and $\pi$ are unitary equivalent, there is a map $U: H^n \rightarrow H^n$ given by
    $U=\mathbbm{1}_n$:\\
    $\pi (a) = \mathbbm{1}_n^*\ \tilde{\pi}((a_{ij}))\ \mathbbm{1}_n = \tilde{\pi}((a_{ij})) = \pi (a_{ij})
    \Rightarrow a_{ij} = a\mathbbm{1}_n$.
\end{MyExercise}

\subsubsection{Commutative Matrix Algebras}
\begin{itemize}
    \item Commutative matrix algebras can be used to reconstruct a discrete space given
        a matrix \textit{commutative} matrix algebra.
    \item The structure space $\hat{A}$ is used for this. Because $A \simeq \mathbb{C}^N$ we all any
        irreducible representation are of the form
        $\pi _i:(\lambda_1,...,\lambda_N)\in \mathbb{C}^N \mapsto \lambda_i \in \mathbb{C}$ \\
        for $i = 1,...,N \Rightarrow \hat{A} \simeq \{1,...,N\}.$
    \item Conclusion is that there is a duality between discrete spaces and commutative matrix algebra
        this duality is called the \textit{finite dimensional Gelfand duality}
\end{itemize}

\subsubsection{Noncommutative Matrix Algebras}
Aim is to construct duality between finite dimensional spaces and \textit{equivalence classes}
of matrix algebras, to preserve general non-commutivity of matrices.
\begin{itemize}
    \item Equivalence classes are described by a generalized notion of ispomorphisms between matrix
        algebras (\textit{Morita Equivalence})
\end{itemize}

\subsubsection{Algebraic Modules}
\begin{definition}
    Let $A$, $B$ be algebras (need not be matrix algebras)
    \begin{enumerate}
        \item \textit{left} A-module is a vector space $E$, that carries a left representation of $A$,
            that is $\exists$ a bilinear map $\gamma: A \times E \rightarrow E$ with
            \begin{align}
                (a_1a_2)\cdot e = a_1 \cdot (a_2 \cdot e);\;\;\; a_1, a_2 \in A, e \in E
            \end{align}
        \item \textit{right} B-module is a vector space $F$, that carries a right representation of $A$,
            that is $\exists$ a bilinear map $\gamma: F \times B \rightarrow F$ with
            \begin{align}
                f \cdot (b_1b_2)= (f \cdot b_1) \cdot b_2;\;\;\; b_1, b_2 \in B, f \in F
            \end{align}
        \item \textit{left} A-module and \textit{right} B-module is a \textit{bimodule}, a vector space $E$
            satisfying
            \begin{align}
                a \cdot (e \cdot b)= (a \cdot e) \cdot b;\;\;\;  a \in A, b \in B, e \in E
            \end{align}
    \end{enumerate}
\end{definition}
Notion of A-\textbf{module homomorphism} as linear map $\phi: E\rightarrow F$ which respects the
representation of A, e.g. for left module.
\begin{align}
    \phi (ae) = a \phi (e); \;\;\; a \in A, e \in E.
\end{align}
Remark on the notation
\begin{itemize}
    \item ${}_A E$ left $A$-module $E$;
    \item ${}_A E_B$ right $B$-module $F$;
    \item ${}_A E_B$ $A$-$B$-bimodule $E$;
\end{itemize}

\begin{MyExercise}
    \textbf{
    Check that a representation of $\pi : A \rightarrow L(H)$ of a $*$-algebra A turns H into a
    left module ${}_A H$.
}\newline

    Not quite sure but \\
    $a \in A$, $h_1, h_2 \in H$, we know $\pi (a) = T \in L(H)$ than
    \begin{align}
        \langle \pi (a) h_1, \pi (a) h_2\rangle  = \langle T h_1, T h_2\rangle  = \langle T^*T h_1, h_2\rangle  = \langle h_1, h_2\rangle
    \end{align}
    Or maybe this \\
    If $_A H$ than $(a_1a_2) h = a_1 (a_2 h)$ for $a_1, a_2 \in A$ and $h \in H$.\\
    Then we take the representation of an $a \in A$, $\pi (a)$:
    \begin{align}
        (\pi(a_1)\pi(a_2))h = \pi(a_1)(\pi(a_2) h) = (T_1T_2) h = T_1 (T_2 h)
    \end{align}
    For $T_1, T_2 \in L(H)$, which operate naturally from the left on $h$.
\end{MyExercise}

\begin{MyExercise}
    \textbf{
    Show that $A$ is a bimodule ${}_A A_A$ with itself.
}\newline

    $\gamma: A\times A\times A \rightarrow A$ which is given by the inner product of the $*$-algebra.
\end{MyExercise}
