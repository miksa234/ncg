\subsection{Noncommutative Geometric Spaces}
\subsubsection{$*$-Algebra}
To grasp the idea of encoding geometrical data into a spectral triple we
introduce the first ingredient of a spectral triple, an unital $*$ algebra.
\begin{mydefinition}
    A \textit{vector space} $A$ over $\mathbb{C}$ is called a
    \textit{complex, unital Algebra} if for all $a,b \in A$:
            \begin{align}
            A \times A \rightarrow A\\
            (a,\ b)\ &\mapsto \ a\cdot b,
            \end{align}
    with an identity element:
            \begin{align}
            1a = a1 =a.
            \end{align}
    Extending the definition, a $*$-algebra is an algebra $A$ with a \textit{conjugate linear map (involution)} $*:A\ \rightarrow  A$,
    $\forall a, b \in A$ satisfying
    \begin{align}
        (a\ b)^* &= b^*a^*,\\
        (a^*)^* &= a.
    \end{align}
\end{mydefinition}
In the following all unital algebras are referred to as algebras.

\subsubsection{Finite Discrete Space}
Let us consider an example, a $*$-algebra of continuous functions $C(X)$
on a discrete topological space $X$ with $N$ points. Functions of a
continuous $*$-algebra $C(X)$ assign values to $\mathbb{C}$ and for $f,\ g \in
C(X)$, $\lambda \in \mathbb{C}$ and $x \in X$ they provide the following structure:
\begin{itemize}
    \item \textit{pointwise linear}
        \begin{align}
            (f + g)(x) &= f(x) + g(x),\\
            (\lambda\ f)(x) &= \lambda (f(x)),
        \end{align}
    \item \textit{pointwise multiplication}
        \begin{align}
        f\ g\ (x) = f(x)g(x),
        \end{align}
    \item \textit{pointwise involution}
        \begin{align}
        f^*(x) = \overline{f(x)}.
        \end{align}
\end{itemize}
The $*$-algebra $C(X)$ is \textit{isomorphic} to a $*$-algebra $\mathbb{C}^N$
with involution ($N$ number of points in $X$), we write $C(X) \simeq
\mathbb{C}^N$.  Isomorphisms are bijective maps that preserve structure and
don't lose physical information.  A function $f:X\ \rightarrow\ \mathbb{C}$
can be represented with $N \times N$ diagonal matrices, where each diagonal
value represents the function value at the corresponding $i$-th point for $i
= 1,...,N$. Matrix multiplication and hermitian conjugation of
matrices we have a preserving structure.

Moreover we can \textit{map} between finite discrete spaces $X_1$ and $X_2$ with a
function
\begin{align}
    \phi:\ X_1 \rightarrow\ X_2.
\end{align}
For every such map there exists a corresponding map
\begin{align}
    \phi ^*:C(X_2)\ \rightarrow C(X_1),
\end{align}
which `pulls back' values even if $\phi$ is not bijective.
Note that the pullback does not map points back, but maps functions on an $*$-algebra $C(X)$.
The pullback, in literature often called a $*$-homomorphism or a $*$-algebra map under
pointwise product has the following properties
\begin{align}
     \phi ^*(f\ g) = \phi ^*(f)\ \phi ^*(g),
     \phi ^*(\overline{f}) = \overline{\phi ^*(f)},
     \phi ^*(\lambda\ f + g) = \lambda\ \phi ^*(f) + \phi ^*(g).
\end{align}
%------------ Exercise
    The map $\phi :X_1\ \rightarrow \ X_2$ is an injective (surjective) map,
    if only and if the corresponding pullback $\phi ^* :C(X_2)\ \rightarrow \
    C(X_1)$ is surjective (injective). To clarify let us say that $X_1$ has $n$ points and
    $X_2$ with $m$ points. Then there are three different cases, first $n=m$ and
    obviously $\phi$ is bijective and $\phi ^*$ too. Then $n >  m$, in this case
    $\phi$ assigns $n$ points to $m$ points when $n >  m$, which is by definition
    surjective. On the other hand $\phi ^*$ assigns $m$ points to $n$ points when
    $n >  m$, which is by definition injective. Lastly $n < m $, which is
    completely analogous to the case $n > m$.
%------------ Exercise

\begin{mydefinition}
    A \textit{(complex) matrix algebra} A is a direct sum, for $n_i, N \in
    \mathbb{N}$
    \begin{align}
        A = \bigoplus _{i=1}^{N} M_{n_i}(\mathbb{C}).
    \end{align}
    The involution is the hermitian conjugate. A $*$ algebra with involution is referred to as
    a matrix algebra
\end{mydefinition}

To summarize, from a topological discrete space $X$, we can construct a
$*$-algebra $C(X)$ which is isomorphic to a matrix algebra $A$. Then the
question instantly arises, if we can construct $X$ given $A$? For a matrix
algebra $A$, which in most cases is not commutative, the answer is generally
no. Hence there are two options. We can restrict ourselves to commutative
matrix algebras, which are the vast minority and not physically interesting.
Or we can allow more morphisms (isomorphisms) between matrix algebras.

\subsubsection{Finite Inner Product Spaces and Representations}
Until now we have looked at finite topological discrete spaces, moreover we can consider a
finite dimensional inner product space $H$ (finite Hilbertspaces), with inner product
$(\cdot,\cdot)\rightarrow \mathbb{C}$. We denote $L(H)$ as the $*$-algebra of operators on $H$
equipped with a product given by composition and involution of the adjoint, $T \mapsto T^*$.
Then $L(H)$ is a \textit{normed vector space} with
\begin{align}
    \|T\|^2 &= \sup_{h \in H}\big\{(T\ h,\ T\ h): (h,\ h) \leq 1\big|\ T
    \in L(H)\big \},\\
    \|T\| &= \sup\big\{\sqrt{\lambda}:\; \lambda \text{ eigenvalue of } T\big\}.
\end{align}
The Hilbert space allows us to define representations of $*$-algebras.
\begin{mydefinition}
    The \textit{representation} of a finite dimensional $*$-algebra $A$ is a
    pair $(H, \pi)$, where $H$ is a finite dimensional inner product space
    and $\pi$ is a $*$-\textit{algebra map}
    \begin{align}
        \pi:A\ \rightarrow \ L(H).
    \end{align}
    We call the representation $(H, \pi)$ \textit{irreducible} if
    \begin{itemize}
        \item $H \neq \emptyset$,
        \item only $\emptyset$ or $H$ is invariant under the action of $A$ on
            $H$.
    \end{itemize}
\end{mydefinition}
Here are some examples of reducible and irreducible representations
\begin{itemize}
    \item For $A = M_n(\mathbb{C})$ the representation $H=\mathbb{C}^n$, $A$ acts as matrix multiplication\\
            $H$ is irreducible.
    \item For $A = M_n(\mathbb{C})$ the representation $H=\mathbb{C}^n\oplus \mathbb{C}^n$, with $a \in A$ acting
        in block form \\ $\pi: a \mapsto \big(\begin{smallmatrix} a & 0\\ 0 & a \end{smallmatrix}\big)$ is
            reducible.
\end{itemize}
Naturally there are also certain equivalences between different
representations.
\begin{mydefinition}
Two representations of a $*$-algebra $A$, $(H_1, \pi _1)$ and
$(H_2, \pi _2)$  are called \textit{unitary equivalent} if there exists a map
$U: H_1 \rightarrow H_2$ such that.
    \begin{align}
        \pi _1(a) = U^* \pi _2(a) U
    \end{align}
\end{mydefinition}

Furthermore we define a mathematical structure called the structure space,
which will become important later when speaking of the duality between a
spectral triple and a geometrical space.
\begin{mydefinition}
    Let $A$ be a $*$-algebra then, $\hat{A}$ is called the structure space of all \textit{unitary equivalence classes
    of irreducible representations of A}.
\end{mydefinition}
%------------- EXERCISE
    Given a representation $(H, \pi)$ of a $*$-algebra $A$, the \textbf{commutant} $\pi (A)'$ of $\pi (A)$ is defined as a set
    of operators in $L(H)$ that commute with all $\pi (a)$
    \begin{align}
        \pi (A)' = \big\{T \in L(H):\ \pi(a)\ T = T\ \pi(a) \;\; \forall a\in
        A\big\}
    \end{align}
    The commutant $\pi (A)'$ is also a $*$-algebra, since it has unital,
    associative and involutive properties.  The unitary property is given by
    the unital operator of the $*$-algebra of operators $L(H)$, which exists
    by definition because $H$ is a inner product space. Associativity is
    given by the $*$-algebra of $L(H)$, where $L(H) \times L(H)~\mapsto
    L(H)$, which is associative by definition. The involutive property is
    also given by the $*$-algebra $L(H)$ with a map $*: L(H) \mapsto L(H)$
    only for a $T \in H$ that commutes with $\pi (a)$.
%------------- EXERCISE

%------------- EXERCISE
    For a unital algebra $*$-algebra $A$, the matrices $M_n(A)$ with entries
    in $A$ form a unital $*$-algebra, because the unitary operation in
    $M_n(A)$ is given by the identity Matrix, which exists in every
    entry in $M_n(A)$ and behaves like in $A$. Associativity is given by
    matrix multiplication. Lastly, involution is given by the conjugate
    transpose.

    Consider a representation $\pi :A\ \rightarrow \ L(H)$ of a $*$-algebra
    $A$ and set $H^n = H \oplus ... \oplus H$, $n$ times. Then we have the following
    representation $\tilde{\pi}:M_n(A) \rightarrow L(H^n)$ for the Matrix
    Algebra with $\tilde{\pi}((a_{ij})) = (\tilde{\pi}(a_{ij})) \in M_n(A)$,
    since a direct isomorphisms of $A \simeq M_n(A)$ and $H \simeq H^n$
    exists. Meaning $\tilde{\pi}$ is a valid reducible representation.

    By looking at $\tilde{\pi}:M_n(A) \rightarrow L(H^n)$ a $*$ algebra
    representation of $M_n(A)$. We see that $\pi: A \rightarrow L(H^n)$ is a representation of $A$.
    The fact that $\tilde{\pi}$ and $\pi$ are unitary equivalent, there is
    a map $U: H^n \rightarrow H^n$ given by $U=\mathbbm{1}_n$, thus
    \begin{align}
        \pi (a) &= \mathbbm{1}_n^*\ \tilde{\pi}((a_{ij})), \\
        \mathbbm{1}_n &= \tilde{\pi}((a_{ij})) = \pi (a_{ij})
    \Rightarrow a_{ij} = a\ \mathbbm{1}_n.
    \end{align}
%------------- EXERCISE


With help of the structure space $\hat{A}$, a commutative matrix algebra can be used to reconstruct a discrete space.
Since $A \simeq \mathbb{C}^N$ all irreducible representation are of the form
\begin{align}
   \pi _i:(\lambda_1,...,\lambda_N)\in \mathbb{C}^N \mapsto \lambda_i \in
   \mathbb{C}
\end{align}
for $i = 1,...,N$, and thus $\hat{A} \simeq \{1,...,N\}$.
We can conclude that there is a duality between discrete spaces and
commutative matrix algebras. This duality is called the \textit{finite
dimensional Gelfand duality}

Our aim is to make a further generalization by constructing a duality between
finite dimensional spaces and \textit{equivalence classes} of matrix
algebras that preserves general non-commutativity of matrices. Equivalence
classes are described by a concept of isomorphisms between matrix
algebras called \textit{Morita Equivalence}.

\subsubsection{Algebraic Modules}
An important part of the Morita Equivalence are algebraic modules, later
extended by Hilbert bimodules.
\begin{mydefinition}
    Let $A$, $B$ be algebras (need not be matrix algebras)
    \begin{enumerate}
        \item \textit{left} A-module is a vector space $E$, that carries a left
            representation of $A$, that is $\exists$ a bilinear map $\gamma: A
            \times E \rightarrow E$ with
            \begin{align}
                (a_1\ a_2)\cdot e = a_1 \cdot (a_2 \cdot e);\;\;\; a_1, a_2 \in
                A, e \in E.
            \end{align}
        \item \textit{right} B-module is a vector space $F$, that carries a
            right representation of $A$, that is there exists a bilinear map
            $\gamma: F \times B \rightarrow F$ with
            \begin{align}
                f \cdot (b_1\ b_2)= (f \cdot b_1) \cdot b_2;\;\;\; b_1, b_2 \in B, f \in F
            \end{align}
        \item \textit{left} A-module and \textit{right} B-module is a
            \textit{bimodule}, a vector space $E$ satisfying
            \begin{align}
                a \cdot (e \cdot b)= (a \cdot e) \cdot b;\;\;\;  a \in A, b \in B, e \in E
            \end{align}
    \end{enumerate}
\end{mydefinition}
An $A$-\textbf{module homomorphism} is linear map $\phi: E\rightarrow F$ which respects the
representation of A, e.g.\ for left module.
\begin{align}
    \phi (a\ e) = a \phi (e); \;\;\; a \in A, e \in E.
\end{align}
We will use the notation
\begin{itemize}
    \item ${}_A E$, for left $A$-module $E$;
    \item ${}_A E_B$, for right $B$-module $F$;
    \item ${}_A E_B$, for $A$-$B$-bimodule $E$, simply bimodule.
\end{itemize}
%------------------- EXERCISE
From a simple observation, we see that an arbitrary representation $\pi : A
\rightarrow L(H)$ of a $*$-algebra A, turns H into a left module ${}_A H$.  If
$_A H$ than $(a_1\ a_2) h = a_1 (a_2\ h)$ for $a_1, a_2 \in A$ and $h \in H$. We
take the representation of an $a \in A$, $\pi (a)$, and write
\begin{align}
    \big(\pi(a_1)\ \pi(a_2)\big)h = \pi(a_1)\big(\pi(a_2)\ h\big) =
    \big(T_1\ T_2\big) h = T_1 \big(T_2\ h\big)
\end{align}
For $T_1, T_2 \in L(H)$, which operate naturally from the left on $h$.

%------------------- EXERCISE
%------------------- EXERCISE

Furthermore notice that that an $*$-algebra $A$ is a bimodule ${}_A A_A$ with
itself, given by the map
\begin{align}
    \gamma: A\times A\times A \rightarrow A,
\end{align}
which is the inner product of a $*$-algebra.
%------------------- EXERCISE

\subsubsection{Balanced Tensor Product and Hilbert Bimodules}
In this chapter we introduce the balanced tensor product later called the
Kasparov product. This operation allows us to naturally construct a bimodule
of a third algebra in chapter \ref{chap: kasparov product}.
\begin{mydefinition}
    Let $A$ be an algebra, $E$ be a \textit{right} $A$-module and $F$ be a
    \textit{left} $A$-module.  The \textit{balanced tensor product} of $E$ and
    $F$ forms a $A$-bimodule.
    \begin{align}
        E \otimes _A F := E \otimes F / \left\{\sum _i e_i a_i \otimes f_i -
        e_i \otimes a_i f_i : \;\;\; a_i \in A,\ e_i \in E,\ f_i \in F
    \right\}.
    \end{align}
\end{mydefinition}
The symbol $/$ denotes the quotient space. By careful examination we can say
that the operation $\otimes _A$ takes two left/right modules and makes a
bimodule. Additionally with the help of the tensor product of the two modules and the quotient
space which takes out all the elements from the tensor product that don't
preserver the left/right representation and that are duplicates.
\begin{mydefinition}
    Let $A$, $B$ be \textit{matrix algebras}. The \textit{Hilbert bimodule} for
    $(A, B)$ is given by an $A$-$B$-bimodue $E$ and by an $B$-valued
    \textit{inner product} $\langle \cdot,\cdot\rangle_E: E\times E \rightarrow
    B$, which satisfies the following conditions for $e, e_1, e_2 \in
    E,\ a \in A$ and $b \in B$
\begin{align}
    \langle e_1,\ a\cdot e_2\rangle_E &= \langle a^*\cdot e_1,\ e_2\rangle_E
    \;\;\;\; & \text{sesquilinear in $A$},\\
    \langle e_1,\ e_2 \cdot b\rangle_E
             &= \langle e_1,\ e_2\rangle_E b \;\;\;\; & \text{scalar in $B$},\\
    \langle e_1,\ e_2\rangle_E &= \langle e_2,\ e_1\rangle^*_E \;\;\;\; &
    \text{hermitian}, \\
    \langle e,\ e\rangle_E &\ge 0 \;\;\;\; & \text{equality
    holds iff $e=0$}.
\end{align}
We denote $KK_f(A,\ B)$ as the set of all \textit{Hilbert bimodules} of $(A,\ B)$.
\end{mydefinition}
%-------------- EXERCISE

And indeed the Hilbert bimodule extension takes a representation $\pi:\ A \
\rightarrow L(H)$ of a matrix algebra $A$ and turns $H$ into a Hilbert bimodule for
$(A, \mathbb{C})$, because the representation for a $a \in A$, $\pi(a)=T \in L(H)$ fulfills
the conditions of the $\mathbb{C}$-valued inner product for $h_1, h_2 \in H$
\begin{itemize}
    \item $\langle h_1,\ \pi(a)\ h_2\rangle _\mathbb{C} = \langle h_1,\ T\ h_2\rangle _\mathbb{C} =
        \langle T^* h_1, h_2\rangle _\mathbb{C}$, $T^*$ given by the adjoint,
    \item $\langle h_1,\ h_2\ \pi(a)\rangle _\mathbb{C} = \langle h_1,\ h_2\
        T\rangle _\mathbb{C} = \langle h_1,\ h_2\rangle _\mathbb{C}$ , $T$ acts
        from the left,
    \item $\langle h_1,\ h_2\rangle _\mathbb{C}^* = \langle h_2,\ h_1\rangle _\mathbb{C}$, hermitian because of the
        $\mathbb{C}$-valued inner product
    \item $\langle h_1,\ h_2\rangle  \ge 0$, $\mathbb{C}$-valued inner product.
\end{itemize}
%-------------- EXERCISE

%-------------- EXERCISE
Take again the $A-A$ bimodule given by an $*$-algebra $A$. By looking at the
following inner product $\langle \cdot,\cdot\rangle_A:A \times A \rightarrow A$
\begin{align}
    \langle a,\ a\rangle_A = a^*a' \;\;\;\; a,a'\in A.
    \label{eq:inner-product},
\end{align}
it becomes clear that $A \in KK_f(A,\ A)$.
Simply checking the conditions in $\langle \cdot, \cdot\rangle _A$ for
$a, a_1, a_2 \in~A$
\begin{align}
    &\langle a_1,\ a\cdot a_2\rangle _A = a^* a\cdot a_2 =
        (a^*a_1)^*\ a_2 = \langle  a^*\ a_1,\ a_2\rangle, \\
   &\langle a_1,\ a_2 \cdot a\rangle _A = a^*_1\ (a_2\cdot a) =
    (a^*a_2)\cdot a = \langle a_1,\ a_2\rangle _A\ a,\\
   &\langle a_1,\ a_2\rangle _A^* = (a_1^*\ a_2)^* = a_2^*\
        (a_1^*)^* = a_2^*\ a_1 = \langle a_2,\ a_1\rangle.
\end{align}

%-------------- EXERCISE

%-------------- EXAMPLE
%As an for overview consider a $*$ homomorphism between two matrix
%algebras $\phi:A\rightarrow B$, we can construct a Hilbert bimodule
%$E_{\phi} \in KK_f(A, B)$ in the following way. We let $E_{\phi}$ be $B$ in
%as an vector space and an inner product from above in equation
%\eqref{eq:inner-product}, with $A$ acting on the left with $\phi$.
%\begin{align}
%    a\cdot b = \phi(a)\ b
%\end{align}
%for $a\in A, b\in E_{\phi}$.
%-------------- EXAMPLE

\subsubsection{Kasparov Product and Morita Equivalence\label{chap: kasparov
product}}
\begin{mydefinition}
    Let $E \in KK_f(A, B)$ and $F \in KK_F(B, D)$ the \textit{Kasparov product} is defined as
    with the balanced tensor product
    \begin{align}
        F \circ E := E \otimes _B F.
    \end{align}
    Then $F\circ E \in KK_f(A,D)$ is equipped with a $D$-valued inner product
    \begin{align}
        \langle e_1 \otimes f_1,\ e_2 \otimes f_2\rangle _{E\otimes _B F} =
        \langle f_1,\langle e_1,\ e_2\rangle _E f_2\rangle _F
    \end{align}
\end{mydefinition}

%-------------- EXERCISE
The Kasparov product for $*$-algebra homomorphism $\phi: A \rightarrow B$ and
$\psi: B \rightarrow C$ are isomorphisms in the sense that
\begin{align}
                E_{\psi} \circ E_{\phi}\ \equiv\ E_{\phi} \otimes _B E_{\psi}\
                \simeq\
                E_{\psi \circ \phi} \in KK_f(A,C).
\end{align}

The direct computation for $a \in A$, $b\in B$, and $c\in C$ which is $\psi
\circ \phi$ shows us
\begin{align}
a \cdot b \cdot c = \psi(\phi (a) \cdot b) \cdot c
\end{align}
An interesting case arises when looking at $E_{\text{id}_A} \simeq A \in
KK_f(A,A)$, where $\text{id}_A$ is the identity in $A$. Let $E_{\phi}$ be $A$
with a natural right representation. It follows that $E_{\phi}\simeq A$, where
an inner product, acting from the left on $A$ for $\phi$, $a', a\in A$ reads
\begin{align}
    a'\ a = (\phi(a')\ a) \in A,
\end{align}
which is satisfied only by $\phi = \text{id}_A$.

\begin{mydefinition}
    Let $A$, $B$ be \textit{matrix algebras}. They are called \textit{Morita equivalent} if there
    exists an $E \in KK_f(A, B)$ and an $F \in KK_f(B, A)$ such that
    \begin{align}
        E \otimes _B F \simeq A \;\;\; \text{and} \;\;\; F \otimes _A E \simeq
        B,
    \end{align}
    where $\simeq$ denotes the isomorphism between Hilbert bimodules and note
    that $A$ or $B$ is a bimodule by itself.
\end{mydefinition}

Since we land in the same space as we started, the modules $E$ and $F$ are
each others inverse in regards to the Kasparov Product. More clearly, in the
definition we have $E \in KK_f(A, B)$. Naturally we start from $A$ and $E
\otimes _B F$, which lands in $A$. On the other hand we have $F \in KK_f(B,
D)$ and start from $B$, $F \otimes _A E$, which lands in $B$.

%------------- EXERCISE
By definition  $E \otimes _B F$ is a $A-D$ bimodule. Since
\begin{align}
    E \otimes _B F = E \otimes F / \bigg\{\sum_i\ e_i\ b_i \otimes f_i - e_i
        \otimes b_i\ f_i\ \big|\;\; e_i \in E_i,\ b_i \in B,\ f_i \in F\bigg\},
\end{align}
the last part takes out all tensor product elements of $E$ and $F$ that don't
preserver the left/right representation and that are duplicates.

Additionally $\langle \cdot,\cdot\rangle _{E\oplus _B F}$ defines a $D$ valued
inner product, as $\langle e_1,\ e_2\rangle _E \in B$ and $\langle f_1,\ f_2\rangle _F \in C$ by
definition. So for $\langle e_1,\ e_2\rangle _E =b$ we have
\begin{align}
    \langle e_1 \otimes f_1,\ e_2 \otimes f_2\rangle _{E\otimes _B F} = \langle
    f_1,\ \langle e_1,\ e_2\rangle _E\ f_2\rangle _F = \langle f_1,\ b\ f_2\rangle _F \in C
\end{align}
%------------- EXERCISE
%------------- EXAMPLE
Picking up the example of $(A, A)$, the Hilbert bimodule $A$, we can
consider an $E \in KK_f(A,B)$ for
\begin{align}
    E \circ A = A\oplus _A E \simeq E.
\end{align}
We conclude, that $_A A_A$ is the identity element in the Kasparov product (up
to isomorphism).
%------------- EXAMPLE
%------------- EXAMPLE
Let us examine another example for $E = \mathbb{C}^n$, which is a
$(M_n(\mathbb{C}), \mathbb{C})$ Hilbert bimodule with the standard $\mathbb{C}$
inner product. Further let $F = \mathbb{C}^n$, which is a $(\mathbb{C},
M_n(\mathbb{C}))$ Hilbert bimodule by right matrix multiplication with
$M_n(\mathbb{C})$ valued inner product, we can write
    \begin{align}
        \langle v_1, v_2\rangle =\bar{v_1}v_2^t \;\; \in M_n(\mathbb{C}).
    \end{align}
If we take the Kasparov product of $E$ and $F$
   \begin{align}
       F\circ E\ &=\  E\otimes _{\mathbb{C}}F\ \;\;\;\;\;\; \simeq \
        M_n(\mathbb{C}),\\
        E\circ F\ &=\ F\otimes _{M_n(\mathbb{C})}E\ \simeq\ \mathbb{C},
    \end{align}
we see that $M_n(\mathbb{C})$ and $\mathbb{C}$ are Morita equivalent!
%------------- EXAMPLE

\begin{mylemma}
    Two matrix algebras are Morita Equivalent if, and only if their their structure spaces
    are isomorphic as discreet spaces (have the same cardinality / same number
    of elements).
\end{mylemma}
\begin{proof}
    Let $A$, $B$ be \textit{Morita equivalent}. Then there exist the modules
    $_A E_B$ and $_B F_A$ with
    \begin{align}
        E \otimes _B F \simeq A \;\;\; \text{and} \;\;\; F \otimes _A E \simeq
        B.
    \end{align}
    Also consider $[(\pi _B, H)] \in \hat{B}$. We can construct a
    representation of $A$, which reads
    \begin{align}
        \pi _A \rightarrow L(E \otimes _B H)\;\;\; \text{with} \;\;\; \pi _A(a)
        (e \otimes v) = a e \otimes w
    \end{align}
    Vice versa, we have $[(\pi _A, W)] \in \hat{A}$ and we can construct $\pi _B$
    as
    \begin{align}
        \pi _B: B \rightarrow L(F \otimes _A W) \;\;\; \text{and}\;\;\; \pi
        _B(b) (f\otimes w) = bf\otimes w.
    \end{align}
    Now we need to show that the representation $\pi _A$ is irreducible if and
    only if $\pi _B$ is irreducible. For $(\pi _B, H)$ to be irreducible, we
    need $H \neq \emptyset$ and only $\emptyset$ or $H$ to be invariant under
    the Action of $B$ on $H$. Than $E\otimes _B H$ and $E\otimes _B H \simeq A$
    cannot be empty, because $E$ preserves left representation of $A$.

    Lastly we need to check if the association of the class $[\pi _A]$ to $[\pi
    _B]$ is independent of the choice of representatives $\pi _A$ and $\pi _B$.
    The important thing is that $[\pi _A] \in \hat{A}$ respectively $[\pi _B] \in
    \hat{B}$, hence any choice of representation is irreducible, because the
    structure space denotes all unitary equivalence classes of irreducible
    representations.

    Note that the statements $E \simeq H$ and $F \simeq W$ are not particularly
    true, since all infinite dimensional Hilbert spaces are isomorphic.  Here
    we are looking at finite dimensional Hilbert spaces. Another thing to keep
    in mind, is that for $[\pi _B, H] \in \hat{B}$ and looking at algebraic
    bimodules, we know that $H$ is a bimodule of $B$, hence $E \otimes _B
    H\simeq A$, and for $[\pi _A, W]$, which is the same.
    Finally we can conclude, that these maps are each others inverses, thus
    $\hat{A} \simeq \hat{B}$.
\end{proof}

\begin{mylemma}
    The matrix algebra $M_n(\mathbb{C})$ has a unique irreducible
    representation (up to isomorphism) given by the defining representation on
    $\mathbb{C}^n$.
\end{mylemma}
\begin{proof}
    We know $\mathbb{C}^n$ is a irreducible representation of $A=
    M_n(\mathbb{C})$. Let $H$ be irreducible and of dimension $k$, then we
    define a map
    \begin{align}
        \phi : A\oplus...\oplus A &\rightarrow H^* \\
    (a_1,...,a_k)&\mapsto e^1\circ a_1^t+...+e^k\circ a_k^t,
    \end{align}
where $\{e^1,...,e^k\}$ is the basis of the dual space $H^*$ and
$(\circ)$ being the pre-composition of elements in $H^*$ and $A$ acting on $H$.
This forms a morphism of $M_n(\mathbb{C})$ modules, provided a matrix $a \in A$
acts on $H^*$ with $v\mapsto v\circ a^t$ ($v\in H^*$).  Furthermore this
morphism is surjective, thus making the pullback $\phi ^*:H\mapsto (A^k)^*$
injective.  Now identify $(A^k)^*$ with $A^k$ as a $A$-module and note that
$A=M_n(\mathbb{C}) \simeq \oplus ^n \mathbb{C}^n$ as a n A module.  It follows
that $H$ is a submodule of $A^k \simeq \oplus ^{nk}\mathbb{C}$. By
irreducibility $H \simeq \mathbb{C}$.
\end{proof}

%---------------- EXAMPLE
Let us look at an example, two matrix algebras $A$, and $B$.
\begin{align}
    A = \bigoplus ^N_{i=1} M_{n_i}(\mathbb{C}), \;\;\;
    B = \bigoplus ^M_{j=1} M_{m_j}(\mathbb{C}).
\end{align}
Let $\hat{A} \simeq \hat{B}$, this implies $N=M$. Further define $E$ with $A$
acting by block-diagonal matrices on the first tensor and B acting in the same
manner on the second tensor. Define $F$ vice versa, ultimately reading
\begin{align}
    E:= \bigoplus _{i=1}^N \mathbb{C}^{n_i} \otimes \mathbb{C}^{m_i}, \;\;\;
    F:= \bigoplus _{i=1}^N \mathbb{C}^{m_i} \otimes \mathbb{C}^{n_i}.
\end{align}
When we calculate the Kasparov product we get the following
\begin{align}
    E \otimes _B F &\simeq \bigoplus _{i=1}^N (\mathbb{C}^{n_i}\otimes\mathbb{C}^{m_i})
        \otimes _{M_{m_i}(\mathbb{C})} (\mathbb{C}^{m_i}\otimes\mathbb{C}^{n_i}) \\
                   &\simeq \bigoplus _{i=1}^N \mathbb{C}^{n_i}\otimes
                   \left(\mathbb{C}^{m_i}\otimes _{M_{m_i}(\mathbb{C})}\mathbb{C}^{m_i}\right)
                    \oplus \mathbb{C}^{n_i} \\
                   &\simeq \bigoplus _{i=1}^N
                   \mathbb{C}^{m_i}\otimes\mathbb{C}^{n_i} \simeq A.
\end{align}
On the other hand we get
\begin{align}
    F \otimes _A E \simeq B.
\end{align}
%---------------- EXAMPLE

To summarize, there is a duality between finite spaces and Morita equivalence
classes of matrix algebras. Furthermore by replacing $*$-homomorphism $A\rightarrow B$
with Hilbert bimodules $(A,B)$ we introduce a richer structure of morphism
between matrix algebras.
