\subsection{Noncommutative Geometric Spaces}
\subsubsection{$*$-Algebra}
To grasp the idea of encoding geometrical data into a spectral triple we
introduce the first ingredient of a spectral triple, an unital $C^*$ algebra.
\begin{definition}
    A \textit{vector space} $A$ over $\mathbb{C}$ is called a \textit{complex, unital Algebra} if, \\
    $\forall a,b \in A$ :
    \begin{enumerate}
        \item
            $A \times A \rightarrow A$,
            $(a, b)\ \mapsto \ a\cdot b$,
        \item with an identity element
            $1a = a1 =a$.
    \end{enumerate}
    Extending the definition, a $*$-algebra is an algebra $A$ with a \textit{conjugate linear map (involution)} $*:A\ \rightarrow  A$,
    $\forall a, b \in A$ satisfying:
    \begin{enumerate}
        \item
            $(ab)^* = b^*a^*$,
        \item
            $(a^*)^* = a$.
    \end{enumerate}
\end{definition}
In the following all unital algebras are referred to as algebras.

\subsubsection{Finite Discrete Space}
Let us consider an example of an $*$-algebra of continuous functions $C(X)$
on a discrete topological space $X$ with $N$ points. Functions of a
continuous $*$-algebra $C(X)$ assign values to $\mathbb{C}$, thus $f, g \in
C(X)$, $\lambda \in \mathbb{C}$ and $x \in X$ they provide the following structure:
\begin{itemize}
    \item \textit{pointwise linear} \\
      $(f + g)(x) = f(x) + g(x)$,\\
      $(\lambda f)(x) = \lambda (f(x)),$
    \item \textit{pointwise multiplication} \\
        $fg(x) = f(x)g(x)$,
    \item \textit{pointwise involution} \\
        $f^*(x) = \overline{f(x)}.$
\end{itemize}
The $*$-algebra $C(X)$ is \textit{isomorphic} to a $*$-algebra $\mathbb{C}^N$
with involution ($N$ number of points in $X$), we write $C(X) \simeq
\mathbb{C}^N$.  Isomorphisms are bijective maps that preserve structure and
don't lose physical information.  A function $f:X\ \rightarrow\ \mathbb{C}$
can be represented with $N \times N$ diagonal matrices, where each diagonal
value represents the function value at the corresponding $i$-th point for $i
= 1,...,N$. Because of matrix multiplication and hermitian conjugate of
matrices we have a preserving structure.

Moreover we can \textit{map} between finite discrete spaces $X_1$ and $X_2$ with a
function
\begin{align}
    \phi:\ X_1 \rightarrow\ X_2.
\end{align}
For every such map there exists a corresponding map
\begin{align}
    \phi ^*:C(X_2)\ \rightarrow C(X_1),
\end{align}
which `pulls back' values even if $\phi$ is not bijective.
Note that the pullback doesn't map points back, but maps functions on an $*$-algebra $C(X)$.
The pullback, in literature often called a $*$-homomorphism or a $*$-algebra map under
pointwise product has the following properties
\begin{itemize}
    \item $\phi ^*(fg) = \phi ^*(f) \phi ^*(g)$,
    \item $\phi ^*(\overline{f}) = \overline{\phi ^*(f)}$,
    \item $\phi ^*(\lambda f + g) = \lambda \phi ^*(f) + \phi ^*(g)$.
\end{itemize}
%------------ Exercise
    The map $\phi :X_1\ \rightarrow \ X_2$ is an injective (surjective) map,
    if only if the corresponding pullback $\phi ^* :C(X_2)\ \rightarrow \
    C(X_1)$ is surjective (injective). Let us say, that $X_1$ has $n$ points and
    $X_2$ with $m$ points. Then there are three different cases, first $n=m$ and
    obviously $\phi$ is bijective and $\phi ^*$ too. Then $n >  m$, in this case
    $\phi$ assigns $n$ points to $m$ points when $n >  m$, which is by definition
    surjective. On the other hand $\phi ^*$ assigns $m$ points to $n$ points when
    $n >  m$, which is by definition injective. Lastly $n < m $, which is
    completely analogous to the case $n > m$.
%------------ Exercise

\subsubsection{Matrix Algebras}
\begin{definition}
    A \textit{(complex) matrix algebra} A is a direct sum, for $n_i, N \in
    \mathbb{N}$
    \begin{align}
        A = \bigoplus _{i=1}^{N} M_{n_i}(\mathbb{C}).
    \end{align}
    The involution is the hermitian conjugate, a $*$ algebra with involution is referred to as
    a matrix algebra
\end{definition}
From a topological discrete space $X$, we can construct a $*$-algebra
$C(X)$ which is isomorphic to a matrix algebra $A$. Then the question instantly
arises, if we can construct $X$ given $A$? For a matrix algebra $A$,
which in most cases is not commutative, the answer is generally no.

Thus there are two options. We can restrict ourselves to commutative matrix algebras,
which are the vast minority and not physically interesting.
Or we can allow more morphisms (isomorphisms) between matrix algebras.

\subsubsection{Finite Inner Product Spaces and Representations}
Until now we looked at finite topological discrete spaces, moreover we can consider a
finite dimensional inner product space $H$ (finite Hilbert-spaces), with inner product
$(\cdot,\cdot)\rightarrow \mathbb{C}$. We denote $L(H)$ as  the $*$-algebra of operators on $H$
equipped with a product given by composition and involution of the adjoint, $T \mapsto T^*$.
Then $L(H)$ is a \textit{normed vector space} with
\begin{align}
    &\|T\|^2 = \text{sup}_{h \in H}\{(Th,Th): (h,h) \leq 1\} \hspace{0.1\textwidth} T \in L(H) \\
    &\|T\| = \text{sup}\{\sqrt{\lambda}: \lambda \text{ eigenvalue of } T\}
\end{align}
This allows us to define representations of $*$-algebras.
\begin{definition}
    The \textit{representation} of a finite dimensional $*$-algebra $A$ is a
    pair $(H, \pi)$, where $H$ is a finite dimensional inner product space
    and $\pi$ is a $*$-\textit{algebra map}
    \begin{align}
        \pi:A\ \rightarrow \ L(H).
    \end{align}
    We call the representation $(H, \pi)$ \textit{irreducible} if
    \begin{itemize}
        \item $H \neq \emptyset$,
        \item only $\emptyset$ or $H$ is invariant under the action of $A$ on
            $H$.
    \end{itemize}
\end{definition}
Here are some examples of reducible and irreducible representations
\begin{itemize}
    \item For $A = M_n(\mathbb{C})$ the representation $H=\mathbb{C}^n$, $A$ acts as matrix multiplication\\
            $H$ is irreducible.
    \item For $A = M_n(\mathbb{C})$ the representation $H=\mathbb{C}^n\oplus \mathbb{C}^n$, with $a \in A$ acting
        in block form \\ $\pi: a \mapsto \big(\begin{smallmatrix} a & 0\\ 0 & a \end{smallmatrix}\big)$ is
            reducible.
\end{itemize}
Naturally there are also certain equivalences between different
representations.
\begin{definition}
Two representations of a $*$-algebra $A$, $(H_1, \pi _1)$ and
$(H_2, \pi _2)$  are called \textit{unitary equivalent} if there exists a map
$U: H_1 \rightarrow H_2$ such that.
    \begin{align}
        \pi _1(a) = U^* \pi _2(a) U
    \end{align}
\end{definition}

Furthermore we define a mathematical structure called the structure space,
which will later become important, when speaking of the duality between a spectral
triple and a space.
\begin{definition}
    Let $A$ a $*$-algebra then, $\hat{A}$ is called the structure space of all \textit{unitary equivalence classes
    of irreducible representations of A}
\end{definition}
%------------- EXERCISE
    Given a representation $(H, \pi)$ of a $*$-algebra $A$, the \textbf{commutant} $\pi (A)'$ of $\pi (A)$ is defined as a set
    of operators in $L(H)$ that commute with all $\pi (a)$
    \begin{align}
        \pi (A)' = \{T \in L(H):\pi (a)T = T\pi (a) \;\;\; \forall a\in A\}
    \end{align}
    The commutant $\pi (A)'$ is also a $*$-algebra, because it has unital,
    associative and involutive properties.
    We note that $\pi (a) \in L(H)\ \forall a \in A$, unitary property is given
    by the unital operator of the $*$-algebra of operators $L(H)$, which exists
    by definition because H is a inner product space. Associativity is given by
    the $*$-algebra of $L(H)$, where $L(H) \times L(H) \mapsto L(H)$, which is
    associative by definition. The involutive property is also given by the $*$-algebra $L(H)$
    with a map $*: L(H) \mapsto L(H)$ only for a $T$ that commutes with $\pi (a)$.
%------------- EXERCISE

%------------- EXERCISE
    For a unital algebra $*$-algebra $A$, the matrices $M_n(A)$ with entries
    in $A$ form a unital $*$-algebra, because unitary operation in $M_n(A)$ is given by the identity Matrix, which
    has to exists in every entry in $M_n(A)$, and behaves like in $A$. Associativity is given by
    matrix multiplication. Lastly involution is given by the conjugate transpose.

    A representation $\pi :A\ \rightarrow \ L(H)$ of a $*$-algebra $A$, for
    $H^n = H \oplus ... \oplus H$, $n$ times.  Then we have the following
    representation $\tilde{\pi}:M_n(A) \rightarrow L(H^n)$ for the Matrix
    Algebra with $\tilde{\pi}((a_{ij})) = (\tilde{\pi}(a_{ij})) \in M_n(A)$.
    We have direct isomorphisms of $A \simeq M_n(A)$ and $H \simeq H^n$
    meaning $\tilde{\pi}$ is a valid reducible representation.

    Let $\tilde{\pi}:M_n(A) \rightarrow L(H^n)$ be a $*$ algebra
    representation of $M_n(A)$, then $\pi: A \rightarrow L(H^n)$ is a representation of $A$.
    The fact that $\tilde{\pi}$ and $\pi$ are unitary equivalent, there is
    a map $U: H^n \rightarrow H^n$ given by $U=\mathbbm{1}_n$, thus
    \begin{align}
        \pi (a) &= \mathbbm{1}_n^*\ \tilde{\pi}((a_{ij})), \\
        \mathbbm{1}_n &= \tilde{\pi}((a_{ij})) = \pi (a_{ij})
    \Rightarrow a_{ij} = a\mathbbm{1}_n.
    \end{align}
%------------- EXERCISE


A commutative matrix algebra can be used to reconstruct a discrete space.
The structure space $\hat{A}$ is used for this. Because $A \simeq
\mathbb{C}^N$ all
irreducible representation are of the form
\begin{align}
   \pi _i:(\lambda_1,...,\lambda_N)\in \mathbb{C}^N \mapsto \lambda_i \in
   \mathbb{C}
\end{align}
for $i = 1,...,N$ and thus $\hat{A} \simeq \{1,...,N\}$.
The conclusion is that, there is a duality between discrete spaces and
commutative matrix algebra this duality is called the \textit{finite
dimensional Gelfand duality}

Our aim is to construct a duality between finite dimensional spaces and
\textit{equivalence classes} of matrix algebras, to preserve general
non-commutativity of matrices.  Equivalence classes are described by a
generalized notion of isomorphisms between matrix algebras (\textit{Morita
Equivalence})

\subsubsection{Algebraic Modules}
\begin{definition}
    Let $A$, $B$ be algebras (need not be matrix algebras)
    \begin{enumerate}
        \item \textit{left} A-module is a vector space $E$, that carries a left representation of $A$,
            that is $\exists$ a bilinear map $\gamma: A \times E \rightarrow E$ with
            \begin{align}
                (a_1a_2)\cdot e = a_1 \cdot (a_2 \cdot e);\;\;\; a_1, a_2 \in A, e \in E
            \end{align}
        \item \textit{right} B-module is a vector space $F$, that carries a right representation of $A$,
            that is $\exists$ a bilinear map $\gamma: F \times B \rightarrow F$ with
            \begin{align}
                f \cdot (b_1b_2)= (f \cdot b_1) \cdot b_2;\;\;\; b_1, b_2 \in B, f \in F
            \end{align}
        \item \textit{left} A-module and \textit{right} B-module is a \textit{bimodule}, a vector space $E$
            satisfying
            \begin{align}
                a \cdot (e \cdot b)= (a \cdot e) \cdot b;\;\;\;  a \in A, b \in B, e \in E
            \end{align}
    \end{enumerate}
\end{definition}
Notion of A-\textbf{module homomorphism} as linear map $\phi: E\rightarrow F$ which respects the
representation of A, e.g. for left module.
\begin{align}
    \phi (ae) = a \phi (e); \;\;\; a \in A, e \in E.
\end{align}
Remark on the notation
\begin{itemize}
    \item ${}_A E$ left $A$-module $E$;
    \item ${}_A E_B$ right $B$-module $F$;
    \item ${}_A E_B$ $A$-$B$-bimodule $E$;
\end{itemize}

\begin{MyExercise}
    \textbf{
    Check that a representation of $\pi : A \rightarrow L(H)$ of a $*$-algebra A turns H into a
    left module ${}_A H$.
}\newline

    Not quite sure but \\
    $a \in A$, $h_1, h_2 \in H$, we know $\pi (a) = T \in L(H)$ than
    \begin{align}
        \langle \pi (a) h_1, \pi (a) h_2\rangle  = \langle T h_1, T h_2\rangle  = \langle T^*T h_1, h_2\rangle  = \langle h_1, h_2\rangle
    \end{align}
    Or maybe this \\
    If $_A H$ than $(a_1a_2) h = a_1 (a_2 h)$ for $a_1, a_2 \in A$ and $h \in H$.\\
    Then we take the representation of an $a \in A$, $\pi (a)$:
    \begin{align}
        (\pi(a_1)\pi(a_2))h = \pi(a_1)(\pi(a_2) h) = (T_1T_2) h = T_1 (T_2 h)
    \end{align}
    For $T_1, T_2 \in L(H)$, which operate naturally from the left on $h$.
\end{MyExercise}

\begin{MyExercise}
    \textbf{
    Show that $A$ is a bimodule ${}_A A_A$ with itself.
}\newline

    $\gamma: A\times A\times A \rightarrow A$ which is given by the inner product of the $*$-algebra.
\end{MyExercise}

%\subsubsection{Balanced Tensor Product and Hilbert Bimodules}
%
%\begin{definition}
%    Let $A$ be an algebra, $E$ be a \textit{right} $A$-module and $F$ be a \textit{left} $A$-module.
%    The \textit{balanced tensor product} of $E$ and $F$ forms a $A$-bimodule.
%    \begin{align}
%        E \otimes _A F := E \otimes F / \left\{\sum _i e_i a_i \otimes f_i - e_i \otimes a_i f_i : \;\;\;
%                                         a_i \in A,\ e_i \in E,\ f_i \in F \right\}
%    \end{align}
%\end{definition}
%Note $/$ denotes the quotient space. So $\otimes _A$ takes two left/right modules and makes a
%bimodule with the help the tensor product of the two modules and the quotient space that takes
%out all the elements from the tensor product that dont preserver the left/right representation and that
%are duplicates.
%\begin{definition}
%    Let $A$, $B$ be \textit{matrix algebras}. The \textit{Hilbert bimodule} for $(A, B)$ is given by
%    \begin{itemize}
%        \item $E$, an $A$-$B$-bimodue $E$ and by
%        \item an $B$-valued \textit{inner product} $\langle \cdot,\cdot\rangle_E: E\times E \rightarrow B$
%    \end{itemize}
%$\langle \cdot,\cdot\rangle_E$ needs to satisfy the following for $e, e_1, e_2 \in E,\ a \in A$ and $b \in B$.
%\begin{align}
%    \langle e_1, a\cdot e_2\rangle_E &= \langle a^*\cdot e_1, e_2\rangle_E \;\;\;\; & \text{sesquilinear in $A$}\\
%    \langle e_1, e_2 \cdot b\rangle_E &= \langle e_1, e_2\rangle_E b \;\;\;\; & \text{scalar in $B$} \\
%    \langle e_1, e_2\rangle_E &= \langle e_2,e_1\rangle^*_E \;\;\;\; & \text{hermitian} \\
%    \langle e, e\rangle_E &\ge 0 \;\;\;\; & \text{equality holds iff $e=0$}
%\end{align}
%
%\end{definition}
%
%We denote $KK_f(A,B)$ the set of all \textit{Hilbert bimodules} of $(A,B)$.
%
%%\begin{MyExercise}
%%    \textbf{
%%    Check that a representation $\pi:\ A \ \rightarrow L(H)$ of a matrix algebra $A$ turns $H$ into
%%    a Hilbert bimodule for $(A, \mathbb{C})$.
%%    \label{ex: bimodule}
%%}\newline
%%
%%
%%    We check if the representation of $a \in A$, $\pi(a)=T \in L(H)$ fulfills
%%    the conditions on the $\mathbb{C}$-valued inner product for $h_1, h_2 \in H$:
%%    \begin{itemize}
%%        \item $\langle h_1, \pi(a) h)2\rangle _\mathbb{C} = \langle h_1, T h_2\rangle _\mathbb{C} =
%%            \langle T^* h_1, h_2\rangle _\mathbb{C}$, $T^*$ given by the adjoint
%%        \item $\langle h_1, h_2 \pi(a)\rangle _\mathbb{C} = \langle h_1, h_2 T\rangle _\mathbb{C} = \langle h_1, h_2\rangle _\mathbb{C}$, $T$ acts from the left
%%        \item $\langle h_1, h_2\rangle _\mathbb{C}^* = \langle h_2,h_1\rangle _\mathbb{C}$, hermitian because of the
%%            $\mathbb{C}$-valued inner product
%%        \item $\langle h_1, h_2\rangle  \ge 0$, $\mathbb{C}$-valued inner product.
%%    \end{itemize}
%%\end{MyExercise}
%
%%\begin{MyExercise}
%%    \textbf{
%%    Show that the $A-A$ bimodule given by $A$ is in $KK_f(A,A)$ by taking the following inner product
%%    $\langle \cdot,\cdot\rangle_A:A \times A \rightarrow A$:
%%    \begin{align}
%%        \langle a, a\rangle_A = a^*a' \;\;\;\; a,a'\in A
%%    \end{align}
%%    \label{exercise: inner-product}
%%}\newline
%%
%%
%%    We check again the conditions on $\langle \cdot, \cdot\rangle _A$, let $a, a_1, a_2 \in A$:
%%    \begin{itemize}
%%        \item $\langle a_1, a\cdot a_2\rangle _A = a^*\ a\cdot a_2 = (a^*a_1)^* a_2 = \langle  a^*a_1, a_2\rangle  $
%%        \item $\langle a_1, a_2 \cdot a\rangle _A = a^*_1 (a_2\cdot a) = (a^*a_2)\cdot a = \langle a_1, a_2\rangle _A a$
%%        \item $\langle a_1, a_2\rangle _A^* = (a_1^* a_2)^* = a_2^*(a_1^*)^* = a_2^* a_1 = \langle a_2, a_1\rangle $
%%    \end{itemize}
%%\end{MyExercise}
%
%\begin{example}
%    Consider a $*$ homomorphism between two matrix algebras $\phi:A\rightarrow B$.
%    From it we can construct a Hilbert bimodule $E_{\phi} \in KK_f(A, B)$ in the following way.
%    We let $E_{\phi}$ be $B$ in the vector space sense and an inner product from the above
%    Exercise \ref{exercise: inner-product}, with $A$ acting on the left with $\phi$.
%    \begin{align}
%        a\cdot b = \phi(a)b \;\;\;\; a\in A, b\in E_{\phi}
%    \end{align}
%\end{example}
%
%
%\subsubsection{Kasparov Product and Morita Equivalence}
%\begin{definition}
%    Let $E \in KK_f(A, B)$ and $F \in KK_F(B, D)$ the \textit{Kasparov product} is defined as
%    with the balanced tensor product
%    \begin{align}
%        F \circ E := E \otimes _B F
%    \end{align}
%    Such that $F\circ E \in KK_f(A,D)$ with a $D$-valued inner product.
%    \begin{align}
%        \langle e_1 \otimes f_1, e_2 \otimes f_2\rangle _{E\otimes _B F} = \langle f_1,\langle e_1, e_2\rangle _E f_2\rangle _F
%    \end{align}
%\end{definition}
%
%\begin{question}
% How do we go from $\langle e_1 \otimes f_1, e_2 \otimes f_2\rangle _{E\otimes _B F}$ to $
%    \langle f_1,\langle e_1, e_2\rangle _E f_2\rangle _F$ \label{q: tensorproduct}\\
%    This statement is still in the definition.
%\end{question}
%
%%\begin{question}
%%What is the meaning of `associative up to isomorphism'? Isomorphism of $F \circ E$ or of $A, B$ or $D$?
%%\end{question}
%
%%    \begin{MyExercise}
%%        \textbf{
%%    Show that the association $\phi \leadsto E_\phi$ (from the previous Example) is natural
%%    in the sense
%%    \begin{enumerate}
%%        \item $E_{\text{id}_A} \simeq A \in KK_f(A,A)$
%%        \item for $*$-algebra homomorphism $\phi: A \rightarrow B$ and $\psi: B \rightarrow C$ we have
%%            an isomorphism
%%            \begin{align}
%%                E_{\psi} \circ E_{\phi}\ \equiv\ E_{\phi} \otimes _B E_{\psi}\ \simeq\
%%                E_{\psi \circ \phi} \in KK_f(A,C)
%%            \end{align}
%%    \end{enumerate}
%%}
%%    \begin{enumerate}
%%        \item $\text{id}_A: A \rightarrow A$.\\
%%            To construct $E_{\phi}\in KK_f(A,A)$, we let $E_{\phi}$ be $A$ with a natural right
%%            representation, so $\Rightarrow E_{\phi}\simeq A$.\\
%%            With an inner product, acting on $A$ from the left with $\phi$, $a', a\in A$\\
%%            $a'a = (\phi(a') a) \in A $, which is satisfied by $\text{id}_A$, so $\phi = \text{id}_A$.
%%        \item $a \cdot b \cdot c = \psi(\phi (a) \cdot b) \cdot c$ for $a \in A$, $b\in B$, and $c\in C$
%%                which is $\psi \circ \phi$
%%    \end{enumerate}
%%\end{MyExercise}
%
%%\begin{MyExercise}
%%    \textbf{
%%    In the definition of Morita equivalence:
%%    \begin{enumerate}
%%        \item Check that $E \otimes _B F$ is a $A-D$ bimodule
%%        \item Check that $\langle \cdot,\cdot\rangle _{E\oplus _B F}$ defines a $D$ valued inner product
%%        \item Check that $\langle a^*(e_1 \otimes f_1), e_2 \otimes f_2\rangle _{E \otimes _B F} = \langle e_1 \otimes f_1, a(e_2 \otimes f_2)\rangle _{E \otimes _B F}$.
%%    \end{enumerate}
%%}
%%    \begin{enumerate}
%%        \item $E \otimes _B F = E \otimes F / \{\sum_i e_i b_i \otimes f_i - e_i \otimes b_i f_i;
%%            e_i \in E_i, b_i \in B, f_i \in F\}$ the last part takes out all tensor product elements of
%%            $E$ and $F$ that don't preserver the left/right representation and that are duplicates.
%%        \item $\langle e_1, e_2\rangle _E \in B$ and $\langle f_1, f_2\rangle _F \in C$ by definition. So let $\langle e_1, e_2\rangle _E =b$. \\
%%            Then $\langle e_1 \otimes f_1, e_2 \otimes f_2\rangle _{E\otimes _B F} = \langle f_1, \langle e_1, e_2\rangle _E f_2\rangle _F =
%%            \langle f_1, b f_2\rangle _F \in C$
%%        \item Check Question \ref{q: tensorproduct}.\\
%%            But let $G := E\otimes _B F \in KK_f(A,C)$ then $\forall g_1, g_2 \in G$ and $a \in A$ we need
%%            by definition $\langle g_1, ag_2\rangle _G = \langle a^*g_1, g_2\rangle _G$ and we set $g_1 = e_1 \otimes f_1$ and
%%            $g_2 = e_2 \otimes f_2$ for some $e_1, e_2 \in E$ and $f_1, f_2 \in F$, or else
%%            $G \notin KK_f(A,C)$ which would violate the Kasparov product
%%    \end{enumerate}
%%    \end{MyExercise}
%
%\begin{definition}
%    Let $A$, $B$ be \textit{matrix algebras}. They are called \textit{Morita equivalent} if there
%    exists an $E \in KK_f(A, B)$ and an $F \in KK_f(B, A)$ such that:
%    \begin{align}
%        E \otimes _B F \simeq A \;\;\; \text{and} \;\;\; F \otimes _A E \simeq B
%    \end{align}
%    Where $\simeq$ denotes the isomorphism between Hilbert bimodules, note that $A$ or $B$ is a bimodule by
%    itself.
%\end{definition}
%
%\begin{question}
%    Why are $E$ and $F$ each others inverse in the Kasparov Product? \\
%    They are each others inverse with respect to the Kasparov Product because we land in the same space as we started.
%    In the definition we have $E \in KK_f(A, B)$ we start from $A$ and $E \otimes _B F$ lands in $A$.\\
%    On the other hand we have $F \in KK_f(B, D)$ we start from $B$ and $F \otimes _A E$ lands in $B$.
%\end{question}
%
%\begin{example}
%    \
%    \begin{itemize}
%        \item Hilber bimodule of $(A,A)$ is $A$
%        \item Let $E \in KK_f(A,B)$, we take $E \circ A = A\oplus _A E \simeq E$
%        \item we conclude, that $_A A_A$ is the identity in the Kasparov product (up to isomorphism)
%    \end{itemize}
%\end{example}
%
%\begin{example}
%    Let $E = \mathbb{C}^n$, which is a $(M_n(\mathbb{C}), \mathbb{C})$ Hilbert bimodule with the
%    standard $\mathbb{C}$ inner product.\\
%    On the other hand let $F = \mathbb{C}^n$, which is a $(\mathbb{C}, M_n(\mathbb{C}))$ Hilbert
%    bimodule by right matrix multiplication with $M_n(\mathbb{C})$ valued inner product:
%    \begin{align}
%        \langle v_1, v_2\rangle =\bar{v_1}v_2^t \;\; \in M_n(\mathbb{C})
%    \end{align}
%    Now we take the Kasparov product of $E$ and $F$:
%   \begin{itemize}
%        \item $F\circ E\ =\  E\otimes _{\mathbb{C}}F\ \;\;\;\;\;\; \simeq \  M_n(\mathbb{C})$
%        \item $E\circ F\ =\ F\otimes _{M_n(\mathbb{C})}E\ \simeq\ \mathbb{C}$
%    \end{itemize}
%    $M_n(\mathbb{C})$ and $\mathbb{C}$ are Morita equivalent
%\end{example}
%
%\begin{theorem}
%    Two matrix algebras are Morita Equivalent iff their their Structure spaces
%    are isomorphic as discreet spaces (have the same cardinality / same number of elements)
%\end{theorem}
%\begin{proof}
%    Let $A$, $B$ be \textit{Morita equivalent}. So there exists $_A E_B$ and $_B F_A$ with
%    \begin{align}
%        E \otimes _B F \simeq A \;\;\; \text{and} \;\;\; F \otimes _A E \simeq B
%    \end{align}
%    Consider $[(\pi _B, H)] \in \hat{B}$ than we construct a representation of $A$,
%    \begin{align}
%        \pi _A \rightarrow L(E \otimes _B H)\;\;\; \text{with} \;\;\; \pi _A(a) (e \otimes v) = a e \otimes w
%    \end{align}
%    \begin{question}
%        Is $E \simeq H$ and $F \simeq W$? \\
%        Not in particular, there is a theorem that all infinite dimensional Hilbert spaces are isomorphic.
%        Here we are looking at finite dimensional Hilbert spaces.\\
%        Another thing to is that $[\pi _B, H] \in \hat{B}$ and looking at Exercise \ref{ex: bimodule}
%        we know that $H$ is a bimodule of $B$, hence $E \otimes _B H\simeq A$, and for $[\pi _A, W]$
%        the same.
%    \end{question}
%    \textit{vice versa}, consider $[(\pi _A, W)] \in \hat{A}$ we can construct $\pi _B$
%    \begin{align}
%        \pi _B: B \rightarrow L(F \otimes _A W) \;\;\; \text{and}\;\;\; \pi _B(b) (f\otimes w) = bf\otimes w
%    \end{align}
%    These maps are each others inverses, thus $\hat{A} \simeq \hat{B}$
%\end{proof}
%
%%\begin{MyExercise}
%%    \textbf{
%%    Fill in the gaps in the above proof:
%%    \begin{enumerate}
%%        \item show that the representation of $\pi _A$ defined is irreducible iff $\pi _B$ is.
%%        \item Show that the association of the class $[\pi _A]$ to $[\pi _B]$ is independent
%%            of the choice of representatives $\pi _A$ and $\pi _B$
%%    \end{enumerate}
%%}
%%
%%    \begin{enumerate}
%%        \item $(\pi _B, H)$ is irreducible means $H \neq \emptyset$ and only $\emptyset$ or $H$
%%            is invariant under the Action of $B$ on $H$.
%%            Than $E\otimes _B H$ cannot be empty, because also $E$ preserves left representation of $A$
%%            and also $E\otimes _B H \simeq A$.
%%        \item The important thing is that $[\pi _A] \in \hat{A}$ respectively $[\pi _B] \in \hat{B}$,
%%            hence any choice of representation is irreducible, because the structure space denotes all unitary
%%            equivalence classes of irreducible representations.
%%    \end{enumerate}
%%\end{MyExercise}
%
%    \begin{lemma}
%    The matrix algebra $M_n(\mathbb{C})$ has a unique irreducible representation (up to isomorphism)
%    given by the defining representation on $\mathbb{C}^n$.
%\end{lemma}
%\begin{proof}
%    We know $\mathbb{C}^n$ is a irreducible representation of $A= M_n(\mathbb{C})$. Let $H$ be irreducible
%    and of dimension $k$, then we define a map
%    \begin{align}
%        \phi : A\oplus...\oplus A &\rightarrow H^* \\
%        (a_1,...,a_k)             &\mapsto e^1\circ a_1^t+...+e^k\circ a_k^t
%    \end{align}
%    With $\{e^1,...,e^k\}$ being the basis of the dual space $H^*$ and $(\circ)$ being the pre-composition
%    of elements in $H^*$ and $A$ acting on $H$. This forms a morphism of $M_n(\mathbb{C})$ modules,
%    provided a matrix $a \in A$ acts on $H^*$ with $v\mapsto v\circ a^t$ ($v\in H^*$).
%    Furthermore this morphism is surjective, thus making the pullback $\phi ^*:H\mapsto (A^k)^*$ injective.
%    Now identify $(A^k)^*$ with $A^k$ as a $A$-module and note that
%    $A=M_n(\mathbb{C}) \simeq \oplus ^n \mathbb{C}^n$ as a n A module.
%    It follows that $H$ is a submodule of $A^k \simeq \oplus ^{nk}\mathbb{C}$. By irreducibility
%    $H \simeq \mathbb{C}$.
%\end{proof}
%
%\begin{example}
%    Consider two matrix algebras $A$, and $B$.
%    \begin{align}
%        A = \bigoplus ^N_{i=1} M_{n_i}(\mathbb{C}) \;\;\; B = \bigoplus ^M_{j=1} M_{m_j}(\mathbb{C})
%    \end{align}
%    Let $\hat{A} \simeq \hat{B}$ that implies $N=M$ and define $E$ with $A$ acting by block-diagonal
%    matrices on the first tensor and B acting in the same way on the second tensor. Define $F$ vice versa.
%    \begin{align}
%        E:= \bigoplus _{i=1}^N \mathbb{C}^{n_i} \otimes \mathbb{C}^{m_i} \;\;\;
%        F:= \bigoplus _{i=1}^N \mathbb{C}^{m_i} \otimes \mathbb{C}^{n_i}
%    \end{align}
%    Then we calculate the Kasparov product.
%    \begin{align}
%        E \otimes _B F &\simeq \bigoplus _{i=1}^N (\mathbb{C}^{n_i}\otimes\mathbb{C}^{m_i})
%            \otimes _{M_{m_i}(\mathbb{C})} (\mathbb{C}^{m_i}\otimes\mathbb{C}^{n_i}) \\
%                       &\simeq \bigoplus _{i=1}^N \mathbb{C}^{n_i}\otimes
%                       \left(\mathbb{C}^{m_i}\otimes _{M_{m_i}(\mathbb{C})}\mathbb{C}^{m_i}\right)
%                        \oplus \mathbb{C}^{n_i} \\
%                       &\simeq \bigoplus _{i=1}^N \mathbb{C}^{m_i}\otimes\mathbb{C}^{n_i} \simeq A
%    \end{align}
%    and from $F \otimes _A E \simeq B$.
%\end{example}
%
%We conclude that.
%\begin{itemize}
%    \item There is a duality between finite spaces and Morita equivalence classes of matrix algebras.
%    \item By replacing $*$-homomorphism $A\rightarrow B$ with Hilbert bimodules $(A,B)$ we introduce
%        a richer structure of morphism between matrix algebras.
%\end{itemize}
%
%\subsection{Noncommutative Geometric Spaces }
%\subsubsection{Exercises}
%
%%\begin{MyExercise}
%%\textbf{
%%    Make the proof of the last theorem (see week4.pdf) explicit for $N=3$.
%%}\newline
%%
%%    For the C* algebra we have $A=\mathbb{C}^3$
%%    For $H$ we have $H = (\mathbb{C}^2)^{\oplus 3} = H_2 \oplus H_2^1 \oplus H_2^2$.
%%    The symmetric operator $D$ acting on $H$ and the representation $\pi (a)$:
%%    \begin{align}
%%        \pi((a(1), a(2), a(3)) &=
%%        \begin{pmatrix}
%%            a(1) & 0 \\ 0 & a(2)
%%        \end{pmatrix} \oplus
%%        \begin{pmatrix}
%%            a(1) & 0 \\ 0 & a(3)
%%        \end{pmatrix} \oplus
%%        \begin{pmatrix}
%%            a(2) & 0 \\ 0 & a(2)
%%        \end{pmatrix} \nonumber  \\
%%        & =
%%        \begin{pmatrix}
%%            a(1) & 0 & 0 & 0 & 0 & 0 \\
%%            0    & a(2) & 0 & 0 & 0 & 0 \\
%%            0    & 0 & a(1) & 0 & 0 & 0 \\
%%            0    & 0 & 0 & a(3) & 0 & 0 \\
%%            0    & 0 & 0 & 0 & a(2) & 0 \\
%%            0    & 0 & 0 & 0 & 0 & a(3)
%%        \end{pmatrix} \\
%%        D &=
%%        \begin{pmatrix}
%%            0 & x_1 \\ x_1 & 0
%%        \end{pmatrix} \oplus
%%        \begin{pmatrix}
%%            0 & x_2 \\ x_2 & 0
%%        \end{pmatrix} \oplus
%%        \begin{pmatrix}
%%            0 & x_3 \\ x_3 & 0
%%        \end{pmatrix} \nonumber \\
%%        &=
%%        \begin{pmatrix}
%%            0   & x_1 & 0 & 0 & 0 & 0 \\
%%            x_1 & 0   & 0 & 0 & 0 & 0 \\
%%            0   & 0   & 0 & x_2 & 0 & 0 \\
%%            0   & 0   & x_2 & 0 & 0 & 0 \\
%%            0   & 0   & 0 & 0 & 0 & x_3 \\
%%            0   & 0   & 0 & 0 & x_3 & 0 \\
%%        \end{pmatrix} \\
%%    \end{align}
%%    Then the norm of the commutator would be the largest eigenvalue
%%    \begin{align}
%%        &||[D, \pi(a)]|| = ||D\pi(a) - \pi(a)D||\nonumber\\
%%        &=
%%        \left|\left|
%%    \setlength{\arraycolsep}{0.1cm}
%%      \renewcommand{\arraystretch}{0.1}
%%        \begin{pmatrix}
%%            0   & x_1(a(2)-a(1)) & 0 & 0 & 0 & 0 \\
%%            -x_1(a(2)-a(1)) & 0   & 0 & 0 & 0 & 0 \\
%%            0   & 0   & 0 & x_2(a(3)-a(1)) & 0 & 0 \\
%%            0   & 0   & -x_2(a(3)-a(1)) & 0 & 0 & 0 \\
%%            0   & 0   & 0 & 0 & 0 & x_3(a(3)-a(2)) \\
%%            0   & 0   & 0 & 0 & -x_3(a(2)-a(3)) & 0 \\
%%        \end{pmatrix}\right|\right| \label{skew matrix}
%%    \end{align}
%%The matrix in Equation \ref{shew matrix} is a skew symmetric matrix its eigenvalues
%%are $i\lambda_1, i\lambda_2, i\lambda_3, i\lambda_4$, where the $\lambda$'s are on the
%%upper and lower diagonal check \url{https://en.wikipedia.org/wiki/Skew-symmetric_
%%matrix#Skew-symmetrizable_matrix}. The matrix norm of would be the maximum of the norm of
%%the larges eigenvalues:
%%\begin{align}
%%    ||[D, \pi(a)]|| = \max_{a\in A}\{&x_1|a(2)-a(1)|,\\
%%    &x_2|(a(3)-a(1))|,\nonumber\\
%%    &x_3|(a(3)-a(2))|,\}\nonumber
%%\end{align}
%%The metric is then:
%%\begin{align}
%%    d =
%%    \begin{pmatrix}
%%        0 & a(1)-a(2) & a(1)-a(3)\\
%%        a(2)-a(1) & 0 & a(2)-a(3)\\
%%        a(3)-a(1) & a(3)-a(2) & 0
%%    \end{pmatrix}
%%\end{align}
%%\end{MyExercise}
%
%%\begin{MyExercise}
%%    \textbf{
%%	Compute the metric on the space of three points given by $d_{ij} =
%%	\sup_{a\in A}\{|a(i) - a(j)|: ||[D, \pi(a)]|| \leq 1\}$ for the set of data
%%    $A = \mathbb{C}^3$ acting in the defining representation $H = \mathbb{C}^3$, and
%%    \begin{align}
%%    D =
%%    \begin{pmatrix}
%%        0 & d^{-1} & 0 \\
%%        d^{-1} & 0 & 0 \\
%%        0 & 0 & 0
%%    \end{pmatrix}
%%    \end{align}
%%    for some $d \in \mathbb{R}$
%%}\newline
%%
%%    We have $A=\mathbb{C}^3$, $H=\mathbb{C}^3$ and $D$ from above, then
%%
%%    \begin{align}
%%        ||[D, \pi(a)]|| &= d^{-1}\left|\left|
%%    \begin{pmatrix}
%%        0 & a(2)-a(1) & 0 \\
%%        -(a(2)-a(1)) & 0 & 0 \\
%%        0 & 0 & 0
%%    \end{pmatrix} \right|\right|
%%    \end{align}
%%    The metric is then
%%    \begin{align}
%%    d =
%%        \begin{pmatrix}
%%            0 & a(1)-a(2) & a(1)  \\
%%            a(2)-a(1) & 0 & a(2) \\
%%            -a(1) & -a(2) & 0
%%        \end{pmatrix}
%%    \end{align}
%%\end{MyExercise}
%
%%\begin{MyExercise}
%%    \textbf{
%%    Show that $d_{ij}$ from Equation \ref{ext metric} is a metric on $\hat{A}$ by
%%    establishing that:
%%    \begin{align}
%%        d_{ij} &= 0\;\;\; \Leftrightarrow \;\;\; i=j \label{metric 1} \\
%%        d_{ij} &= d_{ji} \label{metric 2}\\
%%        d_{ij} &\leq d_{ik} + d_{kj} \label{metric 3}
%%    \end{align}
%%    \begin{equation} \label{ext metric}
%%        d_{ij} = \sup_{a\in A}\big\{|\text{Tr}(a(i)) - \text{Tr}((a(j))|: ||[D, a]|| \leq 1\big\}
%%    \end{equation}
%%}\newline
%%
%%For Equation \ref{metric 1} set $i=j$ in \ref{ext metric}.
%%\begin{align}
%%    d_{ii} &= \sup_{a \in A}\{|\text{Tr}(a(i)) - \text{Tr}((a(i))|: ||[D, a]|| \leq
%%    1\big\} \\
%%    &= \sup_{a \in A}\{0: ||[D, a]|| \leq 1\big\} = 0
%%\end{align}
%%For Equation \ref{metric 2} obviously we have the commuting property of
%%addition.
%%\newline
%%For Equation \ref{metric 3}, for $k=j$ then $d_{kj} = 0$ and the equality
%%holds. For $i = k$ then $d_{ik} = 0$ and equality holds. Else set $d_{ik} =
%%1$ and $d_{kj} = 1$ then $d_{ij} = 1 \leq d_{ik} + d_{kj} = 2$
%%\end{MyExercise}
%
%\subsubsection{Properties of Matrix Algebras}
%\begin{lemma}
%    If $A$ is a unital C* algebra that acts faithfully on a finite
%    dimensional Hilbert space, then $A$ is a matrix algebra of the Form:
%    \begin{equation}
%        A \simeq \bigoplus _{i=1}^N M_{n_i}(\mathbb{C})
%    \end{equation}
%\end{lemma}
%\begin{proof}
%    Since $A$ acts faithfully on a Hilbert space, then $A$ is a C*
%    subalgebra of a matrix algebra $L(H) = M_{\dim (H)}(\mathbb{C}
%    \Rightarrow A \simeq \text{Matrix algebra}$.
%\end{proof}
%
%\begin{question}
%    What does the author mean when he sais 'acts faithfully on a
%    Hilbertspace`? Then the representation is fully reducible, or that the
%    presentation is irreducible?
%    \newline
%
%    For a *-representation 'faithful` if it is injective. For a
%    *-homomorphism 'faithful` means one-to-one correspondance
%\end{question}
%
%\begin{example}
%    $A = M_n(\mathbb{C})$ and $H=\mathbb{C}^n$, $A$ acts on $H$ with matrix
%    multiplication and standard inner product. $D$ on $H$ is a hermitian
%    matrix $n\times n$ matrix.
%\end{example}
%
%$D$ is referred to as a finite Dirac operator as in as its $\infty$
%dimensional on Riemannian Spin manifolds coming in Chapter 4.
%\newline
%
%Now can introduce a 'differential 'geometric structure` on the finite space X
%with the \textbf{devided difference}
%\begin{equation}
%    \frac{a(i)-a(j)}{d_{ij}}
%\end{equation}
%for each pair $i$, $j$ $\in X$ the finite dimensional discrete space $X$.
%This appears in the entries in the commutator $[D, a]$ in the above
%exercises.
%
%\begin{definition}
%    Given an finite spectral triple $(A, H, D)$, the $A$-bimodule of
%    Connes' differential one-forms is:
%    \begin{equation}
%        \Omega _D ^1 (A) := \left\{ \sum _k a_k[D, b_k]: a_k, b_k \in A \right\}
%    \end{equation}
%\end{definition}
%
%\begin{question}
%    Is the Conne's differential one form  the set of all '1st order
%    differential operators` given $A$, that act on $H$?
%\end{question}
%Then there is a map $d:A\rightarrow \Omega _D ^1 (A)$, $d = [D, \cdot]$.
%%\begin{MyExercise}
%%    \textbf{
%%    Verify that 'd` is a derivation of the C* algebra
%%    \begin{align}
%%        d(ab) = d(a)b + ad(b) \\
%%        d(a^*) = -d(a)^*
%%    \end{align}
%%}\newline
%%
%%    For the record $d(\cdot) = [D, \cdot]$, then we have
%%    \begin{enumerate}
%%        \item
%%            \begin{align}
%%                d(ab) &= [D, ab] = [D, a]b + a[D,b]\\
%%                &= d(a)b + ad(b)
%%            \end{align}
%%        \item
%%            \begin{align}
%%                d(a^*) &= [D, a^*] = Da^* - a^*D \\
%%                &=-(D^*a - aD^*) = -[D^*, a] \\
%%                &= -d(a)^*
%%            \end{align}
%%    \end{enumerate}
%%\end{MyExercise}
%%\begin{MyExercise}
%%    \textbf{
%%    Verify that $\Omega _D^1 (A)$ is an $A$-bimodule by rewriting
%%    }
%%    \begin{align}
%%        a(a_k[D, b_k])b = \sum_k a'_k[D, b'_k] \;\;\;\; a'_k, b'_k \in A
%%    \end{align}
%%    \newline
%%
%%    Begin
%%    \begin{align}
%%        a(a_k[D, b_k])b &= aa_k(Db_k - b_k D) b = \\
%%           &= aa_k(Db_k b - b_k D b) = aa_k(Db_k b - b_k Db - b_kbD +b_kbD)=\\
%%           &= aa_k(Db_kb-b_kbD + b_k b D - b_k D b) = \\
%%           &= aa_k [D, b_kb] + aa_k b [D, b]=\\
%%           &= \sum _k a_k' [D, b_k']
%%    \end{align}
%%
%%\end{MyExercise}
%
%\begin{lemma}
%    Let $(A, H, D) = (M_n(\mathbb{C}, \mathbb{C}^n, D)$, with $D$ a hermitian
%    $n\times n$ matrix. If $D$ is not a multiple of the identity then:
%    \begin{equation}
%        \Omega _D ^1 (A)  \simeq  M_n(\mathbb{C}) = A
%    \end{equation}
%\end{lemma}
%
%\begin{proof}
%    Assume $D = \sum _i \lambda _i e_{ii}$ (diagonal), $\lambda _i \in \mathbb{R}$ and
%    $\{e_{ij}\}$ the basis of $M_n(\mathbb{C}$. For fixed $i$, $j$ choose $k$
%    such that $\lambda _k \neq \lambda _j$ then
%    \begin{align} \label{basis}
%        \left(\frac{1}{\lambda _k - \lambda _j} e_{ik}\right) [D, e_{kj}] =
%        e_{ij}
%    \end{align}
%    $e_{ij}\in \Omega _D ^1 (A)$ by the above definition. And $\Omega _D ^1
%    (A) \subset L(\mathbb{C}^n) = H \simeq M_n(\mathbb{C}) = A$
%\end{proof}
%
%%\begin{MyExercise}
%%    \textbf{
%%     Consider $(A=\mathbb{C}^2, H=\mathbb{C}^2,
%%     D = \begin{pmatrix} 0 & \lambda \\ \bar{\lambda} & 0
%%     \end{pmatrix})$ with $\lambda \neq 0$. Show that $\Omega _D^1(A)
%%     \simeq M_2(\mathbb{C})$
%% }
%%\newline
%%
%%    Because of the Hilbert Basis $D$ can be extended in terms of
%%    the basis of $M_2(\mathbb{C})$, plugging this into Equation
%%    \ref{basis} will get us the same cyclic result, thus
%%    $\Omega _D^1(A) \simeq M_2(\mathbb{C})$
%%\
%%\end{MyExercise}
%
%\subsubsection{Morphisms Between Finite Spectral Triples}
%\begin{definition}
%    two finite spectral tripes $(A_1, H_1, D_1)$ and $(A_2, H_2, D_2)$ are
%    called unitarily equivalent if
%    \begin{itemize}
%        \item $A_1 = A_2$
%        \item $\exists \;\; U: H_1 \rightarrow H_2$, unitary with
%            \begin{enumerate}
%                \item  $U\pi_1(a)U^* = \pi_2(a)$ with $a \in A_1$
%                \item  $UD_1 U^* = D_2$
%            \end{enumerate}
%    \end{itemize}
%\end{definition}
%
%Some remarks
%\begin{itemize}
%    \item the above is an equivalence relation
%    \item spectral unitary equivalence is given by the unitaries of the
%        matrix algebra itself
%    \item for any such $U$ then $(A, H, D) \sim (A, H, UDU^*)$
%    \item $UDU^* = D + U[D, U^*]$ of the form of elements in
%        $\Omega _D^1 (A)$.
%\end{itemize}
%
%%\begin{MyExercise}
%%    \textbf{
%%    Show that the unitary equivalence between finite spectral
%%    triples is a equivalence relation
%%}\newline
%%
%%    An equivalence relation needs to satisfy reflexivity, symmetry
%%    transitivity.
%%    Let $(A_1, H_1, D_1)$, $(A_2, H_2, D_2)$ and $(A_3, H_3, D_3)$
%%    be three finite spectral triples.
%%    \newline
%%
%%    For reflexivity $(A_1, H_1, D_1) \sim (A_1, H_1, D_1)$. So there
%%    exists a $U: H_1 \rightarrow H_1$ unitary, which is the identity
%%    and always exists.
%%    \newline
%%
%%    For symmetry we need
%%    \begin{align}
%%        (A_1, H_1, D_1) \sim (A_2, H_2, D_2) \Leftrightarrow
%%        (A_2, H_2, D_2) \sim (A_1, H_1, D_1)
%%    \end{align}
%%    because $U$ is unitary:
%%    \begin{align}
%%        &U\pi_1(a)U^* = \pi_2(a) \;\;\; | \cdot U^*\boxdot U \\
%%        &U^*U\pi_1(a)U^*U = \pi_1(a) = U^*\pi_2(a)U \\
%%    \end{align}
%%    The same with the symmetric operator $D$.
%%    \newline
%%
%%    For transitivity we need
%%    \begin{align}
%%        (A_1, H_1, D_1) &\sim (A_2, H_2, D_2) \;\;\; \text{and} \;\;\;
%%        (A_2, H_2, D_2) \sim (A_3, H_3, D_3) \\
%%        &\Rightarrow (A_1, H_1, D_1) \sim (A_3, H_3, D_3)
%%    \end{align}
%%    There are two unitary maps $U_{12}:H_1 \rightarrow H_2$ and
%%    $U_{23}: H_2 \rightarrow H_3$ then
%%    \begin{align}
%%        U_{23}U_{12} \pi_1(a) U^*_{12}U^*_{23} &= U_{23}
%%        \pi_2(a) U_23^* \\
%%        &= \pi_3(a) \\
%%        U_{23}U_{12} D_1U^*_{12}U^*_{23} &= U_{23}
%%        D_2 U_23^* \\
%%        &= D_3
%%    \end{align}
%%\end{MyExercise}
%
%Extending the this relation we look again at  the notion of equivalence from
%Morita equivalence of Matrix Algebras.
%\newline
%
%\begin{definition}
%    Let $A$ be an algebra. We say that $I \subset A$, as a vector space, is a
%    right(left) ideal if $ab \in I$ for $a \in A$ and $b\in I$ (or $ba \in
%    I$, $b\in I$, $a\in A$). We call a left-right ideal simply an ideal.
%\end{definition}
%
%Given a Hilbert bimodule $E \in KK_f(B, A)$ and $(A, H, D)$ we construct
%a finite spectral triple on $B$, $(B, H', D')$
%\begin{equation}
%    H' = E \otimes _A H
%\end{equation}
%We might define $D'$ with $D'(e \otimes \xi) = e\otimes D\xi$, thought this
%would not satisfy the ideal defining the balanced tensor product over $A$,
%which is generated by elements of the form
%\begin{align}
%    e a \otimes \xi - e\otimes a \xi ;\;\;\;\; e\in E, a\in A, \xi \in H
%\end{align}
%This inherits the left action on $B$ from $E$ and has a $\mathbb{C}$
%valued inner product space. $B$ also satisfies the ideal.
%\begin{equation}
%    D'(e\otimes \xi) = e \otimes D \xi + \nabla (e) \xi \;\;\;\; e\in
%    E, a\in A
%\end{equation}
%Where $\nabla$ is called the \textit{connection on the right A-module E}
%associated with the  derivation $d=[D, \cdot]$ and satisfying the
%\textit{Leibnitz Rule} which is
%\begin{equation}
%    \nabla(ae) = \nabla(e)a + e \otimes [D, a] \;\;\;\;\;  e\in E,\; a\in A
%\end{equation}
%Then $D'$ is well defined on $E \otimes _A H$:
%\begin{align}
%    D'(ea \otimes \xi - e \otimes a \xi) &=  D'(ea \otimes \xi) - D'(e
%    \otimes \xi) \\
%    &= ea\otimes D\xi + \nabla(ae) \xi - e \otimes D(a\xi ) - \nabla (e)a
%    \xi \\
%    &= 0.
%\end{align}
%With the information thus far we can prove the following theorem
%\begin{theorem}
%    If $(A, H, D)$ a finite spectral triple, $E \in KK_f(B, A)$.
%    Then $(V, E\otimes _A H, D')$ is a finite spectral triple, provided that
%    $\nabla$ satisfies the compatibility condition
%    \begin{equation}
%        \langle e_1, \nabla e_2 \rangle _E - \langle \nabla e_1, e_2
%        \rangle _E = d\langle e_1, e_2 \rangle _E \;\;\;\; e_1, e_2 \in E
%    \end{equation}
%\end{theorem}
%\begin{proof}
%    $E\otimes _A H$ was shown in the previous subsection (text before the
%    theorem). The only thing left is to show that $D'$ is a symmetric
%    operator, this we can just compute. Let $e_1, e_2 \in E$ and $\xi _1,
%    \xi _2 \in H$ then
%    \begin{align}
%        \langle e_1 \otimes \xi _1, D'(e_2 \otimes \xi_2)\rangle _{E\otimes _A H} &=
%        \langle \xi _1, \langle e_1, \nabla e_2\rangle _E  \xi _2\rangle  + \langle \xi _1 , \langle e_1, e_2\rangle _E D\xi
%        _2\rangle _H \\
%        &= \langle \xi _1, \langle \nabla e_1, e_2\rangle _E \xi _2\rangle _H + \langle \xi _1, d\langle e_1, e_2\rangle  _E
%        \xi _2\rangle _H \\
%        &+ \langle D\xi _1,\langle e_1, e_2\rangle _E \xi _2\rangle _H - \langle \xi _1, [D, \langle e_1, e_2\rangle _E] \xi
%        _2 \rangle _H \\
%        &= \langle D'(e_1 \otimes \xi _1), e_2 \otimes \xi _2\rangle _{E \otimes _A H}
%    \end{align}
%\end{proof}
%
%%\begin{MyExercise}
%%    \textbf{
%%    Let $\nabla$ and $\nabla'$ be two connections on a right $A$-module
%%    $E$. Show that $\nabla - \nabla'$ is a right $A$-linear map
%%    $E \rightarrow E\otimes _A \Omega _D^1(A)$
%%}\newline
%%
%%    Both $\nabla$ and $\nabla'$ need to satisfy the Leiblitz rule, so
%%    let's see if $\nabla - \nabla'$ does.
%%
%%    \begin{align}
%%        \nabla(ea)-\nabla'(ea)&=\nabla(e) + e\otimes[D, a]\\
%%        &-(\nabla'(e)a + e\otimes[D',a])\\
%%        &=\bar{\nabla}a + e\otimes(Da-aD-D'a+aD')\\
%%        &=\bar{\nabla}a + e\otimes((D-D')a-a(D-D'))\\
%%        &=\bar{\nabla}a + e\otimes[D', a]\\
%%        &=\bar{\nabla}(ea)
%%    \end{align}
%%    Therefore $\nabla-\nabla'$ is a linear map.
%%\end{MyExercise}
%
%%\begin{MyExercise}
%%    \textbf{
%%    Construct a finite spectral triple $(A, H', D')$ from $(A, H, D)$
%%    \begin{enumerate}
%%        \item show that the derivation $d(\cdot):A \rightarrow A\otimes _A
%%            \Omega_D^1(A)=\Omega_D^1(A)$ is a connection on $A$
%%            considered a right $A$-module
%%        \item Upon identifying $A\otimes_A H\simeq H$, what is $D'$
%%            when the connection is $d(\cdot)$.
%%        \item Use 1) and 2) to show that any connection $\nabla:
%%            A\rightarrow A\otimes_A \Omega_D^1(A)$ is given by
%%            \begin{align}
%%                \nabla = d + \omega
%%            \end{align}
%%            where $\omega \in \Omega_D^1(A)$
%%        \item Upon identifying $A\otimes_A H \simeq H$, what is the
%%            difference operator $D'$ with the connection on $A$ given by
%%            $\nabla = d + \omega$
%%    \end{enumerate}
%%}
%%    \begin{enumerate}
%%        \item $\nabla(e \cdot a) =  d(a)$
%%        \item
%%            $D'(a\xi) = a(D\xi) + (\nabla a) \xi = D(a\xi)$
%%        \item Use the identity element $e \in A$\\
%%                $\nabla (e\cdot a) = \nabla(e) a + 1 \otimes d(a)=d(a)
%%                \nabla(e) a$
%%            \item $D'(a\otimes \xi) = D'(a \xi) = a(D\xi) + (\nabla a)\xi =
%%                a(D\xi) + \nabla(e \cdot a) \xi \\
%%                = D(a\xi) + \nabla(e) (a\xi)$
%%    \end{enumerate}
%%\end{MyExercise}
%
%\subsubsection{Graphing Finite Spectral Triples}
%\begin{definition}
%    A \textit{graph} is a ordered pair $(\Gamma ^{(0)}, \Gamma ^{(1)})$.
%    Where $\Gamma ^{(0)}$ is the set of vertices (nodes) and $\Gamma ^{(1)}$
%    a set of pairs of vertices (edges)
%\end{definition}
%\begin{figure}[h!]
%    \centering
%\begin{tikzpicture}[
%        mass/.style = {draw,circle, minimum size=0.2cm, inner sep=0pt, thick},
%        spring/.style = {decorate,decoration={zigzag, pre length=1cm,post length=1cm,segment length=5pt}},]
%        \node[mass] (m1) at (1,1.5) {};
%        \node[mass] (m2) at (-1,1.5) {};
%        \node[mass] (m3) at (0,0) {};
%
%        \draw (m1) -- (m2);
%        \draw (m1) -- (m3);
%        \draw (m2) -- (m3);
%    \end{tikzpicture}
%    \caption{A simple graph with three vertices and three edges}
%\end{figure}
%%\begin{MyExercise}
%%    \textbf{
%%    Show that any finite-dimensional faithful representation $H$ of a matrix
%%    algebra $A$ is completely reducible. To do that show that the complement
%%    $W^{\perp}$ of an $A$-submodule $W\subset H$ is also an $A$-submodule
%%    of $H$.
%%}\newline
%%
%%    $A\simeq \bigoplus_{i=1}^N M_{n_i}(\mathbb{C})$ is the matrix algebra
%%    then $H$ is a Hilbert $A$-bimodule and $W$ a submodule of $A$.
%%    Because we have $H = W \cup W^{\perp}$, then $W^{\perp}$ is naturally a
%%    $A$-submodule, because elements in $W^{\perp}$ need to satisfy the
%%    bimodularity.
%%\end{MyExercise}
%\begin{definition}
%    A $\Lambda$-decorated graph is given by an ordered pair $(\Gamma,
%    \Lambda)$ of a finite graph $\Gamma$ and a set of positive integers
%    $\Lambda$ with the labeling
%    \begin{itemize}
%        \item of the vetices $v\in \Gamma ^{(0)}$ given by $n(\nu) \in
%            \Lambda$
%        \item of the edges $e = (\nu _1, \nu _2) \in \Gamma ^{(1)}$ by
%            operators
%            \begin{itemize}
%                \item $D_e: \mathbb{C}^{n(\nu _1)} \rightarrow
%                    \mathbb{C}^{n(\nu _2)}$
%                \item and $D_e^*: \mathbb{C}^{n(\nu _2)} \rightarrow
%                    \mathbb{C}^{n(\nu _1)}$ its conjugate traspose
%                    (pullback?)
%            \end{itemize}
%    \end{itemize}
%    such that
%    \begin{equation}
%        n(\Gamma ^{(0)}) = \Lambda
%    \end{equation}
%\end{definition}
%\begin{question}
%    Would then $D_e$ be the pullback?
%\end{question}
%\begin{question}
%    These graphs are important in the next chapter I should look
%    into it more, I don't understand much here, specific
%    how to construct them with the abstraction of a spectral triple...
%\end{question}
%
%The operator $D_e$ between $\textbf{n}_i$ and $\textbf{n}_j$ add up to
%$D_{ij}$
%\begin{align}
%    D_{ij} = \sum\limits_{\substack{e = (\nu _1, \nu _2) \\ n(\nu _1) =
%    \textbf{n}_i \\ n(\nu _2) = \textbf{n}_j}} D_e
%\end{align}
%
%\begin{theorem}
%    There is a on to one correspondence between finite spectral triples
%    modulo unitary equivalence and $\Lambda$-decorated graphs, given by
%    associating a finite spectral triples $(A, H, D)$ to  a $\Lambda$ decorated
%    graph $(\Gamma, \Lambda)$ in the following way:
%    \begin{equation}
%        A = \bigoplus _{n\in \Lambda} M_n(\mathbb{C}); \;\;\;
%        H = \bigoplus _{\nu \in \Gamma ^{(0)}} \mathbb{C}^{n(\nu)}; \;\;\;
%        D = \sum _{e \in \Gamma ^{(1)}} D_e + D_e^*
%    \end{equation}
%\end{theorem}
%    \begin{figure}[h!]
%    \centering
%    \begin{tikzpicture}[
%        mass/.style = {draw,circle, minimum size=0.3cm, inner sep=0pt, thick},
%    ]
%
%    \node[mass, label={\textbf{n}}] (m1) at (1,0) {};
%    \draw (m1) to [out=330, in=210, looseness=25] node[above] {$D_e$} (m1);
%    \end{tikzpicture}
%    \caption{A $\Lambda$-decorated Graph of $(M_n(\mathbb{C}), \mathbb{C}^n,
%    D = D_e + D_e^*)$}
%\end{figure}
%
%%\begin{MyExercise}
%%    \textbf{
%%    Draw a $\Lambda$ decorated graph corresponding to the spectral triple
%%    $(A=\mathbb{C}^3, H=\mathbb{C}^3, D=\begin{pmatrix}0 & \lambda & 0\\
%%    \bar{\lambda} &0 &0 \\ 0&0&0\end{pmatrix})$
%%}\newline
%%
%%\centering
%%\begin{tikzpicture}[
%%        mass/.style = {draw,circle, minimum size=0.4cm, inner sep=0pt, thick},
%%        spring/.style = {decorate,decoration={zigzag, pre length=1cm,post length=1cm,segment length=5pt}},]
%%        \node[mass] (m1) at (-1,1.5) {\textbf{1}};
%%        \node[mass] (m2) at (1,1.5) {\textbf{2}};
%%        \node[mass] (m3) at (3,1.5) {\textbf{3}};
%%
%%        \draw[style=thick, -] (1.1,1.7) -- (-1.1,1.7);
%%        \draw[style=thick, -] (1.1,1.3) -- (-1.1,1.3);
%%    \end{tikzpicture}
%%    %    \captionof{figure}{Solution}
%%\end{MyExercise}
%%\begin{MyExercise}
%%    \textbf{
%%    Use $\Lambda$-decorated graphs to classify all finite spectral triples
%%    (modulo unitary equivalence) on the matrix algebra
%%    $A=\mathbb{C}\oplus M_2(\mathbb{C})$
%%}\newline
%%
%%    \centering
%%\begin{tikzpicture}[
%%        mass/.style = {draw,circle, minimum size=0.4cm, inner sep=0pt, thick},
%%        spring/.style = {decorate,decoration={zigzag, pre length=1cm,post length=1cm,segment length=5pt}},]
%%        \node[mass] (m1) at (-1,1) {\textbf{1}};
%%        \node[mass] (m2) at (1,1) {\textbf{2}};
%%        \node[mass] (m3) at (3,1) {\textbf{3}};
%%
%%        \node[mass] (m4) at (-1,0) {\textbf{1}};
%%        \node[mass] (m5) at (1,0) {\textbf{2}};
%%        \node[mass] (m6) at (3,0) {\textbf{3}};
%%
%%        \node[mass] (m7) at (-1,-1) {\textbf{1}};
%%        \node[mass] (m8) at (1,-1) {\textbf{2}};
%%        \node[mass] (m9) at (3,-1) {\textbf{3}};
%%
%%        \node[mass] (m10) at (-1,-2) {\textbf{1}};
%%        \node[mass] (m11) at (1,-2) {\textbf{2}};
%%        \node[mass] (m12) at (3,-2) {\textbf{3}};
%%
%%        \draw[style=thick, -] (1.1,0.2) -- (-1.1,0.2);
%%        \draw[style=thick, -] (1.1,-0.2) -- (-1.1,-0.2);
%%        \draw[style=thick, -] (m7) to [out=330, in=210, looseness=10] node[above] {} (m7);
%%        \draw[style=thick, -] (m10) -- (m11) ;
%%
%%\end{tikzpicture}
%%%    \captionof{figure}{Solution $A=M_3(\mathbb{C})$}
%%\end{MyExercise}
%\subsubsection{Graph Construction of Finite Spectral Triples}
%\textbf{Algebra:}We know if a acts on a finite dimensional Hilbert space then
%this C* algebra is isomorphic to a matrix algebra so $A \simeq
%\bigoplus_{i=1}^{N}M_{n_i}(\mathbb{C})$. Where $i\in
%\hat{A}$ represents an equivalence class and runs from $1$ to $N$,
%thus $\hat{A}\simeq\{1,\dots, N\}$. We label equivalence classes by
%$\textbf{n}_i$, then $\hat{A}\simeq\{\textbf{n}_1,\dots,\textbf{n}_N\}$.
%\newline
%
%\textbf{Hilbert Space:} Since every Hilbert space that acts faithfully on a
%C* algebra is completely reducible, it is isomorphic to the composition
%of irreducible representations. $H \simeq \bigoplus_{i=1}^N\mathbb{C}^{n_i}
%\otimes V_i$. Where all $V_i$'s are Vector spaces, their dimension is the
%multiplicity of the representation landed by $\textbf{n}_i$ to $V_i$ itself
%by the multiplicity space.
%\newline
%
%\textbf{Finite Dirac Operator:} $D_{ij}$ is connecting nodes $\textbf{n}_i$
%and $\textbf{n}_j$, with a symmetric map $D_{ij}:\mathbb{C}^{n_i}\otimes V_i
%\rightarrow \mathbb{C}^{n_j}\otimes V_j$
%\newline
%
%To draw a graph, draw nodes in position $\textbf{n}_i\in \hat{A}$.
%Multiple nodes at the same position represent multiplicities in $H$.
%Draw lines between nodes to represent $D_{ij}$.
%
%\begin{figure}[h!]
%    \centering
%\begin{tikzpicture}
%    \node[draw, label=above:{$\textbf{n}_1$},circle, thick] at (-3,0) {};
%    \node[label=above:{$\dots$}] at (-2,0) {};
%    \node[draw, label=above:{$\textbf{n}_i$},circle, thick] at (-1,0) {};
%    \node[label=above:{$\dots$}] at (0,0) {};
%    \node[draw, label=above:{$\textbf{n}_j$},circle, thick] at (1,0) {};
%    \node[draw, label=above:{},circle, thick, inner sep=0cm, minimum
%    size=0.2cm]  at (1,0) {};
%    \node[label=above:{$\dots$}] at (2,0) {};
%    \node[draw, label=above:{$\textbf{n}_N$},circle, thick] at (3,0) {};
%
%        \draw[style=thick, -] (-1,-0.2) -- (1,-0.2);
%        \draw[style=thick, -] (-1,0.2) -- (1,0.2);
%        \path[style=thick, -] (-1,-0.2) edge[bend right=15]
%        node[pos=0.5,below] {} (3,-0.2);
%    \end{tikzpicture}
%    \caption{Example}
%\end{figure}
%
%\subsection{Finite Real Noncommutative Spaces}
%\subsubsection{Finite Real Spectral Triples}
%Add on to finite real spectral triples a \textit{real structure}. The
%requirement is that $H$ is a $A$-$A$-bimodule (before only a $A$-left
%module).
%\newline
%
%For this we introduce a $\mathbb{Z}_2$-grading $\gamma$ with
%\begin{align}
%    &\gamma ^* = \gamma \\
%    &\gamma ^2 = 1 \\
%    &\gamma D = - D \gamma\\
%    &\gamma a = a \gamma \;\;\;\; a\in A
%\end{align}
%
%\begin{definition}
%    A \textit{finite real spectral triple} is given by a finite spectral
%    triple $(A, H, D)$ and a anti-unitary operator $J:H\rightarrow H$ called
%    the \textit{real structure}, such that
%    \begin{align}
%        a^\circ := J a^* J^{-1}
%    \end{align}
%    is a right representation of $A$ on $H$, that is $(ab)^\circ = b^\circ
%    a^\circ$. With two requirements
%    \begin{align}
%        &[a, b^\circ] = 0\\
%        &[[D, a],b^\circ] = 0.
%    \end{align}
%    They are called the \textit{commutant property}, and mean that the left
%    action of an element in $A$ and $\Omega _D^1(A)$ commutes with the right
%    action on $A$.
%\end{definition}
%\begin{definition}
%    The $KO$-dimension of a real spectral triple is determined by the sings
%    $\epsilon, \epsilon ' ,\epsilon '' \in \{-1, 1\}$ appearing in
%    \begin{align}
%        &J^2 = \epsilon \\
%        &JD = \epsilon \ DJ\\
%        &J\gamma = \epsilon '' \gamma J.
%    \end{align}
%\end{definition}
%\begin{table}[h!]
%    \centering
%    \caption{$KO$-dimension $k$ modulo $8$ of a real spectral triple}
%    \begin{tabular}{ c | c c c c c c c c}
%        \hline
%        $k$        & 0 & 1 & 2 & 3 & 4 & 5 & 6 & 7 \\
%        \hline
%     $\epsilon$    & 1 & 1 & -1 & -1 & -1 & -1 & 1 & 1 \\
%     $\epsilon '$  & 1 & -1 & 1 & 1 & 1 & -1 & 1 & 1 \\
%     $\epsilon ''$ & 1 &  & -1 &  & 1 &  & -1 &  \\
%     \hline
%    \end{tabular}
%\end{table}
%
%
%\begin{definition}
%An opposite-algebra $A^\circ$ of a $A$ is defined to be equal to $A$ as a
%vector space with the opposite product
%\begin{align}
%    &a\circ b := ba\\
%    &\Rightarrow a^\circ = Ja^* J^{-1} \;\;\; \text{defines the left
%    representation of $A^\circ$ on $H$}
%\end{align}
%\end{definition}
%
%
%\begin{example}
%    Matrix algebra $M_N(\mathbb{C})$ acting on $H=M_N(\mathbb{C})$ by left
%    matrix multiplication with the Hilbert Schmidt inner product.
%    \begin{align}
%        \langle a , b \rangle = \text{Tr}(a^* b)
%    \end{align}
%    Then we define $\gamma (a) = a$ and $J(a) = a^*$ with $a\in H$.
%    Since $D$ mus be odd with respect to $\gamma$ it vanishes identically.
%\end{example}
%
%\begin{definition}
%    We call $\xi \in H$ \textbf{cyclic vector} in $A$ if:
%    \begin{align}
%        A\xi := { a\xi:\;\; a\in A} = H
%    \end{align}
%
%    We call $\xi \in H$ \textbf{separating vector} in $A$ if:
%    \begin{align}
%        a\xi = 0\;\; \Rightarrow \;\; a=0;\;\;\; a\in A
%    \end{align}
%\end{definition}
%
%%\begin{MyExercise}
%%    \textbf{
%%        In the previous example, show that the right action on $M_N(\mathbb{C})$
%%    on $H = M_N(\mathbb{C})$ as defined by $a \mapsto a^\circ$
%%    is given by right matrix multiplication.
%%}\newline
%%
%%    \begin{align}
%%        a^\circ \xi = J a^* J^{-1}\xi = Ja^* \xi^* = J\xi a=\xi^* a
%%    \end{align}
%%\end{MyExercise}
%%\begin{MyExercise}
%%    \textbf{
%%        Let $A= \bigoplus _i M_{n_i}(\mathbb{C})$, represented on $H = \bigoplus_i \mathbb{C}^{n_i}
%%        \otimes \mathbb{C}^{m_i}$, meaning that the irreducible representation $\textbf{n}_i$ has
%%        multiplicity $m_i$.
%%        \begin{enumerate}
%%            \item Show that the commutant $A'$ of $A$ is $A'\simeq \bigoplus_i M_{m_i} (\mathbb{C})$. As a consequence show $A'' \simeq A$.
%%            \item Show that if $\xi$ is a separating vector for $A$ than it is cyclic for $A'$.
%%        \end{enumerate}
%%    }
%%
%%
%%    \begin{enumerate}
%%        \item We know the multiplicity space is $V_i = \mathbb{C}^{m_i}$. We know that
%%            for $T\in H$ and
%%            $a\in A'$ to work we need $aT=Ta$ by laws of matrix multiplication we need
%%            $A' \simeq \oplus _i M_{m_i}(\mathbb{C})$ for this to work since $H = \bigoplus_i
%%            \mathbb{C}^{n_i}
%%        \otimes \mathbb{C}^{m_i}$
%%
%%        \item Suppose $\xi$ is cyclic for $A$ then $A'\xi = \{0\}$. Under the action of $A$ we
%%            then have $A'A\xi = AA' \xi = 0 \Rightarrow A' = 0$.\\
%%            Suppose now $\xi$ is separating for $A'$, we have $A'\xi = \{0\}$. We can define a
%%            projection in $A'$, $A\xi = P'$. With this projection we have $(1-P')\xi = 0
%%            \Rightarrow 1-P' = 0 \Rightarrow A\xi = H$.
%%    \end{enumerate}
%%\end{MyExercise}
%%\begin{MyExercise}
%%    \textbf{ Suppose $(A, H, D = 0)$ is a finite spectral triple such that $H$ possesses a
%%        cyclic and separating vector for $A$.
%%        \begin{enumerate}
%%            \item Show that the formula $S(a \xi) = a* \xi$ defines a anti-linear operator\\
%%                $S: H \rightarrow H$.
%%            \item Show that $S$ is invertible
%%            \item Let $J: H \rightarrow H$ be the operator in $S = J \Delta ^{1/2}$ with
%%                $\Delta = S^*S$. Show that $J$ is anti-unitary
%%        \end{enumerate}
%%    }
%%
%%
%%    \begin{enumerate}
%%    \item By composition $S(a\xi) = a*\xi$ this is literally anti-linearity. Does this mean
%%        $S\xi = \xi$?
%%    \item Let $\xi \in H$ be cyclic then: $S(A\xi) = A^*\xi = A\xi = H$. The same has to work
%%        for $S^{-1}$ if not then $\xi$ wouldn't exist. $S^{-1}(A^*\xi) = S^{-1}(H) = H$.
%%    \item Since $S$ is bijective then $\Delta ^{1/2}$ and $J$ need to be bijective.
%%        We also have $J = S \Delta^{-1/2}$ and $\Delta^* = \Delta$\\
%%        Now let $\xi _1 , \xi _2 \in H$        \begin{align}
%%            <J \xi _1 , J \xi _2 > &= < J^*J\xi_1 , \xi_2>^* =\\
%%            &= <(\Delta ^{-1/2})^* S^* S \Delta ^{-1/2} \xi_1, \xi_2>^* = \\
%%            &= <(\Delta^{-1/2})^* \Delta \Delta^{-1/2} \xi_1, \xi_2>^* =\\
%%            &= <\Delta^{-1/2} \Delta^{1/2}\Delta^{1/2} \Delta^{-1/2} \xi_1, \xi_2>^* =\\
%%            &= <\xi _1, \xi_2>^* = <\xi_2 , \xi_1>.
%%        \end{align}
%%    \end{enumerate}
%%\end{MyExercise}
%\subsubsection{Morphisms Between Finite Real Spectral Triples}
%Extend unitary equivalence of finite spectral triples to real ones (with $J$
%and $\gamma$)
%
%\begin{definition}
%    We call two finite real spectral triples $(A_1, H_1 ,D_1 ; J_1 , \gamma
%    _1)$ and $(A_2, H_2, D_2; J_2, \gamma _2)$ unitarily equivalent if $A_1 =
%    A_2$ and if there exists a unitary operator $U: H_1 \rightarrow H_2$ such
%    that
%    \begin{align}
%        &U\pi_1(a) U^* = \pi _2(a)\\
%        &UD_1U^*=D_2\\
%        &U\gamma _1 U^* = \gamma _2\\
%        &UJ_1 U^* = J_2
%    \end{align}
%\end{definition}
%\begin{definition}
%    Let $E$ be a $B$-$A$ bimodule. The \textit{conjugate Module} $E^\circ$ is
%    given by the $A$-$B$-bimodule.
%    \begin{align}
%        E^\circ = \{\bar{e} : e\in E\}
%    \end{align}
%    with
%    \begin{align}
%    a \cdot \bar{e} \cdot b = b^* \bar{e} a^* \;\;\;\; \forall a\in A, b \in
%        B
%    \end{align}
%\end{definition}
%$E^\circ$ is not a Hilbert bimodule for $(A, B)$ because it doesn't have a
%natural $B$-valued inner product. But there is a $A$-valued inner product on
%the left $A$-module $E^\circ$ with
%\begin{align}
%    \langle \bar{e}_1, \bar{e}_2 \rangle = \langle e_2 , e_1 \rangle
%    \;\;\;\; e_1, e_2 \in E
%\end{align}
%and linearity in $A$:
%\begin{align}
%    \langle a \bar{e}_1, \bar{e}_2 \rangle = a \langle \bar{e}_1, \bar{e}_2
%    \rangle \;\;\;\; \forall a \in A.
%\end{align}
%
%%\begin{MyExercise}
%%    \textbf{Show that $E^\circ$ is a Hilbert bimodule $(B^{\circ}, A^{\circ})$
%%    }\newline
%%
%%
%%    Straightforward show properties of the Hilbert bimodule and its $B^{\circ}$
%%    valued inner product. Let $\bar{e}_1, \bar{e}_2 \in E^{\circ}$ and $a^\circ \in A,
%%    b^\circ \in B$. \\
%%    \begin{align}
%%        <\bar{e}_1, a^\circ \bar{e}_2> &= <\bar{e}_1, Ja^*J^{-1} \bar{e}_2>=\\
%%        &= <\bar{e}_1 , J a^* e_2> = \\
%%        &= <J^{-1} e_1, a^* e_2> =\\
%%        & = <a^* e_1, e_2>= <J^{-1}(a^\circ)^* J e_1, e_2> = \\
%%        & = <J^{-1} (a^\circ)^* \bar{e}_1, e_2> =\\
%%        & = <(a^\circ)^* \bar{e}_1 , \bar{e}_2>.
%%    \end{align}
%%
%%    Next $<\bar{e}_1, \bar{e}_2 b^\circ> = <\bar{e}_1, \bar{e_2}> b^\circ$.
%%    \begin{align}
%%        <\bar{e}_1, \bar{e}_2 b^\circ>  &= <\bar{e}_1, \bar{e}_2 Jb^*J^{-1}> =\\
%%        &= <\bar{e}_1, \bar{e_2}> Jb^*J^{-1} = \\
%%        &= <\bar{e}_1, \bar{e}_2> b^\circ.
%%    \end{align}
%%    Then:
%%    \begin{align}
%%        (<\bar{e}_1, \bar{e}_2)>_{E^\circ})^* &= (<e_2, e_1>_E)^* =\\
%%        &= <e_1, e_2>_E^* = <\bar{e}_2, \bar{e}_2>_{E^\circ}
%%    \end{align}
%%    And of course $<\bar{e}, \bar{e}> = <e, e> \geq 0$
%%\end{MyExercise}
%
%\subsubsection{Construction of a Finite Real Spectral Triple from a Finite
%Real Spectral Triple}
%Given a Hilbert bimodule $E$ for $(B, A)$ we construct a spectral triple
%$(B, H', D'; J', \gamma ')$ from $(A, H, D; J, \gamma)$
%
%For the $H'$ we make a $\mathbb{C}$-valued inner product on $H'$ by combining
%the $A$ valued inner product on $E$ and $E^\circ$ with the
%$\mathbb{C}$-valued inner product on $H$.
%\begin{align}
%    H' := E\otimes _A H \otimes _A E^\circ
%\end{align}
%
%Then the action of $B$ on $H'$ is:
%\begin{align}
%    b(e_2 \otimes \xi \otimes \bar{e}_2 ) = (be_1) \otimes \xi \otimes
%    \bar{e}_2
%\end{align}
%The right action of $B$ on $H'$ defined by action on the right component
%$E^\circ$
%\begin{align}
%    J'(e_1 \otimes \xi \otimes \bar{e}_2) = e_2 \otimes J \xi \otimes
%    \bar{e}_1
%\end{align}
%with $b^\circ = J' b^* (J')^{-1}$, $b^* \in B$ action on $H'$.
%\newline
%
%
%\newpage
%%\begin{MyExercise}
%%    \textbf{ Let $\nabla : E \Rightarrow E \otimes _A \Omega _d^1 (A)$ be a right connection on $E$
%%    consider the following anti-linear map:
%%    \begin{align}
%%        \tau : E \otimes_A \Omega _D^1 (A) &\rightarrow \Omega _D^1 (A) \otimes_A E^\circ\\
%%                e \otimes \omega &\mapsto -\omega ^* \otimes \bar{e}
%%    \end{align}
%%    Show that the map $\bar{\nabla} : E^\circ \rightarrow \Omega _D^1(A) \otimes E^\circ$
%%    with $\bar{\nabla}(\bar{e}) = \tau \circ \nabla(e)$ is a left connection, that means
%%    show that it satisfied the left Leibniz rule:
%%    \begin{equation}
%%        \bar{\nabla}(a\bar{e}) = [D, a] \otimes \bar{e} + a \bar{\nabla}(\bar{e})
%%    \end{equation}
%%    }
%%
%%    Hagime:
%%    \begin{align}
%%        &\text{For one:}\\
%%        &\tau \circ \nabla(ae) = \bar{\nabla}(a\bar{e}) = \bar{\nabla}(a^* \bar{e})\\
%%        &\text{For two:}\\
%%         &\tau \circ \nabla(ae) = \tau(\nabla(e)a) + \tau \circ(e \otimes d(a))=\\
%%         &=a^*\bar{\nabla}(\bar{e}) - d(a)^* \otimes \bar{e}. \\
%%         &= a^*\bar{\nabla}(\bar{e}) + d(a^*) \otimes \bar{e}.
%%    \end{align}
%%\end{MyExercise}
%Then the connections
%\begin{align}
%    &\nabla: E \rightarrow E\otimes _A \Omega _D ^1(A) \\
%    &\bar{\nabla}:E^\circ \rightarrow \Omega _D^1(A) \otimes _A E^\circ
%\end{align}
%give us the Dirac operator on $H' = E \otimes _A H \otimes _A E^\circ$
%\begin{align}
%    D'(e_1 \otimes \xi \otimes \bar{e}_2) = (\nabla e_1) \xi \otimes
%    \bar{e_2}+ e_1 \otimes D\xi \otimes \bar{e}_2 + e_1 \otimes
%    \xi(\bar{\nabla}\bar{e}_2)
%\end{align}
%
%And the right action of $\omega \in \Omega _D ^1(A)$ on $\xi \in H$ is
%defined by
%\begin{align}
%    \xi \mapsto \epsilon' J \omega ^* J^{-1}\xi
%\end{align}
%
%Finally for the grading
%\begin{align}
%    \gamma ' = 1 \otimes \gamma \otimes 1
%\end{align}
%
%\begin{theorem}
%    Suppose $(A, H, D; J, \gamma)$ is a finite spectral triple of
%    $KO$-dimension $k$, let $\nabla$ be like above satisfying the
%    compatibility condition (like with finite spectral triples).
%
%    Then $(B, H',D'; J', \gamma')$ is a finite spectral triple of
%    $KO$-Dimension $k$. ($H', D', J', \gamma'$ like above)
%\end{theorem}
%
%\begin{proof}
%    The only thing left is to check if the $KO$-dimension is preserved,
%    for this we check if the $\epsilon$'s are the same.
%    \begin{align}
%        &(J')^2 = 1 \otimes J^2 \otimes 1 = \epsilon\\
%        &J' \gamma '= \epsilon ''\gamma'J'
%    \end{align}
%    and for $\epsilon '$
%    \begin{align}
%        J'D'(e_1 \otimes \xi \otimes \bar{e}_2)&=J'((\nabla e_1) \xi \otimes
%        \bar{e_2} + e_1 \otimes D\xi \otimes \bar{e}_2 + e_1 \otimes \xi (\tau
%        \nabla e_2))\\
%        &= \epsilon' D'(e_2 \otimes J\xi \otimes \bar{e}_2)\\
%        &= \epsilon' D'J'(e_1 \otimes \xi \bar{e}_2)
%    \end{align}
%\end{proof}
%
%
%\end{document}
