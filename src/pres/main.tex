\documentclass[fleqn]{beamer}
\usepackage[T1]{fontenc}
\usepackage[utf8]{inputenc}

\usepackage{amsmath,amssymb}
\usepackage{graphicx}
\usepackage{mathptmx}
\usepackage{subcaption}
\usepackage{amsthm}
\usepackage{tikz}
%\usepackage[colorlinks=true,naturalnames=true,plainpages=false,pdfpagelabels=true]{hyperref}
\usetikzlibrary{patterns,decorations.pathmorphing,positioning, arrows, chains}

\usepackage[backend=biber]{biblatex}
\addbibresource{uni.bib}

% vertical separator macro
\newcommand{\vsep}{
  \column{0.0\textwidth}
    \begin{tikzpicture}
      \draw[very thick,black!10] (0,0) -- (0,7.3);
    \end{tikzpicture}
}
\setlength{\mathindent}{0pt}

% Beamer theme
\usetheme{UniVienna}
\usefonttheme[onlysmall]{structurebold}
\mode<presentation>
\setbeamercovered{transparent=10}

\title
{Noncommutative Geometry}
\subtitle{Bachelor's seminar}
\author[Popovic Milutin]
{Popovic Milutin \newline Supervisor: Dr. Lisa Glaser}
\date{16. April 2021}

\begin{document}
    \begin{frame}
        \titlepage
    \end{frame}

    \begin{frame}{Introduction}
        \begin{itemize}
            \item Noncommutative geometry (NCG) brings many\\
                  mathematical fields together
            \item Interesting physics application
            \item First understand duality between (classical) geometry and commutative Algebras
            \item Gelfand-Naimark-Theoreme in Functional Analysis in the 1940s
        \end{itemize}
    \end{frame}

    \begin{frame}{Spaces and commutative Algebras}
        Introduce:
        \begin{block}
            {Finite topological Space $X$ consisting of $N$ points. (discrete topology)}
            \begin{figure}[h!]
            \centering
            \begin{tikzpicture}[
                dot/.style = {draw, circle, inner sep=0.05cm, fill},
                smalldot/.style = {draw, circle, inner sep=0.015cm,fill},
                ]
                \node[dot] at (-3,0.) [label=below:$1$]{};
                \node[dot] at (-1.5,0) [label=below:$2$]{};
                \node[dot] at (2.1,0) [label=below:$N$]{};
                \node[smalldot] at (-0.4,0) {};
                \node[smalldot] at (0.1,0) {};
                \node[smalldot] at (0.6,0) {};
                \end{tikzpicture}
            \end{figure}

        \end{block}
        \begin{block}{Commutative algebra of continuous functions on $X$}
            \centering
             $C(X) = \{ f: X \rightarrow \mathbb{C}:\;\;
             \text{$f$ is continuous}\}$
        \end{block}
    \end{frame}
    \begin{frame}
        {Spaces and commutative Algebras}
        Results of the Theorem:
            \begin{itemize}
                \item $X$ and $C(X)$ contain the same information (duality)
                \item Construct $X$, given $C(X)$.
                \item Translate geometrical properties of $X$ to algebraic data\\
                    (metric, differential forms, vector fields, curvature, etc.)
            \end{itemize}
    \end{frame}

    \begin{frame}{The Finite Spectral Triple}
        \begin{itemize}
            \item NCG extends this duality to noncommutative algebras
            \item Provides methods to deal with these algebras
            \item NCG is encoded in a spectral triple
        \end{itemize}
        \begin{block}
            {\centering The Finite Spectral Triple}
            \centering
            $(A, H, D)$
        \end{block}
        \begin{itemize}
            \item $A$ - Algebra
            \item $H$ - Hilbertspace
            \item $D$ - Symmetric operator acting on $H$
        \end{itemize}
    \end{frame}

    \begin{frame}{Introducing the Metric}
        \begin{itemize}
            \item The metric describes distances between points on a space
            \item Simple example the discrete metric on a discrete space
        \end{itemize}
        \centering
            \begin{align*}
                d_{ij} =
                \begin{cases}
                    1\;\;\; \text{if}\;\;\; i \neq j \\
                    0\;\;\; \text{if}\;\;\; i = j
                \end{cases}
            \end{align*}
            \begin{figure}[h!] \centering
            \begin{tikzpicture}[
                dot/.style = {draw, circle, inner sep=0.05cm, fill},
                smalldot/.style = {draw, circle, inner sep=0.015cm,fill},
                ]
                \node[dot](m1) at (-3,0.) [label=left:$1$] {};
                \node[dot](m2) at (-1.5, 2) [label=above right:$2$] {};
                \node[dot](m3) at (2.1,0) [label=right:$3$] {};

                \draw[<->, >=stealth](m1) -- ++(m2) node [midway, fill=white] {$d_{12}$};
                \draw[<->, >=stealth](m2) -- ++(m3) node [midway, fill=white] {$d_{23}$};
                \draw[<->, >=stealth](m3) -- ++(m1) node [midway, fill=white] {$d_{13}$};
                \end{tikzpicture}
            \end{figure}
    \end{frame}

    \begin{frame}{Metric on NCG}
        \begin{itemize}
            \item In NCG we can also define a metric
            \item replace algebra with a noncommutative matrix Algebra $A$
            \item a finite-dimensional Hilbertspace $H$\\
            \item and a hermitian matrix $D$
        \end{itemize}
        \begin{block}{A metric constructed with $(A, H, D)$}
            \centering
            $d_{ij} = \sup_{a \in A}\{ |\text{Tr}(a(i)) - \text{Tr}(a(j)) :
            ||[D, a]|| < 1\}$
        \end{block}
    \end{frame}

    \begin{frame}{Differential One Forms}
        \begin{itemize}
                \item Example: 1D calculus $f(x)dx$\\
                \item In NCG defining the differential one form requires only\\
                    the spectral Triple $(A, H, D)$
        \end{itemize}
        \begin{block}{Connes' Differential One Form}
            \begin{center}
                $\Omega_D^1 (A) = \{\sum_k a_k [D, b_k]: a_k, b_k \in A\}$
            \end{center}
            With a consequent derivation of a algebra
            $d: A \rightarrow \Omega_D^1$, $d(\cdot) = [D, \cdot]$
        \end{block}
   \end{frame}

    \begin{frame}{Generalization to the Continuum}
        \begin{itemize}
            \item Introduce a richer geometry
            \item From finite topological space to Manifolds generalized with noncommutativity
            \item From finite to general spectral triples with a \\
                self adjoint operator (Dirac Operator)
        \end{itemize}
    \end{frame}

    \begin{frame}{Applications In Physics}
        \begin{itemize}
            \item NCG of Electrodynamics
            \item NCG of the Quantum Hall Effect
            \item NCG of the Standard Model, full Lagrangian
        \end{itemize}
    \end{frame}

    \begin{frame}{Bibliography}
        \nocite{ncgwalter}
        \nocite{liealgebra}
        \nocite{ncg4pages}
        \nocite{ncgshort}
        \printbibliography
    \end{frame}
\end{document}

