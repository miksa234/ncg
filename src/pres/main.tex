\documentclass[fleqn]{beamer}
\beamertemplatenavigationsymbolsempty

\usepackage[T1]{fontenc}
\usepackage[utf8]{inputenc}

\usepackage{amsmath,amssymb}
\usepackage{graphicx}
\usepackage{mathptmx}
\usepackage{subcaption}
\usepackage{amsthm}
\usepackage{tikz}
%\usepackage[colorlinks=true,naturalnames=true,plainpages=false,pdfpagelabels=true]{hyperref}
\usetikzlibrary{patterns,decorations.pathmorphing,positioning, arrows, chains}

\usepackage[backend=biber, sorting=none]{biblatex}
\addbibresource{uni.bib}

\setbeamertemplate{endpage}{%
    \begin{frame}
        \centering
        \Large \emph{Thank You!}
    \end{frame}
}

\AtEndDocument{\usebeamertemplate{endpage}}

% vertical separator macro
\newcommand{\vsep}{
  \column{0.0\textwidth}
    \begin{tikzpicture}
      \draw[very thick,black!10] (0,0) -- (0,7.3);
    \end{tikzpicture}
}
\setlength{\mathindent}{0pt}

% Beamer theme
\usetheme{UniVienna}
\usefonttheme[onlysmall]{structurebold}
\mode<presentation>
\setbeamercovered{transparent=10}

\title
{Noncommutative Geometry}
\subtitle{Bachelor's seminar}
\author[Popovic Milutin]
{Popovic Milutin \newline Supervisor: Dr. Lisa Glaser}
\date{16. April 2021}

\begin{document}
    \begin{frame}
        \titlepage
    \end{frame}

    \begin{frame}{Introduction}
        \begin{itemize}
            \item Noncommutative geometry (NCG) brings many\\
                mathematical fields together (e.g. K-Theory, Differential Geometry)
            \item Physics application (spectral Standard Model)
            \item Gelfand-Naimark-Theorem in Functional Analysis in the 1940s \\
              duality between (classical) geometry and Algebra
        \end{itemize}
    \end{frame}

    \begin{frame}{Spaces and Algebras}
        Introduce:
        \begin{block}
            {Algebra}
            \centering
            Vectorspace with a multiplication operation\\
            (associative and possesses an identity element)
        \end{block}
        \begin{block}
            {Finite topological Space $X$ consisting of $N$ points. (discrete topology)}
            \begin{figure}[h!]
            \centering
            \begin{tikzpicture}[
                dot/.style = {draw, circle, inner sep=0.05cm, fill},
                smalldot/.style = {draw, circle, inner sep=0.015cm,fill},
                ]
                \node[dot] at (-3,0.) [label=below:$1$]{};
                \node[dot] at (-1.5,0) [label=below:$2$]{};
                \node[dot] at (2.1,0) [label=below:$N$]{};
                \node[smalldot] at (-0.4,0) {};
                \node[smalldot] at (0.1,0) {};
                \node[smalldot] at (0.6,0) {};
                \end{tikzpicture}
            \end{figure}

        \end{block}
        \begin{block}{Commutative algebra of continuous functions on $X$}
            \centering
             $C(X) = \{ f: X \rightarrow \mathbb{C}:\;\;
             \text{$f$ is continuous}\}$
        \end{block}
    \end{frame}
    \begin{frame}
        {Spaces and commutative Algebras}
        Results of the Theorem:
            \begin{itemize}
                \item $X$ and $C(X)$ contain the same information (duality)
                \item Construct $X$, given $C(X)$.
                \item Translate geometrical properties of $X$ to algebraic data\\
                    (metric, differential forms, vector fields, curvature, etc.)
            \end{itemize}
    \end{frame}

    \begin{frame}{Geometry as a Spectral Triple}
        \begin{block}
            {\centering The Spectral Triple}
            \centering
            $(\;A,\;\; H,\;\; D\;)$
        \end{block}
        \begin{itemize}
            \item $A$ - Algebra
            \item $H$ - Hilbertspace
            \item $D$ - self adjoint Operator acting on $H$
        \end{itemize}
    \end{frame}

    \begin{frame}{Geometry as a Spectral Triple}
        \begin{block}
            {The Spectral Triple of a Circle $\mathbb{S}^1$}
        \centering
                $ (\; C^{\infty}(\mathbb{S}^1),\;\; L^2(\mathbb{S}^1),\;\; -i\frac{d}{dt} \;)$
        \end{block}
    \end{frame}

    \begin{frame}{Introducing the Metric}
        \begin{itemize}
            \item The metric describes distances between points on a space
        \end{itemize}
  \begin{columns}[T]
    \column{0.4\textwidth}
        \begin{block}{\centering Discrete Metric}
            \centering
            $
                d_{ij} =
                \begin{cases}
                    0\;\;\; \text{if}\;\;\; i = j \\
                    1\;\;\; \text{if}\;\;\; i \neq j
                \end{cases}
            $
      \end{block}

    \column{0.4\textwidth}
    \begin{block}{\centering Minkowski Metric}
        \centering
        $
            \eta _{\mu \nu} =
                \begin{pmatrix}
                    -1 & 0 & 0 & 0 \\
                    0 & 1 & 0 & 0 \\
                    0 & 0 & 1 & 0 \\
                    0 & 0 & 0 & 1
                \end{pmatrix}
        $
    \end{block}
    \end{columns}

    \begin{figure}[h!] \centering
    \begin{tikzpicture}[
        dot/.style = {draw, circle, inner sep=0.05cm, fill},
        smalldot/.style = {draw, circle, inner sep=0.015cm,fill},
        ]
        \node[dot](m1) at (-3,0.) [label=left:$1$] {};
        \node[dot](m2) at (-1.5, 2) [label=above right:$2$] {};
        \node[dot](m3) at (2.1,0) [label=right:$3$] {};

        \draw[<->, >=stealth](m1) -- ++(m2) node [midway, fill=white] {$d_{12}$};
        \draw[<->, >=stealth](m2) -- ++(m3) node [midway, fill=white] {$d_{23}$};
        \draw[<->, >=stealth](m3) -- ++(m1) node [midway, fill=white] {$d_{13}$};
        \end{tikzpicture}
        \end{figure}

    \end{frame}


    \begin{frame}{Algebraic Formulation of the Metric}
        \begin{itemize}
        \item Utilize results of the Gelfand-Naimark Theorem
        \item Characterize the Metric with\\
            \begin{itemize}
                \item[\bullet] commutative Algebra
                \item[\bullet] finite-dimensional Hilbertspace $H$
                \item[\bullet] symmetric operator $D$
            \end{itemize}
        \end{itemize}
        \begin{block}{Metric with $(A, H, D)$ on finite Space (commutative case)}
            \centering
            $d_{ij} = \sup_{a \in A}\{ |a(i) - a(j)| : ||[D, a]|| \leq 1\}$
        \end{block}
    \end{frame}

    \begin{frame}{Algebraic Formulation of the Metric}
     In the noncommutative Case:
        \begin{itemize}
            \item replace Algebra with matrix Algebra (noncommutative)
            \item define in terms of invariants
        \end{itemize}
        \begin{block}{Metric with $(A, H, D)$ on finite Space (noncommutative case)}
            \centering
            $d_{ij} = \sup_{a \in A}\{ |\text{Tr}(a(i)) - \text{Tr}(a(j))| :||[D, a]|| \leq 1\}$
        \end{block}
    \end{frame}

    \begin{frame}{Algebraic Formulation of the Metric}
        \begin{itemize}
            \item describe the Metric on a Manifold $M$
            \item We need \\
                \begin{itemize}
                    \item[\bullet] $C^\infty(M)$ - Algebra
                    \item[\bullet] $L^2(S)$ - Hilbertspace
                    \item[\bullet] $D$ - Dirac Operator
                \end{itemize}
        \end{itemize}
        \begin{block}{Metric with $(C^\infty(M),\;\; L^2(S),\;\; D)$ on a Manifod}
            \centering
            $d(x, y) = \sup_{f \in C^\infty(M) }\{ |f(x) - f(y)| :
            ||[D, f]|| \leq 1\}$
        \end{block}
    \begin{figure}[h!] \centering
    \begin{tikzpicture}[
        dot/.style = {draw, circle, inner sep=0.06cm, fill},
        smalldot/.style = {draw, circle, inner sep=0.015cm,fill},
        ]
        \node[dot](b) at (0,0) [label=below left:$x$] {};
        \node[dot](a) at (2, 0) [label=below right:$y$] {};
        \node[dot](a) at (5, 0) [label=below left:$x$] {};
        \node[dot](a) at (8, 0) [label=below right:$y$] {};
        %        \node[dot](m3) at (2.1,0) [label=right:$3$] {};

         \draw[<->, >=stealth, line width=0.4mm, style=dashed](0, 0.2) -- ++(2, 0) {};
         \draw[line width=0.5mm] (-0.3, 0) -- (2.3, 0) {};

         \draw[line width=0.5mm] (4.7, 0) -- (8.3, 0) {};
         \draw[<->, >=stealth, line width=0.4mm, style=dashed](8, 2) -- (8, 0.1) {};
         \draw[line width=0.5mm] (5, 0) -- (8, 2) node [pos=.75, label=:$f$] {} ;
        \end{tikzpicture}
        \end{figure}
    \end{frame}

%\begin{frame}{Noncommutative Case}
%    \begin{itemize}
%        \item Introduce a richer geometry
%        \item From finite topological space to a Manifold with noncommutativity
%            \item From finite to general spectral triples with a \\
%                self adjoint Operator (Dirac Operator)
%        \end{itemize}
%    \end{frame}

    \begin{frame}{Applications In Physics}
        \begin{itemize}
            \item NCG of the Quantum Hall Effect
            \item NCG of the Standard Model
                \begin{itemize}
                    \item[\bullet] going to noncommutative Manifolds
                    \item[\bullet] obtain Standard Model gauge fields (scalar Higgs filed)
                    \item[\bullet] minimal coupling to gravity
                    \item[\bullet] construct the Full Lagrangian
                \end{itemize}
        \end{itemize}
    \end{frame}

    \begin{frame}{Bibliography}
        \nocite{ncgwalter}
        \nocite{liealgebra}
        \nocite{ncg4pages}
        \nocite{ncgshort}
        \printbibliography
    \end{frame}
\end{document}

