\documentclass[a4paper]{article}

\usepackage[T1]{fontenc}
\usepackage[utf8]{inputenc}

\usepackage{mathptmx}

\usepackage{subcaption}
\usepackage[shortlabels]{enumitem}
\usepackage{amssymb}
\usepackage{amsthm}
\usepackage{mathtools}
\usepackage{bbm}
\usepackage{graphicx}
\usepackage[colorlinks=true,naturalnames=true,plainpages=false,pdfpagelabels=true]{hyperref}
\usepackage[parfill]{parskip}

\usepackage{tikz}
\usetikzlibrary{patterns,decorations.pathmorphing,positioning}

\usepackage[framemethod=TikZ]{mdframed}

\tikzstyle{titlered} =
    [draw=black, thick, fill=white,%
        text=black, rectangle,
        right, minimum height=.7cm]

\newcounter{exercise}

\renewcommand*\theexercise{Exercise~\arabic{exercise}}

\makeatletter
\mdfdefinestyle{exercisestyle}{%
    outerlinewidth=1em,%
    outerlinecolor=white,%
    leftmargin=-1em,%
    rightmargin=-1em,%
    middlelinewidth=1.2pt,%
    roundcorner=5pt,%
    linecolor=black,%
    backgroundcolor=blue!5,
    innertopmargin=1.2\baselineskip,
    skipabove={\dimexpr0.5\baselineskip+\topskip\relax},
    skipbelow={-1em},
    needspace=3\baselineskip,
    frametitlefont=\sffamily\bfseries,
    settings={\global\stepcounter{exercise}},
    singleextra={%
        \node[titlered,xshift=1cm] at (P-|O) %
            {~\mdf@frametitlefont{\theexercise}~};},%
    firstextra={%
            \node[titlered,xshift=1cm] at (P-|O) %
                    {~\mdf@frametitlefont{\theexercise}~};},
}
\makeatother

\newenvironment{MyExercise}%
{\begin{mdframed}[style=exercisestyle]}{\end{mdframed}}

\theoremstyle{definition}
\newtheorem{definition}{Definition}

\theoremstyle{definition}
\newtheorem{question}{Question}

\theoremstyle{definition}
\newtheorem{example}{Example}

\theoremstyle{theorem}
\newtheorem{theorem}{Theorem}

\theoremstyle{theorem}
\newtheorem{lemma}{Lemma}


\theoremstyle{theorem}
\newtheorem{proposition}{Proposition}

\newtheorem*{idea}{Proof Idea}


\title{Notes on \\ Noncommutative Geometry and Particle Physics}
\author{Popovic Milutin}
\date{Week 6: 19.03 - 26.03}

\begin{document}

    \maketitle
    \tableofcontents
    \section{Excurse}
    \textbf{Manifold:} A topological space that is locally Euclidean.
    \newline
    \textbf{Riemannian Manifold:}A Manifold equipped with a riemannian
    Metric, a
    symmetric bilinear form on Vector Fields $\Gamma(TM)$
    \begin{align}
        &g: \Gamma(TM) \times \Gamma(TM) \rightarrow C(M) \\
        \text{with}& \nonumber\\
        &g(X, Y) \in \mathbb{R} \;\;\; \text{if $X, Y \in \mathbb{R}$}\\
        &\text{$g$ is $C(M)$-bilinear } \forall f\in C(M):\;\; g(fX, Y) =
        g(X,
        fY) = fg(X,Y)\\
        &g(X,X) \begin{cases}\geq 0  \;\;\; \forall X \\ = 0 \;\;\; \forall X
            =0
        \end{cases}
    \end{align}
    $g$ on $M$ gives rise to a distance function on $M$
    \begin{align}
        d_g(x, y) = \inf_\gamma \left\{\int_0^1(\dot{\gamma}(t),
        \dot{\gamma}(t))dt;\;\; \gamma(0) = x, \gamma(1) = y \right\}
    \end{align}
    Riemannian Manifold is called spin$^c$ if there exists a vector bundle $S
    \rightarrow M$ with an algebra bundle isomorphism
    \begin{align}
        \mathbb{C}\text{I}(TM) &\simeq \text{End}(S)\;\;\; &\text{($dim(M)$
        even)}\\
        \mathbb{C}\text{I}(TM)^\circ &\simeq \text{End}(S)\;\;\;
        &\text{($dim(M)$ odd)}\\
    \end{align}
    $(M,S)$ is called the \textbf{spin$^c$ structure on $M$}.
    \newline
    $S$ is called the \textbf{spinor Bundle}.
    \newline
    $\Gamma(S)$ are the \textbf{spinors}.

    Riemannian spin$^c$ Manifold is called spin if there exists an
    anti-unitary
    operator $J_M:\Gamma(S) \rightarrow \Gamma(S)$ such that:
    \begin{enumerate}
        \item $J_M$ commutes with the action of real-valued  continuous
            functions
            on $\Gamma(S)$.
        \item $J_M$ commutes with $\text{Cliff}^-(M)$ (even case)\\
        $J_M$ commutes with $\text{Cliff}^-(M)^\circ$ (odd case)
    \end{enumerate}
    $(S, J_M)$ is called the \textbf{spin Structure on $M$}
    \newline
    $J_M$ is called the \textbf{charge conjugation}.
    \section{Noncommutative Geomtery of Electrodynamics}
    \subsection{The Two-Point Space}
    Consider a two point space $X := \{x, y\}$. This space=an be described
    with
    the following spectral triple
    \begin{align}
        F_x := (C(X) = \mathbb{C}^2, H_F, D_F; J_F, \gamma _f).
    \end{align}

    Notes on the spectral triple:
    \begin{itemize}
        \item Action of $C(X)$ on $H_F$ is faithful ($\dim (H_F) \geq 2$)\\
            we choose $H_F = \mathbb{C}^2$
        \item $\gamma_F$ is the $\mathbb{Z}_2$ grading, which allows us to
            decompose $H_F = H_F^+ \oplus H_F^- = \mathbb{C} \oplus \mathbb{C}$\\
            where $H_F^{\pm} = \{ \psi \in H_F |\;\; \gamma _F \psi = \pm \psi\}$
            are the two eigenspaces
        \item $D_F$ interchanges between $H_F^\pm$, $D_F =
            \begin{pmatrix}0 & t \\ \bar{t} & 0\end{pmatrix}$ where $t \in
                \mathbb{C}$
    \end{itemize}

    \begin{proposition}
        $F_x$ can only have a real structure if $D_F = 0$ in that case we
        have
        $KO-dim = 0, 2, 6$
    \end{proposition}
    \begin{proof}
        There are two diagram representations of $F_x$ at
        $\underbrace{\mathbb{C} \oplus \mathbb{C}}_{C(X)}$
        on $\underbrace{\mathbb{C} \oplus\mathbb{C}}_{H_F}$

        \begin{figure}[h!] \centering
        \begin{tikzpicture}[
            dot/.style = {draw, circle, inner sep=0.06cm},
            no/.style = {},
            ]
            \node[no](a) at (0,0) [label=left:$\textbf{1}^\circ$] {};
            \node[no](b) at (0, -1) [label=left:$\textbf{1}^\circ$] {};
            \node[no](c) at (1, 0.5) [label=above:$\textbf{1}$] {};
            \node[no](d) at (2, 0.5) [label=above:$\textbf{1}$] {};
            \node[dot](d0) at (2,0) [] {};
            \node[dot](d0) at (1,-1) [] {};

            \node[no](a1) at (6,0) [label=left:$\textbf{1}^\circ$] {};
            \node[no](b2) at (6, -1) [label=left:$\textbf{1}^\circ$] {};
            \node[no](c2) at (7, 0.5) [label=above:$\textbf{1}$] {};
            \node[no](d2) at (8, 0.5) [label=above:$\textbf{1}$] {};
            \node[dot](d0) at (7,0) [] {};
            \node[dot](d0) at (8,-1) [] {};
            \end{tikzpicture}
        \end{figure}
    If $F_x$ a real spectral triple then $D_F$ can only go vertically or
    horizontally $\Rightarrow D_F = 0$.  Furthermore the diagram on the
    left has KO-dimension 2 and 6, diagram on the right has KO-dimension
    0 and 4. Yet KO-dimension 4 is not allowed because
    $dim(H_F^\pm) = 1$ (see Lemma 3.8 Book), so $J_F^2 = -1$ is not
    allowed.
    \end{proof}
    \subsection{The product Space}
    Let $M$ be a 4-dim Riemannian spin Manifold, then we have the almost
    commutative manifold $M\times F_x$
    \begin{align}
        M\times F_x = (C^\infty(M, \mathbb{C}^2, L^2(S)\otimes \mathbb{C}^2,
        D_M\otimes 1 ; J_M\otimes J_F, \gamma_M \otimes \gamma_F)
    \end{align}
    ($J_M$ is missing need to choose)\newline
    $C^\infty(M, \mathbb{C}^2) \simeq C^\infty(M) \oplus  C^\infty(M)$
    (decomposition) and from Gelfand duality we we have
    \begin{align}
        N:= M\otimes X \simeq M\sqcup X
    \end{align}
    $H = L^2(S) \oplus L^2(S)$ (decomposition), such that for
    $\underbrace{a,b
    \in C^\infty(M)}_{(a, b) \in C^\infty(N)}$
    and $\underbrace{\psi, \phi \in L^2(S)}_{(\psi, \phi) \in H}$ we have
    \begin{align}
        (a, b)(\psi, \phi) = (a\psi, b\phi)
    \end{align}
    We can consider a distance formula on $M\times F_x$ by
    \begin{align}
        d_{D_F}(x,y) = \sup\left\{  |a(x) - a(y)|:a\in A_F, ||[D_F, a]|| \leq
        1 \right\}
    \end{align}
    Now lets calculate the distance between two points on the two point space
    $X=
    \{x, y\}$, between $x$ and $y$. Let $a \in \mathbb{C}^2 = C(X)$, $a$ is
    specified with two complex numbers $a(x)$ and $a(y)$
    \begin{align}
        &||[D_F , a]|| = ||(a(y) - a(x))\begin{pmatrix}0 &t\\\bar{t} &0
        \end{pmatrix}|| \leq 1\\
        &\Rightarrow |a(y) - a(x)|\leq \frac{1}{|t|}
    \end{align}
    Therefore the distance between two points $x$ and $y$ is
    \begin{align}
        d_{D_F} (x,y) = \frac{1}{|t|}
    \end{align}
    Note that if there exists $J_M$ (real structure) $\Rightarrow t=0$ then
    $d_{D_F}(x,y) \rightarrow \infty$!
    \newline

    Now let $p \in M$, then take two points on $N=M\times X$, $(p, x)$ and
    $(p,y)$ and $a \in C^\infty(N)$ is determined by $a_x(p):=a(p, x)$ and
    $a_y(p):=a(p, y)$. The distance between these two points is then
    \begin{align}
        d_{D_F\otimes 1}(n_1, n_2) =  \sup \left\{ |a(n_1) - a(n_2)|: a\in
        A, ||[D\otimes 1, a]||\right\}
    \end{align}
    \textbf{Remark}: If $n_1 = (p,x)$ and $n_2 = (q, x)$ for $p,q \in M$ then
    \begin{align}
        d_{D_M \otimes 1} (n_1, n_2) = |a_x(p) - a_x(q)| \;\;\; a_x\in
        C^\infty(M) \;\; \text{with} \;\; ||[D\otimes 1, a_x]|| \leq 1
    \end{align}
    The distance turns to the geodestic distance formula
    \begin{align}
        d_{D_M\otimes1}(n_1, n_2) = d_g(p, q)
    \end{align}

    However if $n_1 = (p, x)$ and $n_2 = (q, y)$ then the two conditions are
    $||[D_M, a_x]|| \leq 1$ and $||[D_M, a_y|| \leq 1$. They have no
    restriction which results in the distance being infinite! And $N =
    M\times X$ is given by two disjoint copies of M  which are separated by
    infinite distance

    \textbf{Note}: distance is only finite if $[D_F, a] \neq 1$. The
    commutator
    generates a scalar field say $\phi$ and the finiteness of the distance is
    related to the existence of scalar fields.
    \subsection{$U(1)$ Gauge Group}
    Here we determine the Gauge theory corresponding to the almost
    commutative
    Manifold $M\times F_x$.

    \textbf{Gauge Group of a Spectral Triple}:
    \begin{align}
        \mathfrak{B}(A, H; J) := \{ U = uJuJ^{-1} | u\in U(A)\}
    \end{align}
    \begin{definition}
        A *-automorphism of a *-algebra $A$ is a linear invertible
        map
        \begin{align}
            &\alpha:A \rightarrow A\;\;\; \text{with}\\
            \nonumber\\
            &\alpha(ab) = \alpha(a)\alpha(b)\\
            &\alpha(a)^* = \alpha(a^*)
        \end{align}
        The \textbf{Group of automorphisms of the *-Algebra $A$} is
        $(A)$.\newline
        The automorphism $\alpha$ is called \textbf{inner} if
        \begin{align}
            \alpha(a) = u a u^* \;\;\; \text{for} \;\; U(A)
        \end{align}
        where $U(\mathfrak{A})$ is
        \begin{align}
            U(A) = \{ u\in A|\;\; uu^* = u^*u=1\} \;\;\;
            \text{(unitary)}
        \end{align}
    \end{definition}
    The Gauge group is given by the quotient $U(A)/U(A_J)$.
    We want a nontrivial Gauge group so we need to choose $U(A_J) \neq
    U(A)$ which is the same as $U((A_F)_{J_F}) \neq
    U(A_F)$.
    We consider $F_x$ to be
    \begin{align}
        F_x := \left(\mathbb{C}^2,\mathbb{C}^2, D_F =\begin{pmatrix}
            0&0\\0&0\end{pmatrix}; J_f =\begin{pmatrix}
        0&C\\C&0\end{pmatrix},
                \gamma_F = \begin{pmatrix}1&0\\0&-1\end{pmatrix}\right).
    \end{align}
    Here $C$ is the complex conjugation, and $F_X$ is a real even finite
        spectral triple (space) with $KO-dim=6$

    \begin{proposition}
        The Gauge group $\mathfrak{B}(F)$ of the two point space is given by
        $U(1)$.
    \end{proposition}
    \begin{proof}
        Note that $U(A_F) = U(1) \times U(1)$. We need to show that
        $U(\mathcal{A}_F)
        \cap U(A_F)_{J_F}) \simeq U(1)$, such that $\mathfrak{B}(F)
        \simeq U(1)$.\newline

        So for $a \in \mathbb{C}^2$ to be in $(A_F)_{J_F}$ it has
        to satisfy $J_F a^* J_F = a$.
        \begin{align}
            J_F a^* J^{-1} =
            \begin{pmatrix}0&C\\C&0\end{pmatrix}
                \begin{pmatrix}\bar{a}_1&0\\0&\bar{a}_2\end{pmatrix}
            \begin{pmatrix}0&C\\C&0\end{pmatrix}
                =
                \begin{pmatrix}a_2&0\\0&a_1\end{pmatrix}
        \end{align}
        Which is only the case if $a_1 = a_2$. So we have
        $(A_F)_{J_F} \simeq \mathbb{C}$, whose unitary elements
        from $U(1)$ are contained in the diagonal subgroup of
        $U(\mathcal{A}_F)$.
    \end{proof}

    Now we need to find the exact from of the field $B_\mu$ to calculate the
    spectral action of a spectral triple. Since $(A_F)_{J_F} \simeq
    \mathbb{C}$ we find that $\mathfrak{h}(F) = \mathfrak{u}((A_F)_{J_F})
    \simeq i\mathbb{R}$.\newline

    An arbitrary hermitian field $A_\mu = -ia\partial _\mu b$  is given by
    two
    $U(1)$ Gauge fields $X_\mu^1, X_\mu^2 \in C^\infty(M, \mathbb{R})$.
    However $A_\mu$ appears in combination $A_\mu - J_F A_\mu J_F^{-1}$:
    \begin{align}
        B_\mu = A_\mu - J_F A_\mu J_F^{-1} =
        \begin{pmatrix}X_\mu^1&0\\0&X_\mu^2 \end{pmatrix}
            -
        \begin{pmatrix}X_\mu^2&0\\0&X_\mu^1 \end{pmatrix}
            =:
        \begin{pmatrix}Y_\mu&0\\0&-Y_\mu \end{pmatrix}=
            Y_\mu \otimes \gamma _F
    \end{align}
    where $Y_\mu$ the $U(1)$ Gauge field is defined as
    \begin{align}
        Y_\mu := X_\mu^1 - X_\mu^2 \in C^\infty(M, \mathbb{R}) = C^\infty(M,
        i\ u(1)).
    \end{align}

    \begin{proposition}
        The inner fluctuations of the almost-commutative manifold $M\times
        F_x$ described above are parametrized by a $U(1)$-gauge field $Y_\mu$
        as
        \begin{align}
            D \mapsto D' = D + \gamma ^\mu Y_\mu \otimes \gamma_F
        \end{align}
        The action of the gauge group $\mathfrak{B}(M\times F_X) \simeq
        C^\infty (M, U(1))$ on $D'$ is implemented by
        \begin{align}
            Y_\mu \mapsto Y_\mu - i\ u\partial_\mu u^*; \;\;\;\;\; (u\in
            \mathfrak{B}(M\times F_X)).
        \end{align}
    \end{proposition}

\section{Electrodynamics}
Now we use the almost commutative Manifold and the abelian gauge group
$U(1)$ to describe Electrodynamics. We arrive at a unified description of
gravity and electrodynamics although in the classical level.
\newline

The almost commutative Manifold $M\times F_X$ describes a local gauge group
$U(1)$. The inner fluctuations of the Dirac operator describe $Y_\mu$ the
gauge field of $U(1)$. There arrise two Problems:
\newline
(1): With $F_X$, $D_F$ must vanish, however this implies that the electrons
are massless (this we do not want)
\newline

(2): The Euclidean action for a free Dirac field is
\begin{align}
    S = - \int i \bar{\psi}(\gamma ^\mu\partial _\mu - m) \psi d^4x,
\end{align}
$\psi,\ \bar{\psi}$ must be considered as independent variables, which means
$S_F$ need two independent Dirac Spinors. We write $\{e, \bar{e}\}$ for the
ONB of $H_F$, where $\{e\}$ is the ONB of $H_F^+$ and $\{\bar{e}\}$ the ONB
of $H_F^-$ with the real structure this gives us the following relations
\begin{align}
    J_F e &= \bar{e} \;\;\;\;\;\; J_F \bar{e} = e \\
    \gamma_F e &= e  \;\;\;\;\;\;   \gamma_F \bar{e} = \bar{e}.
\end{align}
The total Hilbertspace is $H = L^2(S) \otimes H_F$, with $\gamma _F$ we can
decompose $L^2(S) = L^2(S)^+ \oplus L^2(S)^-$, so with $\gamma = \gamma _M
\otimes \gamma _F$ we can obtain the positive eigenspace $H^+$
\begin{align}
    H^+ = L^2(S)^+ \otimes H_F^+ \oplus L^(S)^- \otimes H_F^-.
\end{align}
For a $\xi \i H^+$ we can write
\begin{align}
    \xi = \psi _L \otimes e + \psi _R \otimes \bar{e}
\end{align}
where $\psi _L \in L^2(S)^+$ and $\psi _R \in L^2(S)^-$ are the two Wheyl
spinors. We denote that $\xi$ is only determined by one Dirac spinor $\psi :=
\psi_L + \psi _R$, \textbf{but we require two independent spinors}. This is
too much restriction for $F_X$.
\subsection{The Finite Space}
Here we solve the two problems by enlarging(doubling) the Hilbertspace. This
is done by introducing multiplicities in Krajewski Diagrams which will also
allow us to choose a nonzero Dirac operator which will connect the two
vertices (next chapter).
\newline

We start of with the same algebra $C^\infty(M, \mathbb{C}^2)$, corresponding
to space $N= M\times X \simeq M\sqcup M$.
\newline

The Hilbertspace will describe four particles,
\begin{itemize}
    \item left handed electrons
    \item right handed positrons
\end{itemize}
Thus we have $\{ \underbrace{e_R, e_L}_{\text{left-handed}},
\underbrace{\bar{e}_R, \bar{e}_L}_{\text{right-handed}}\}$ the ONB for $H_F
\mathbb{C}^4$.
\newline
Then with $J_F$ we interchange particles with antiparticles we have the
following properties
\begin{align}
    &J_F e_R = \bar{e}_R \;\;\;\;\; &J_F e_L = \bar{e_L} \\
    &\gamma _F e_R = -e_R \;\;\;\;\; &\gamma_F e_L = e_L \\
    \text{and}& \nonumber \\
    &J_F^2 = 1 \;\;\;\;\; & J_F \gamma_F  = - \gamma_F J_F
\end{align}
This corresponds to KO-dim$= 6$. Then $\gamma_F$ allows us to can decompose
$H$
\begin{align}
    H_F = \underbrace{H_F^+}_{\text{ONB } \{e_L, \bar{e}_L\}}
    \oplus \underbrace{H_F^-}_{\text{ONB } \{e_R, \bar{e}_R\}}.
\end{align}
Alternatively we can decompose $H$ into the eigenspace of particles and their
antiparticles (electrons and positrons) which we will use going further.
\begin{align}
    H_F = \underbrace{H_{e}}_{\text{ONB } \{e_L, e_R\}} \oplus
    \underbrace{H_{\bar{e}}}_{\text{ONB } \{\bar{e}_L, \bar{e}_R\}}
\end{align}
Now the action of $a \in A = \mathbb{C}^2$ on $H$ with respect to the ONB
$\{e_L, e_R, \bar{e}_L, \bar{e}_R\}$ is represented by
\begin{align}
    a =
    \begin{pmatrix}a_1 & a_2 \end{pmatrix} \mapsto
        \begin{pmatrix}
            a_1 &0 &0 &0\\
             0&a_1 &0 &0\\
            0 &0 &a_2 &0\\
            0 &0 &0 &a_2\\
        \end{pmatrix}
\end{align}
Do note that this action commutes wit the grading and that
$[a, b^\circ] = 0$ with $b:= J_F b^*J_F$  because both the left and the right
action is given by diagonal matrices.
\begin{proposition}
    The data
    \begin{align}
        \left( \mathbb{C}^2, \mathbb{C}^2, D_F=0; J_F =
        \begin{pmatrix}
            0 & C \\ C &0
        \end{pmatrix},
        \gamma _F =
        \begin{pmatrix}
            1 & 0 \\ 0 &-1
        \end{pmatrix}
        \right)
    \end{align}
    defines a real even spectral triple of KO-dimension 6.
\end{proposition}
This spectral triple can be represented in the following Krajewski diagram,
with two nodes of multiplicity two
    \begin{figure}[h!] \centering
    \begin{tikzpicture}[
        dot/.style = {draw, circle, inner sep=0.06cm},
        bigdot/.style = {draw, circle, inner sep=0.09cm},
        no/.style = {},
        ]
        \node[no](a) at (0,0) [label=left:$\textbf{1}^\circ$] {};
        \node[no](b) at (0, -1) [label=left:$\textbf{1}^\circ$] {};
        \node[no](c) at (0.5, 0.5) [label=above:$\textbf{1}$] {};
        \node[no](d) at (1.5, 0.5) [label=above:$\textbf{1}$] {};
        \node[dot](d0) at (1.5,0) [] {};
        \node[dot](d0) at (0.5,-1) [] {};
        \node[bigdot](d0) at (1.5,0) [] {};
        \node[bigdot](d0) at (0.5,-1) [] {};
        \end{tikzpicture}
    \end{figure}
\subsection{A noncommutative Finite Dirac Operator}
Add a non-zero Dirac Operator to $F_{ED}$. From the Krajewski Diagram, we see
that edges only exist between the multiple vertices. So we construct a Dirac
operator mapping between the two vertices.
\begin{align}
    D_F =
    \begin{pmatrix}
    0 & d & 0 & 0 \\
    \bar{d} & 0 & 0 & 0 \\
    0 & 0 & 0 & \bar{d} \\
    0 & 0 & d & 0
    \end{pmatrix}
\end{align}
We can now consider the finite space $F_{ED}$.
\begin{align}
    F_{ED} := (\mathbb{C}^2, \mathbb{C}^4, D_F; J_F, \gamma_F)
\end{align}
where $J_F$ and $\gamma_F$ like before, $D_F$ like above.
\subsection{The almost-commutative Manifold}
The almost commutative manifold $M\times F_{ED}$ has KO-dim$=2$, it is the
following spectral triple
\begin{align}
    M\times F_{ED} := \left(C^\infty(M,\mathbb{C}^2, L^2(S)\otimes
    \mathbb{C}^4,
    D_M\otimes 1 +\gamma _M \otimes D_F; J_M\otimes J_F, \gamma_M\otimes
    \gamma _F\right)
\end{align}

The algebra decomposition is like before
\begin{align}
    C^\infty(M, \mathbb{C}^2) = C^\infty (M) \oplus C^\infty (M)
\end{align}

The Hilbertspace decomposition is
\begin{align}
    H = (L^2(S) \otimes H_e ) \oplus (L^2(S) \otimes H_{\bar{e}}).
\end{align}
Here we have the one component of the algebra acting on $L^2(S) \otimes H_e$,
and the other one acting on $L^2(S) \otimes H_{\bar{e}}$
\newline

The derivation of the gauge theory is the same for $F_{ED}$ as for $F_X$, we
have $\mathfrak{B}(F) \simeq U(1)$ and for $B_\mu = A_\mu - J_F A_\mu
J_F^{-1}$
\begin{align}
    B_\mu =
    \begin{pmatrix}
        Y_\mu & 0 & 0 & 0 \\
        0 & Y_\mu& 0 & 0 \\
        0 & 0 & Y_\mu& 0 \\
        0 & 0 & 0 & Y_\mu
    \end{pmatrix} \;\;\;\;\;\ \text{for} \;\;\ Y_\mu (x) \in \mathbb{R}.
\end{align}
We have one single $U(1)$ gauge field $Y_\mu$, carrying the action of the
gauge group
\begin{align}
   \text{$\mathfrak{B}$}(M\times F_{ED}) \simeq C^\infty(M, U(1))
\end{align}

Our space $N = M\times X \simeq M\sqcup M$ consists of two compies of $M$.
If $D_F = 0$ we have infinite distance of the two copies. Now we have $D_F$
nonzero but the $[D_F, a] = 0$ $\forall a \in A$ which still yields infinite
distance.
\begin{question}
    What does this imply (physically, mathematically)? Why can we continue
    even thought we have infinite distance between the same manifold? What do
    we get if we fix this?
\end{question}

\end{document}
