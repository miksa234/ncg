\documentclass[a4paper]{article}

\usepackage[T1]{fontenc}
\usepackage[utf8]{inputenc}

\usepackage{mathptmx}

\usepackage{subcaption}
\usepackage[shortlabels]{enumitem}
\usepackage{amsmath,amssymb}
\usepackage{amsthm}
\usepackage{bbm}
\usepackage{graphicx}
\usepackage[colorlinks=true,naturalnames=true,plainpages=false,pdfpagelabels=true]{hyperref}
\usepackage[parfill]{parskip}

\usepackage{tikz}
\usetikzlibrary{patterns,decorations.pathmorphing,positioning}

\usepackage[framemethod=TikZ]{mdframed}

\tikzstyle{titlered} =
    [draw=black, thick, fill=white,%
        text=black, rectangle,
        right, minimum height=.7cm]

\newcounter{exercise}

\renewcommand*\theexercise{Exercise~\arabic{exercise}}

\makeatletter
\mdfdefinestyle{exercisestyle}{%
    outerlinewidth=1em,%
    outerlinecolor=white,%
    leftmargin=-1em,%
    rightmargin=-1em,%
    middlelinewidth=1.2pt,%
    roundcorner=5pt,%
    linecolor=black,%
    backgroundcolor=blue!5,
    innertopmargin=1.2\baselineskip,
    skipabove={\dimexpr0.5\baselineskip+\topskip\relax},
    skipbelow={-1em},
    needspace=3\baselineskip,
    frametitlefont=\sffamily\bfseries,
    settings={\global\stepcounter{exercise}},
    singleextra={%
        \node[titlered,xshift=1cm] at (P-|O) %
            {~\mdf@frametitlefont{\theexercise}~};},%
    firstextra={%
            \node[titlered,xshift=1cm] at (P-|O) %
                    {~\mdf@frametitlefont{\theexercise}~};},
}
\makeatother

\newenvironment{MyExercise}%
{\begin{mdframed}[style=exercisestyle]}{\end{mdframed}}

\theoremstyle{definition}
\newtheorem{definition}{Definition}

\theoremstyle{definition}
\newtheorem{question}{Question}

\theoremstyle{definition}
\newtheorem{example}{Example}

\theoremstyle{theorem}
\newtheorem{theorem}{Theorem}

\theoremstyle{theorem}
\newtheorem{lemma}{Lemma}

\newtheorem*{idea}{Proof Idea}


\title{University of Vienna\\ Faculty of Physics\\ \vspace{1.25cm}
Notes on\\ Noncommutative Geometry and Particle Phyiscs}
\author{Milutin Popovic \\ Supervisor: Dr. Lisa
Glaser}
\date{Week 2: 12.02 - 19.02}

\begin{document}

    \maketitle
    \tableofcontents
    \newpage

\section{Noncommutative Geometric Spaces}
\subsection{Noncommutative Matrix Algebras}
\subsubsection{Balanced Tensor Product and Hilbert Bimodules}

\begin{definition}
    Let $A$ be an algebra, $E$ be a \textit{right} $A$-module and $F$ be a \textit{left} $A$-module.
    The \textit{balanced tensor product} of $E$ and $F$ forms a $A$-bimodule.
    \begin{align}
        E \otimes _A F := E \otimes F / \left\{\sum _i e_i a_i \otimes f_i - e_i \otimes a_i f_i : \;\;\;
                                         a_i \in A,\ e_i \in E,\ f_i \in F \right\}
    \end{align}
\end{definition}
Note $/$ denotes the quotient space. So $\otimes _A$ takes two left/right modules and makes a
bimodule with the help the tensor product of the two modules and the quotient space that takes
out all the elements from the tensor product that dont preserver the left/right representation and that
are duplicates.
\begin{definition}
    Let $A$, $B$ be \textit{matrix algebras}. The \textit{Hilbert bimodule} for $(A, B)$ is given by
    \begin{itemize}
        \item $E$, an $A$-$B$-bimodue $E$ and by
        \item an $B$-valued \textit{inner product} $\langle \cdot,\cdot\rangle_E: E\times E \rightarrow B$
    \end{itemize}
$\langle \cdot,\cdot\rangle_E$ needs to satisfy the following for $e, e_1, e_2 \in E,\ a \in A$ and $b \in B$.
\begin{align}
    \langle e_1, a\cdot e_2\rangle_E &= \langle a^*\cdot e_1, e_2\rangle_E \;\;\;\; & \text{sesquilinear in $A$}\\
    \langle e_1, e_2 \cdot b\rangle_E &= \langle e_1, e_2\rangle_E b \;\;\;\; & \text{scalar in $B$} \\
    \langle e_1, e_2\rangle_E &= \langle e_2,e_1\rangle^*_E \;\;\;\; & \text{hermitian} \\
    \langle e, e\rangle_E &\ge 0 \;\;\;\; & \text{equality holds iff $e=0$}
\end{align}

\end{definition}

We denote $KK_f(A,B)$ the set of all \textit{Hilbert bimodules} of $(A,B)$.

\begin{MyExercise}
    \textbf{
    Check that a representation $\pi:\ A \ \rightarrow L(H)$ of a matrix algebra $A$ turns $H$ into
    a Hilbert bimodule for $(A, \mathbb{C})$.
    \label{ex: bimodule}
}\newline


    We check if the representation of $a \in A$, $\pi(a)=T \in L(H)$ fulfills
    the conditions on the $\mathbb{C}$-valued inner product for $h_1, h_2 \in H$:
    \begin{itemize}
        \item $\langle h_1, \pi(a) h)2\rangle _\mathbb{C} = \langle h_1, T h_2\rangle _\mathbb{C} =
            \langle T^* h_1, h_2\rangle _\mathbb{C}$, $T^*$ given by the adjoint
        \item $\langle h_1, h_2 \pi(a)\rangle _\mathbb{C} = \langle h_1, h_2 T\rangle _\mathbb{C} = \langle h_1, h_2\rangle _\mathbb{C}$, $T$ acts from the left
        \item $\langle h_1, h_2\rangle _\mathbb{C}^* = \langle h_2,h_1\rangle _\mathbb{C}$, hermitian because of the
            $\mathbb{C}$-valued inner product
        \item $\langle h_1, h_2\rangle  \ge 0$, $\mathbb{C}$-valued inner product.
    \end{itemize}
\end{MyExercise}

\begin{MyExercise}
    \textbf{
    Show that the $A-A$ bimodule given by $A$ is in $KK_f(A,A)$ by taking the following inner product
    $\langle \cdot,\cdot\rangle_A:A \times A \rightarrow A$:
    \begin{align}
        \langle a, a\rangle_A = a^*a' \;\;\;\; a,a'\in A
    \end{align}
    \label{exercise: inner-product}
}\newline


    We check again the conditions on $\langle \cdot, \cdot\rangle _A$, let $a, a_1, a_2 \in A$:
    \begin{itemize}
        \item $\langle a_1, a\cdot a_2\rangle _A = a^*\ a\cdot a_2 = (a^*a_1)^* a_2 = \langle  a^*a_1, a_2\rangle  $
        \item $\langle a_1, a_2 \cdot a\rangle _A = a^*_1 (a_2\cdot a) = (a^*a_2)\cdot a = \langle a_1, a_2\rangle _A a$
        \item $\langle a_1, a_2\rangle _A^* = (a_1^* a_2)^* = a_2^*(a_1^*)^* = a_2^* a_1 = \langle a_2, a_1\rangle $
    \end{itemize}
\end{MyExercise}

\begin{example}
    Consider a $*$ homomorphism between two matrix algebras $\phi:A\rightarrow B$.
    From it we can construct a Hilbert bimodule $E_{\phi} \in KK_f(A, B)$ in the following way.
    We let $E_{\phi}$ be $B$ in the vector space sense and an inner product from the above
    Exercise \ref{exercise: inner-product}, with $A$ acting on the left with $\phi$.
    \begin{align}
        a\cdot b = \phi(a)b \;\;\;\; a\in A, b\in E_{\phi}
    \end{align}
\end{example}



\subsubsection{Kasparov Product and Morita Equivalence}
\begin{definition}
    Let $E \in KK_f(A, B)$ and $F \in KK_F(B, D)$ the \textit{Kasparov product} is defined as
    with the balanced tensor product
    \begin{align}
        F \circ E := E \otimes _B F
    \end{align}
    Such that $F\circ E \in KK_f(A,D)$ with a $D$-valued inner product.
    \begin{align}
        \langle e_1 \otimes f_1, e_2 \otimes f_2\rangle _{E\otimes _B F} = \langle f_1,\langle e_1, e_2\rangle _E f_2\rangle _F
    \end{align}
\end{definition}

\begin{question}
 How do we go from $\langle e_1 \otimes f_1, e_2 \otimes f_2\rangle _{E\otimes _B F}$ to $
    \langle f_1,\langle e_1, e_2\rangle _E f_2\rangle _F$ \label{q: tensorproduct}\\
    This statement is still in the definition.
\end{question}

%\begin{question}
%What is the meaning of `associative up to isomorphism'? Isomorphism of $F \circ E$ or of $A, B$ or $D$?
%\end{question}

    \begin{MyExercise}
        \textbf{
    Show that the association $\phi \leadsto E_\phi$ (from the previous Example) is natural
    in the sense
    \begin{enumerate}
        \item $E_{\text{id}_A} \simeq A \in KK_f(A,A)$
        \item for $*$-algebra homomorphism $\phi: A \rightarrow B$ and $\psi: B \rightarrow C$ we have
            an isomorphism
            \begin{align}
                E_{\psi} \circ E_{\phi}\ \equiv\ E_{\phi} \otimes _B E_{\psi}\ \simeq\
                E_{\psi \circ \phi} \in KK_f(A,C)
            \end{align}
    \end{enumerate}
}
    \begin{enumerate}
        \item $\text{id}_A: A \rightarrow A$.\\
            To construct $E_{\phi}\in KK_f(A,A)$, we let $E_{\phi}$ be $A$ with a natural right
            representation, so $\Rightarrow E_{\phi}\simeq A$.\\
            With an inner product, acting on $A$ from the left with $\phi$, $a', a\in A$\\
            $a'a = (\phi(a') a) \in A $, which is satisfied by $\text{id}_A$, so $\phi = \text{id}_A$.
        \item $a \cdot b \cdot c = \psi(\phi (a) \cdot b) \cdot c$ for $a \in A$, $b\in B$, and $c\in C$
                which is $\psi \circ \phi$
    \end{enumerate}
\end{MyExercise}

\begin{MyExercise}
    \textbf{
    In the definition of Morita equivalence:
    \begin{enumerate}
        \item Check that $E \otimes _B F$ is a $A-D$ bimodule
        \item Check that $\langle \cdot,\cdot\rangle _{E\oplus _B F}$ defines a $D$ valued inner product
        \item Check that $\langle a^*(e_1 \otimes f_1), e_2 \otimes f_2\rangle _{E \otimes _B F} = \langle e_1 \otimes f_1, a(e_2 \otimes f_2)\rangle _{E \otimes _B F}$.
    \end{enumerate}
}
    \begin{enumerate}
        \item $E \otimes _B F = E \otimes F / \{\sum_i e_i b_i \otimes f_i - e_i \otimes b_i f_i;
            e_i \in E_i, b_i \in B, f_i \in F\}$ the last part takes out all tensor product elements of
            $E$ and $F$ that don't preserver the left/right representation and that are duplicates.
        \item $\langle e_1, e_2\rangle _E \in B$ and $\langle f_1, f_2\rangle _F \in C$ by definition. So let $\langle e_1, e_2\rangle _E =b$. \\
            Then $\langle e_1 \otimes f_1, e_2 \otimes f_2\rangle _{E\otimes _B F} = \langle f_1, \langle e_1, e_2\rangle _E f_2\rangle _F =
            \langle f_1, b f_2\rangle _F \in C$
        \item Check Question \ref{q: tensorproduct}.\\
            But let $G := E\otimes _B F \in KK_f(A,C)$ then $\forall g_1, g_2 \in G$ and $a \in A$ we need
            by definition $\langle g_1, ag_2\rangle _G = \langle a^*g_1, g_2\rangle _G$ and we set $g_1 = e_1 \otimes f_1$ and
            $g_2 = e_2 \otimes f_2$ for some $e_1, e_2 \in E$ and $f_1, f_2 \in F$, or else
            $G \notin KK_f(A,C)$ which would violate the Kasparov product
    \end{enumerate}
    \end{MyExercise}

\begin{definition}
    Let $A$, $B$ be \textit{matrix algebras}. They are called \textit{Morita equivalent} if there
    exists an $E \in KK_f(A, B)$ and an $F \in KK_f(B, A)$ such that:
    \begin{align}
        E \otimes _B F \simeq A \;\;\; \text{and} \;\;\; F \otimes _A E \simeq B
    \end{align}
    Where $\simeq$ denotes the isomorphism between Hilbert bimodules, note that $A$ or $B$ is a bimodule by
    itself.
\end{definition}

\begin{question}
    Why are $E$ and $F$ each others inverse in the Kasparov Product? \\
    They are each others inverse with respect to the Kasparov Product because we land in the same space as we started.
    In the definition we have $E \in KK_f(A, B)$ we start from $A$ and $E \otimes _B F$ lands in $A$.\\
    On the other hand we have $F \in KK_f(B, D)$ we start from $B$ and $F \otimes _A E$ lands in $B$.
\end{question}

\begin{example}
    \
    \begin{itemize}
        \item Hilber bimodule of $(A,A)$ is $A$
        \item Let $E \in KK_f(A,B)$, we take $E \circ A = A\oplus _A E \simeq E$
        \item we conclude, that $_A A_A$ is the identity in the Kasparov product (up to isomorphism)
    \end{itemize}
\end{example}

\begin{example}
    Let $E = \mathbb{C}^n$, which is a $(M_n(\mathbb{C}), \mathbb{C})$ Hilbert bimodule with the
    standard $\mathbb{C}$ inner product.\\
    On the other hand let $F = \mathbb{C}^n$, which is a $(\mathbb{C}, M_n(\mathbb{C}))$ Hilbert
    bimodule by right matrix multiplication with $M_n(\mathbb{C})$ valued inner product:
    \begin{align}
        \langle v_1, v_2\rangle =\bar{v_1}v_2^t \;\; \in M_n(\mathbb{C})
    \end{align}
    Now we take the Kasparov product of $E$ and $F$:
   \begin{itemize}
        \item $F\circ E\ =\  E\otimes _{\mathbb{C}}F\ \;\;\;\;\;\; \simeq \  M_n(\mathbb{C})$
        \item $E\circ F\ =\ F\otimes _{M_n(\mathbb{C})}E\ \simeq\ \mathbb{C}$
    \end{itemize}
    $M_n(\mathbb{C})$ and $\mathbb{C}$ are Morita equivalent
\end{example}

\begin{theorem}
    Two matrix algebras are Morita Equivalent iff their their Structure spaces
    are isomorphic as discreet spaces (have the same cardinality / same number of elements)
\end{theorem}
\begin{proof}
    Let $A$, $B$ be \textit{Morita equivalent}. So there exists $_A E_B$ and $_B F_A$ with
    \begin{align}
        E \otimes _B F \simeq A \;\;\; \text{and} \;\;\; F \otimes _A E \simeq B
    \end{align}
    Consider $[(\pi _B, H)] \in \hat{B}$ than we construct a representation of $A$,
    \begin{align}
        \pi _A \rightarrow L(E \otimes _B H)\;\;\; \text{with} \;\;\; \pi _A(a) (e \otimes v) = a e \otimes w
    \end{align}
    \begin{question}
        Is $E \simeq H$ and $F \simeq W$? \\
        Not in particular, there is a theorem that all infinite dimensional Hilbert spaces are isomorphic.
        Here we are looking at finite dimensional Hilbert spaces.\\
        Another thing to is that $[\pi _B, H] \in \hat{B}$ and looking at Exercise \ref{ex: bimodule}
        we know that $H$ is a bimodule of $B$, hence $E \otimes _B H\simeq A$, and for $[\pi _A, W]$
        the same.
    \end{question}
    \textit{vice versa}, consider $[(\pi _A, W)] \in \hat{A}$ we can construct $\pi _B$
    \begin{align}
        \pi _B: B \rightarrow L(F \otimes _A W) \;\;\; \text{and}\;\;\; \pi _B(b) (f\otimes w) = bf\otimes w
    \end{align}
    These maps are each others inverses, thus $\hat{A} \simeq \hat{B}$
\end{proof}

\begin{MyExercise}
    \textbf{
    Fill in the gaps in the above proof:
    \begin{enumerate}
        \item show that the representation of $\pi _A$ defined is irreducible iff $\pi _B$ is.
        \item Show that the association of the class $[\pi _A]$ to $[\pi _B]$ is independent
            of the choice of representatives $\pi _A$ and $\pi _B$
    \end{enumerate}
}

    \begin{enumerate}
        \item $(\pi _B, H)$ is irreducible means $H \neq \emptyset$ and only $\emptyset$ or $H$
            is invariant under the Action of $B$ on $H$.
            Than $E\otimes _B H$ cannot be empty, because also $E$ preserves left representation of $A$
            and also $E\otimes _B H \simeq A$.
        \item The important thing is that $[\pi _A] \in \hat{A}$ respectively $[\pi _B] \in \hat{B}$,
            hence any choice of representation is irreducible, because the structure space denotes all unitary
            equivalence classes of irreducible representations.
    \end{enumerate}
\end{MyExercise}

    \begin{lemma}
    The matrix algebra $M_n(\mathbb{C})$ has a unique irreducible representation (up to isomorphism)
    given by the defining representation on $\mathbb{C}^n$.
\end{lemma}
\begin{proof}
    We know $\mathbb{C}^n$ is a irreducible representation of $A= M_n(\mathbb{C})$. Let $H$ be irreducible
    and of dimension $k$, then we define a map
    \begin{align}
        \phi : A\oplus...\oplus A &\rightarrow H^* \\
        (a_1,...,a_k)             &\mapsto e^1\circ a_1^t+...+e^k\circ a_k^t
    \end{align}
    With $\{e^1,...,e^k\}$ being the basis of the dual space $H^*$ and $(\circ)$ being the pre-composition
    of elements in $H^*$ and $A$ acting on $H$. This forms a morphism of $M_n(\mathbb{C})$ modules,
    provided a matrix $a \in A$ acts on $H^*$ with $v\mapsto v\circ a^t$ ($v\in H^*$).
    Furthermore this morphism is surjective, thus making the pullback $\phi ^*:H\mapsto (A^k)^*$ injective.
    Now identify $(A^k)^*$ with $A^k$ as a $A$-module and note that
    $A=M_n(\mathbb{C}) \simeq \oplus ^n \mathbb{C}^n$ as a n A module.
    It follows that $H$ is a submodule of $A^k \simeq \oplus ^{nk}\mathbb{C}$. By irreducibility
    $H \simeq \mathbb{C}$.
\end{proof}

\begin{example}
    Consider two matrix algebras $A$, and $B$.
    \begin{align}
        A = \bigoplus ^N_{i=1} M_{n_i}(\mathbb{C}) \;\;\; B = \bigoplus ^M_{j=1} M_{m_j}(\mathbb{C})
    \end{align}
    Let $\hat{A} \simeq \hat{B}$ that implies $N=M$ and define $E$ with $A$ acting by block-diagonal
    matrices on the first tensor and B acting in the same way on the second tensor. Define $F$ vice versa.
    \begin{align}
        E:= \bigoplus _{i=1}^N \mathbb{C}^{n_i} \otimes \mathbb{C}^{m_i} \;\;\;
        F:= \bigoplus _{i=1}^N \mathbb{C}^{m_i} \otimes \mathbb{C}^{n_i}
    \end{align}
    Then we calculate the Kasparov product.
    \begin{align}
        E \otimes _B F &\simeq \bigoplus _{i=1}^N (\mathbb{C}^{n_i}\otimes\mathbb{C}^{m_i})
            \otimes _{M_{m_i}(\mathbb{C})} (\mathbb{C}^{m_i}\otimes\mathbb{C}^{n_i}) \\
                       &\simeq \bigoplus _{i=1}^N \mathbb{C}^{n_i}\otimes
                       \left(\mathbb{C}^{m_i}\otimes _{M_{m_i}(\mathbb{C})}\mathbb{C}^{m_i}\right)
                        \oplus \mathbb{C}^{n_i} \\
                       &\simeq \bigoplus _{i=1}^N \mathbb{C}^{m_i}\otimes\mathbb{C}^{n_i} \simeq A
    \end{align}
    and from $F \otimes _A E \simeq B$.
\end{example}

We conclude that.
\begin{itemize}
    \item There is a duality between finite spaces and Morita equivalence classes of matrix algebras.
    \item By replacing $*$-homomorphism $A\rightarrow B$ with Hilbert bimodules $(A,B)$ we introduce
        a richer structure of morphism between matrix algebras.
\end{itemize}

\end{document}
