\documentclass[a4paper]{article}

\usepackage[T1]{fontenc}
\usepackage[utf8]{inputenc}

\usepackage{mathptmx}

\usepackage{subcaption}
\usepackage[shortlabels]{enumitem}
\usepackage{amsmath,amssymb}
\usepackage{amsthm}
\usepackage{bbm}
\usepackage{graphicx}
\usepackage[colorlinks=true,naturalnames=true,plainpages=false,pdfpagelabels=true]{hyperref}
\usepackage[parfill]{parskip}

\usepackage{tikz}
\usetikzlibrary{patterns,decorations.pathmorphing,positioning}

\usepackage[framemethod=TikZ]{mdframed}

\tikzstyle{titlered} =
    [draw=black, thick, fill=white,%
        text=black, rectangle,
        right, minimum height=.7cm]

\newcounter{exercise}

\renewcommand*\theexercise{Exercise~\arabic{exercise}}

\makeatletter
\mdfdefinestyle{exercisestyle}{%
    outerlinewidth=1em,%
    outerlinecolor=white,%
    leftmargin=-1em,%
    rightmargin=-1em,%
    middlelinewidth=1.2pt,%
    roundcorner=5pt,%
    linecolor=black,%
    backgroundcolor=blue!5,
    innertopmargin=1.2\baselineskip,
    skipabove={\dimexpr0.5\baselineskip+\topskip\relax},
    skipbelow={-1em},
    needspace=3\baselineskip,
    frametitlefont=\sffamily\bfseries,
    settings={\global\stepcounter{exercise}},
    singleextra={%
        \node[titlered,xshift=1cm] at (P-|O) %
            {~\mdf@frametitlefont{\theexercise}~};},%
    firstextra={%
            \node[titlered,xshift=1cm] at (P-|O) %
                    {~\mdf@frametitlefont{\theexercise}~};},
}
\makeatother

\newenvironment{MyExercise}%
{\begin{mdframed}[style=exercisestyle]}{\end{mdframed}}

\theoremstyle{definition}
\newtheorem{definition}{Definition}

\theoremstyle{definition}
\newtheorem{question}{Question}

\theoremstyle{definition}
\newtheorem{example}{Example}

\theoremstyle{theorem}
\newtheorem{theorem}{Theorem}

\theoremstyle{theorem}
\newtheorem{lemma}{Lemma}

\newtheorem*{idea}{Proof Idea}

\date{Week 6: 19.03 - 26.03}

\title{University of Vienna\\ Faculty of Physics\\ \vspace{1.25cm}
Notes on\\ Noncommutative Geometry and Particle Phyiscs}
\author{Milutin Popovic \\ Supervisor: Dr. Lisa
Glaser}
\date{Week 6: 19.03 - 26.03}


\begin{document}

    \maketitle
    \tableofcontents
    \newpage
\section{Finite Real Noncommutative Spaces}
\subsection{Finite Real Spectral Triples}
Add on to finite real spectral triples a \textit{real structure}. The
requirement is that $H$ is a $A$-$A$-bimodule (before only a $A$-left
module).
\newline

For this we introduce a $\mathbb{Z}_2$-grading $\gamma$ with
\begin{align}
    &\gamma ^* = \gamma \\
    &\gamma ^2 = 1 \\
    &\gamma D = - D \gamma\\
    &\gamma a = a \gamma \;\;\;\; a\in A
\end{align}

\begin{definition}
    A \textit{finite real spectral triple} is given by a finite spectral
    triple $(A, H, D)$ and a anti-unitary operator $J:H\rightarrow H$ called
    the \textit{real structure}, such that
    \begin{align}
        a^\circ := J a^* J^{-1}
    \end{align}
    is a right representation of $A$ on $H$, that is $(ab)^\circ = b^\circ
    a^\circ$. With two requirements
    \begin{align}
        &[a, b^\circ] = 0\\
        &[[D, a],b^\circ] = 0.
    \end{align}
    They are called the \textit{commutant property}, and mean that the left
    action of an element in $A$ and $\Omega _D^1(A)$ commutes with the right
    action on $A$.
\end{definition}
\begin{definition}
    The $KO$-dimension of a real spectral triple is determined by the sings
    $\epsilon, \epsilon ' ,\epsilon '' \in \{-1, 1\}$ appearing in
    \begin{align}
        &J^2 = \epsilon \\
        &JD = \epsilon \ DJ\\
        &J\gamma = \epsilon '' \gamma J.
    \end{align}
\end{definition}
\begin{table}[h!]
    \centering
    \caption{$KO$-dimension $k$ modulo $8$ of a real spectral triple}
    \begin{tabular}{ c | c c c c c c c c}
        \hline
        $k$        & 0 & 1 & 2 & 3 & 4 & 5 & 6 & 7 \\
        \hline
     $\epsilon$    & 1 & 1 & -1 & -1 & -1 & -1 & 1 & 1 \\
     $\epsilon '$  & 1 & -1 & 1 & 1 & 1 & -1 & 1 & 1 \\
     $\epsilon ''$ & 1 &  & -1 &  & 1 &  & -1 &  \\
     \hline
    \end{tabular}
\end{table}


\begin{definition}
An opposite-algebra $A^\circ$ of a $A$ is defined to be equal to $A$ as a
vector space with the opposite product
\begin{align}
    &a\circ b := ba\\
    &\Rightarrow a^\circ = Ja^* J^{-1} \;\;\; \text{defines the left
    representation of $A^\circ$ on $H$}
\end{align}
\end{definition}


\begin{example}
    Matrix algebra $M_N(\mathbb{C})$ acting on $H=M_N(\mathbb{C})$ by left
    matrix multiplication with the Hilbert Schmidt inner product.
    \begin{align}
        \langle a , b \rangle = \text{Tr}(a^* b)
    \end{align}
    Then we define $\gamma (a) = a$ and $J(a) = a^*$ with $a\in H$.
    Since $D$ mus be odd with respect to $\gamma$ it vanishes identically.
\end{example}

\begin{definition}
    We call $\xi \in H$ \textbf{cyclic vector} in $A$ if:
    \begin{align}
        A\xi := { a\xi:\;\; a\in A} = H
    \end{align}

    We call $\xi \in H$ \textbf{separating vector} in $A$ if:
    \begin{align}
        a\xi = 0\;\; \Rightarrow \;\; a=0;\;\;\; a\in A
    \end{align}
\end{definition}

\begin{MyExercise}
    \textbf{
        In the previous example, show that the right action on $M_N(\mathbb{C})$
    on $H = M_N(\mathbb{C})$ as defined by $a \mapsto a^\circ$
    is given by right matrix multiplication.
}\newline

    \begin{align}
        a^\circ \xi = J a^* J^{-1}\xi = Ja^* \xi^* = J\xi a=\xi^* a
    \end{align}
\end{MyExercise}
\begin{MyExercise}
    \textbf{
        Let $A= \bigoplus _i M_{n_i}(\mathbb{C})$, represented on $H = \bigoplus_i \mathbb{C}^{n_i}
        \otimes \mathbb{C}^{m_i}$, meaning that the irreducible representation $\textbf{n}_i$ has
        multiplicity $m_i$.
        \begin{enumerate}
            \item Show that the commutant $A'$ of $A$ is $A'\simeq \bigoplus_i M_{m_i} (\mathbb{C})$. As a consequence show $A'' \simeq A$.
            \item Show that if $\xi$ is a separating vector for $A$ than it is cyclic for $A'$.
        \end{enumerate}
    }\newline


    \begin{enumerate}
        \item We know the multiplicity space is $V_i = \mathbb{C}^{m_i}$. We know that
            for $T\in H$ and
            $a\in A'$ to work we need $aT=Ta$ by laws of matrix multiplication we need
            $A' \simeq \oplus _i M_{m_i}(\mathbb{C})$ for this to work since $H = \bigoplus_i
            \mathbb{C}^{n_i}
        \otimes \mathbb{C}^{m_i}$

        \item Suppose $\xi$ is cyclic for $A$ then $A'\xi = \{0\}$. Under the action of $A$ we
            then have $A'A\xi = AA' \xi = 0 \Rightarrow A' = 0$.\\
            Suppose now $\xi$ is separating for $A'$, we have $A'\xi = \{0\}$. We can define a
            projection in $A'$, $A\xi = P'$. With this projection we have $(1-P')\xi = 0
            \Rightarrow 1-P' = 0 \Rightarrow A\xi = H$.
    \end{enumerate}
\end{MyExercise}
\begin{MyExercise}
    \textbf{ Suppose $(A, H, D = 0)$ is a finite spectral triple such that $H$ possesses a
        cyclic and separating vector for $A$.
        \begin{enumerate}
            \item Show that the formula $S(a \xi) = a* \xi$ defines a anti-linear operator\\
                $S: H \rightarrow H$.
            \item Show that $S$ is invertible
            \item Let $J: H \rightarrow H$ be the operator in $S = J \Delta ^{1/2}$ with
                $\Delta = S*S$. Show that $J$ is anti-unitary
        \end{enumerate}
    }\newline


    \begin{enumerate}
    \item By composition $S(a\xi) = a*\xi$ this is literally anti-linearity. Does this mean
        $S\xi = \xi$?
    \item Let $\xi \in H$ be cyclic then: $S(A\xi) = A*\xi = A\xi = H$. The same has to work
        for $S^{-1}$ if not then $\xi$ wouldn't exist. $S^{-1}(A*\xi) = S^{-1}(H) = H$.
    \item Since $S$ is bijective then $\Delta ^{1/2}$ and $J$ need to be bijective.\\
        Now let $\xi _1 , \xi _2 \in H$.\\
        \begin{align}
            <J \xi _1 , J \xi _2 > &= < J^*J\xi_1 , \xi_2>^* =\\
            &= <(\Delta ^{1/2})^* S^* S \Delta ^{1/2} \xi_1, \xi_2>^* = \\
            &= <(SS^*)^{1/2}S^*S(SS^*) \xi_1, \xi_2>^* =\\
            &= <(SS^*SS^*)^{1/2} \xi_1, \xi_2>^* = \\
            &= <\xi _1, \xi_2>^* = <\xi_2 , \xi_1>.
        \end{align}
    \end{enumerate}
\end{MyExercise}
\subsection{Morphisms Between Finite Real Spectral Triples}
Extend unitary equivalence of finite spectral triples to real ones (with $J$
and $\gamma$)

\begin{definition}
    We call two finite real spectral triples $(A_1, H_1 ,D_1 ; J_1 , \gamma
    _1)$ and $(A_2, H_2, D_2; J_2, \gamma _2)$ unitarily equivalent if $A_1 =
    A_2$ and if there exists a unitary operator $U: H_1 \rightarrow H_2$ such
    that
    \begin{align}
        &U\pi_1(a) U^* = \pi _2(a)\\
        &UD_1U^*=D_2\\
        &U\gamma _1 U^* = \gamma _2\\
        &UJ_1 U^* = J_2
    \end{align}
\end{definition}
\begin{definition}
    Let $E$ be a $B$-$A$ bimodule. The \textit{conjugate Module} $E^\circ$ is
    given by the $A$-$B$-bimodule.
    \begin{align}
        E^\circ = \{\bar{e} : e\in E\}
    \end{align}
    with
    \begin{align}
    a \cdot \bar{e} \cdot b = b^* \bar{e} a^* \;\;\;\; \forall a\in A, b \in
        B
    \end{align}
\end{definition}
$E^\circ$ is not a Hilbert bimodule for $(A, B)$ because it doesn't have a
natural $B$-valued inner product. But there is a $A$-valued inner product on
the left $A$-module $E^\circ$ with
\begin{align}
    \langle \bar{e}_1, \bar{e}_2 \rangle = \langle e_2 , e_1 \rangle
    \;\;\;\; e_1, e_2 \in E
\end{align}
and linearity in $A$:
\begin{align}
    \langle a \bar{e}_1, \bar{e}_2 \rangle = a \langle \bar{e}_1, \bar{e}_2
    \rangle \;\;\;\; \forall a \in A.
\end{align}

\begin{MyExercise}
    \textbf{Show that $E^\circ$ is a Hilbert bimodule $(B^{\circ}, A^{\circ})$
    }\newline


    Straightforward show properties of the Hilbert bimodule and its $B^{\circ}$
    valued inner product. Let $\bar{e}_1, \bar{e}_2 \in E^{\circ}$ and $a^\circ \in A,
    b^\circ \in B$. \\
    \begin{align}
        <\bar{e}_1, a^\circ \bar{e}_2> &= <\bar{e}_1, Ja^*J^{-1} \bar{e}_2>=\\
        &= <\bar{e}_1 , J a^* e_2> = \\
        &= <J^{-1} e_1, a^* e_2> =\\
        & = <a^* e_1, e_2>= <J^{-1}(a^\circ)^* J e_1, e_2> = \\
        & = <J^{-1} (a^\circ)^* \bar{e}_1, e_2> =\\
        & = <(a^\circ)^* \bar{e}_1 , \bar{e}_2>.
    \end{align}

    Next $<\bar{e}_1, \bar{e}_2 b^\circ> = <\bar{e}_1, \bar{e_2}> b^\circ$.
    \begin{align}
        <\bar{e}_1, \bar{e}_2 b^\circ>  &= <\bar{e}_1, \bar{e}_2 Jb^*J^{-1}> =\\
        &= <\bar{e}_1, \bar{e_2}> Jb^*J^{-1} = \\
        &= <\bar{e}_1, \bar{e}_2> b^\circ.
    \end{align}
    Then:
    \begin{align}
        (<\bar{e}_1, \bar{e}_2)>_{E^\circ})^* &= (<e_2, e_1>_E)^* =\\
        &= <e_1, e_2>_E^* = <\bar{e}_2, \bar{e}_2>_{E^\circ}
    \end{align}
    And of course $<\bar{e}, \bar{e}> = <e, e> \geq 0$
\end{MyExercise}

\subsubsection{Construction of a Finite Real Spectral Triple from a Finite
Real Spectral Triple}
Given a Hilbert bimodule $E$ for $(B, A)$ we construct a spectral triple
$(B, H', D'; J', \gamma ')$ from $(A, H, D; J, \gamma)$

For the $H'$ we make a $\mathbb{C}$-valued inner product on $H'$ by combining
the $A$ valued inner product on $E$ and $E^\circ$ with the
$\mathbb{C}$-valued inner product on $H$.
\begin{align}
    H' := E\otimes _A H \otimes _A E^\circ
\end{align}

Then the action of $B$ on $H'$ is:
\begin{align}
    b(e_2 \otimes \xi \otimes \bar{e}_2 ) = (be_1) \otimes \xi \otimes
    \bar{e}_2
\end{align}
The right action of $B$ on $H'$ defined by action on the right component
$E^\circ$
\begin{align}
    J'(e_1 \otimes \xi \otimes \bar{e}_2) = e_2 \otimes J \xi \otimes
    \bar{e}_1
\end{align}
with $b^\circ = J' b^* (J')^{-1}$, $b^* \in B$ action on $H'$.
\newline


\newpage
\begin{MyExercise}
    \textbf{ Let $\nabla : E \Rightarrow E \otimes _A \Omega _d^1 (A)$ be a right connection on $E$
    consider the following anti-linear map:
    \begin{align}
        \tau : E \otimes_A \Omega _D^1 (A) &\rightarrow \Omega _D^1 (A) \otimes_A E^\circ\\
                e \otimes \omega &\mapsto -\omega ^* \otimes \bar{e}
    \end{align}
    Show that the map $\bar{\nabla} : E^\circ \righarrow \Omega _D^1(A) \otimes E^\circ$
    with $\bar{\nabla}(\bar{e}) = \tau \circ \nabla(e)$ is a left connection, that means
    show that it satisfied the left Leibniz rule:
    \begin{equation}
        \bar{\nabla}(a\bar{e}) = [D, a] \otimes \bar{e} + a \bar{\nabla}(\bar{e})
    \end{equation}
    }\newline


    Hagime:
    \begin{align}
        &\text{For one:}\\
        &\tau \circ \nabla(ae) = \bar{\nabla}(a\bar{e}) = \bar{\nabla}(a^* \bar{e})\\
        &\text{For two:}\\
         &\tau \circ \nabla(ae) = \tau(\nabla(e)a) + \tau \circ(e \otimes d(a))=\\
         &=a^*\bar{\nabla}(\bar{e}) - d(a)^* \otimes \bar{e}. \\
         &= a^*\bar{\nabla}(\bar{e}) + d(a^*) \otimes \bar{e}.
    \end{align}
\end{MyExercise}
Then the connections
\begin{align}
    &\nabla: E \rightarrow E\otimes _A \Omega _D ^1(A) \\
    &\bar{\nabla}:E^\circ \rightarrow \Omega _D^1(A) \otimes _A E^\circ
\end{align}
give us the Dirac operator on $H' = E \otimes _A H \otimes _A E^\circ$
\begin{align}
    D'(e_1 \otimes \xi \otimes \bar{e}_2) = (\nabla e_1) \xi \otimes
    \bar{e_2}+ e_1 \otimes D\xi \otimes \bar{e}_2 + e_1 \otimes
    \xi(\bar{\nabla}\bar{e}_2)
\end{align}

And the right action of $\omega \in \Omega _D ^1(A)$ on $\xi \in H$ is
defined by
\begin{align}
    \xi \mapsto \epsilon' J \omega ^* J^{-1}\xi
\end{align}

Finally for the grading
\begin{align}
    \gamma ' = 1 \otimes \gamma \otimes 1
\end{align}

\begin{theorem}
    Suppose $(A, H, D; J, \gamma)$ is a finite spectral triple of
    $KO$-dimension $k$, let $\nabla$ be like above satisfying the
    compatibility condition (like with finite spectral triples).

    Then $(B, H',D'; J', \gamma')$ is a finite spectral triple of
    $KO$-Dimension $k$. ($H', D', J', \gamma'$ like above)
\end{theorem}

\begin{proof}
    The only thing left is to check if the $KO$-dimension is preserved,
    for this we check if the $\epsilon$'s are the same.
    \begin{align}
        &(J')^2 = 1 \otimes J^2 \otimes 1 = \epsilon\\
        &J' \gamma '= \epsilon ''\gamma'J'
    \end{align}
    and for $\epsilon '$
    \begin{align}
        J'D'(e_1 \otimes \xi \otimes \bar{e}_2)&=J'((\nabla e_1) \xi \otimes
        \bar{e_2} + e_1 \otimes D\xi \otimes \bar{e}_2 + e_1 \otimes \xi (\tau
        \nabla e_2))\\
        &= \epsilon' D'(e_2 \otimes J\xi \otimes \bar{e}_2)\\
        &= \epsilon' D'J'(e_1 \otimes \xi \bar{e}_2)
    \end{align}
\end{proof}

\end{document}
